%!TEX root =  matrix-est.tex

In this section, we present upper and lower bounds on achievable rates
for estimating the spectral gap as a function of the length of the
sample path $n$.

\subsection{A plug-in based point estimator and its accuracy}
\label{sec:rates-upper}

Our upper bound is demonstrated using estimators based on natural
plug-in estimates.
We also provide an estimator for the minimum stationary probability
in the process, so that one can use the bounds from
\cref{eq:mixing-time-bound} to trap
the mixing time.

Define the random matrix $\wh\vM \in [0,1]^{d \times d}$ and random
vector $\hat\vpi \in \Delta^{d-1}$ by
\begin{align*}
  \wh{M}_{i,j}
  & := \frac{|\{ t \in [n-1] : (X_t,X_{t+1}) = (i,j) \}|}{n-1}
  , \quad i,j \in [d]\,,
  \\
  \hat{\pi}_i
  & := \frac{|\{ t \in [n] : X_t = i \}|}{n}
  , \quad i \in [d]
  \,.
\end{align*}
Furthermore, define
\[
  \Sym(\wh\vL) := \frac12 \parens{ \wh\vL + \wh\vL^\t }
\]
to be the symmetrized version of the (possibly non-symmetric) matrix
\[
  \wh\vL := \Diag(\hat\vpi)^{-1/2} \wh\vM \Diag(\hat\vpi)^{-1/2}
  .
\]
Let $\hat\lambda_1 \geq \hat\lambda_2 \geq \dotsb \geq \hat\lambda_d$
be the eigenvalues of $\Sym(\wh\vL)$.
Our estimator of the minimum stationary probability $\pimin$ is
\[
  \hatpimin := \min_{i \in [d]} \hat\pi_i ,
\]
and our estimator of the spectral gap $\gap$ is
\[
  \hatgap := 1 - \max\{ \hat\lambda_2, |\hat\lambda_d| \} .
\]

These estimators have the following accuracy guarantees:
\begin{theorem}
  \label{thm:err}
  There exists an absolute constant $C>0$ such that the following
  holds.
  Assume the estimators $\hatpimin$ and $\hatgap$ described above are
  formed from a sample path of length $n$ from an ergodic and
  reversible Markov chain.
  Let $\gap>0$ denote the spectral gap and $\pimin>0$ the minimum
  stationary probability.
  For any $\delta \in (0,1)$, with probability at least $1-\delta$,
  \begin{equation}\label{eq:piminbound}
    \Abs{\hatpimin-\pimin}
    \le
    C \,
    \Parens{
      \sqrt{\frac{\pimin\log\frac{d}{\pimin\delta}}{\gap n}}
      +
      \frac{\log\frac{d}{\pimin\delta}}{\gap n}
    }
  \end{equation}
  and
  \begin{equation}\label{eq:gapbound}
    \Abs{\hatgap-\gap}
    \leq
    C \,
    \Parens{
      \sqrt{\frac{\log\frac{d}{\delta}\cdot\log\frac{n}{\pimin\delta}}{\pimin\gap n}}
%      + \frac{\log\frac{d}{\delta}\cdot\log\frac{n}{\pimin\delta}}{\pimin\gap n}  
      + \frac{\log\frac{1}{\gap}}{\gap n}  
    }
    \,.
  \end{equation}
\end{theorem}

\Cref{thm:err} implies that the sequence lengths required to estimate
$\pimin$ and
$\gap$ to within constant multiplicative factors are, respectively,
\[
  \tilde{O}\Parens{\frac1{\pimin\gap}}
  \quad\text{and}\quad
  \tilde{O}\Parens{\frac1{\pimin\gap^3}}
  .
\]

The proof of \cref{thm:err} is based on analyzing the
convergence of the sample averages $\wh{\vM}$ and
$\hat\vpi$ to their expectation, and then using perturbation bounds
for eigenvalues to derive a bound on the error of $\hatgap$.
However, since these averages are formed using a \emph{single sample
path} from a (possibly) non-stationary Markov chain, we cannot use
standard large deviation bounds; moreover applying Chernoff-type
bounds for Markov chains to each entry of $\wh{\vM}$ would result in a
significantly worse sequence length requirement, roughly a factor of
$d$ larger.
%\todoc{In terms of the dependence on the number of states, right? Can we be more specific? Factor of $d$ more, right?}
Instead, we adapt probability tail bounds for sums of independent
random matrices~\citep{tropp2015intro} to our non-iid setting by
directly applying a blocking technique of~\citet{Bernstein27} as
described in the article of~\citet{Yu94}.
Due to ergodicity, the convergence rate can be bounded without any
dependence on the initial state distribution $\vpi^{(1)}$.
The proof of \cref{thm:err} is given in \cref{app:upper}.

\todoc{Turn this paragraph into a remark?}
Note that because the eigenvalues of $\vL$ are the same as that of the
transition probability matrix $\vP$, 
we could have instead opted to
estimate $\vP$, say, using simple frequency estimates obtained from
the sample path, and then computing the second largest eigenvalue of
this empirical estimate $\wh\vP$.
The difficulty with this approach is that $\vP$ lacks the structure
required by certain strong eigenvalue perturbation results.
One could instead invoke the Ostrowski-Elsner theorem
\citep[cf.~Theorem 1.4 on Page 170 of][]{stewart1990matrix}, which
bounds the \emph{matching distance} between the eigenvalues of a
matrix $\vA$ and its perturbation $\vA+\vE$ by $O(\norm{\vE}^{1/d})$.
%\footnote{Actually, this result is not even readily applicable: 
%the Haussdorff distance is actually an optimistic estimate of the deviation between
%$\gap$ and its estimate.}
Since $\norm{\wh\vP-\vP}$ is expected to be of size $O(n^{-1/2})$,
this approach will give a confidence interval for $\gap$ whose width
shrinks at a rate of $O(n^{-1/(2d)})$---an exponential slow-down
compared to the rate from \cref{thm:err}.
As demonstrated through an example from \citet{stewart1990matrix}, the
dependence on the $d$-th root of the norm of the perturbation cannot
be avoided in general.
Our approach based on estimating a symmetric matrix affords us the use
of perturbation results that exploit more structure.
%(In \cref{sec:empirical}, we develop another estimator of
%$\vL$---in order to exploit symmetry---that is based on estimating
%$\vP$ for reasons discussed below.)

Unfortunately, \cref{thm:err} falls short of being directly
usable for obtaining a fully empirical confidence interval for $\gap$
and $\pimin$, at least without further assumptions.
This is because the deviation terms themselves depend inversely both
on $\gap$ and $\pimin$, and hence can never rule out $0$ (or an
arbitrarily small positive value) as a possibility for $\gap$ or
$\pimin$.\footnote{%
  Using \cref{thm:err}, it is possible to trap $\gap$ in the
  union of \emph{two} empirical confidence intervals---one around
  $\hatgap$ and the other around zero, both of which shrink in width
  as the sequence length increases.%
}
In effect, the fact that the Markov chain could be slow mixing and the
long-term frequency of some states could be small makes it difficult
to be confident in the estimates provided by $\hatgap$ and
$\hatpimin$.
This suggests that in order to obtain fully empirical confidence
intervals, we need an estimator that is not subject to such
effects---we pursue this in \cref{sec:empirical}.
\Cref{thm:err} thus primarily serves as a point of comparison
for what is achievable in terms of estimation accuracy when one does
not need to provide empirical confidence bounds.

\subsection{Lower bounds}

We complement \cref{thm:err} with lower bounds showing the
necessity of some of the quantities appearing in the upper bound.
We begin with a result showing that any algorithm for estimating
$\gap$ (even for just constant additive accuracy) requires an
observation sequence of length $\Omega(1/\pimin)$
even when the number of states is only two.

\begin{theorem}
  \label{thm:lb-pimin}
  There is an absolute constant $c>0$ such that the following holds.
  Pick any $\bar\pi \in (0,1/4)$.
  Consider any estimator $\hatgap$ that takes as input a random sample
  path of length $n < c/\bar\pi$ from a Markov chain starting from any
  desired initial state distribution.
  There exists a two-state ergodic and reversible Markov chain
  distribution with spectral gap $\gap \geq 1/2$ and minimum
  stationary probability $\pimin \geq \bar\pi$ such that
  \[
    \Pr\Brackets{ |\hatgap - \gap| \geq 1/6 } \geq 1/8 .
  \]
\end{theorem}

Next, we show that in order to estimate
$\gap$ to within a constant multiplicative accuracy in a $d$ state chain,
a sequence of length $\Omega(d\log(d)/\gap)$ is required.
Essentially, the sequence may have to visit all $d$ states $\log(d)/\gap$ times each on average.
This hold \emph{even} if $\pimin$ is within a factor of two of the
\emph{largest} possible value of $1/d$ that it can take, i.e., when
$\vpi$ is nearly uniform.
\begin{theorem}
  \label{thm:lb-gap}
  There is an absolute constant $c>0$ such that the following holds.
  Pick any positive integer $d \geq 3$ and any $\bar\gamma \in
  (0,1/2)$.
  Consider any estimator $\hatgap$ that takes as input a random sample
  path of length $n < c d\log(d) / \bar\gamma$ from a $d$-state
  reversible Markov chain starting from any desired initial state
  distribution.
  There is an ergodic and reversible Markov chain distribution
  with spectral gap $\gap \in [\bar\gamma,2\bar\gamma]$ and minimum
  stationary probability $\pimin \geq 1/(2d)$ such that
  \[
    \Pr\Brackets{ |\hatgap - \gap| \geq \bar\gamma} \geq 1/4 .
  \]
\end{theorem}

The proofs of \cref{thm:lb-pimin,thm:lb-gap} are given in
\cref{app:lower}.

