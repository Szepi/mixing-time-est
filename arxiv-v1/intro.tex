This work
tackles the challenge of constructing
fully empirical bounds on the
mixing time of Markov chains based on a single sample path.
Let $(X_t)_{t=1,2,\dotsc}$ be an irreducible, aperiodic
time-homogeneous Markov chain on a finite state space $[d] :=
\{1,2,\dotsc,d\}$ with transition matrix $\vP$.
Under this assumption, the chain converges to its unique stationary
distribution $\vpi =
(\pi_i)_{i=1}^d$ regardless of the initial state distribution $\vq$: 
\[
  \lim_{t\to\infty} {\Pr}_{\vq}\Parens{X_t = i}
  = \lim_{t\to\infty} (\vq \vP^t)_i = \pi_i
  \quad \text{for each $i \in [d]$} .
\]
The \emph{mixing time} $\tmix$ of the Markov chain is the
number of time steps required
for the chain to
be within a fixed threshold of
its stationary
distribution:
\begin{align}
\label{eq:mixtimedef}
  \tmix
  :=
  \min\Braces{
    t \in \bbN :
    \sup_{\vq}
    \max_{A\subset [d]}
    \Abs{
      \textstyle\Pr_{\vq}\Parens{ X_t \in A } - \vpi(A)
    }
    \leq 1/4
  }\,.
\end{align}
Here, $\vpi(A) = \sum_{i\in A} \pi_i$ is the probability assigned to
set $A$ by $\vpi$, and the supremum is over all possible initial
distributions $\vq$.
The problem studied in this work is the construction of a non-trivial
confidence interval $C_n = C_n(X_1,X_2,\dotsc,X_n,\delta) \subset
[0,\infty]$, based only on the observed sample path
$(X_1,X_2,\dotsc,X_n)$ and $\delta \in (0,1)$, that
succeeds with probability $1-\delta$
in trapping
the value of the
mixing time $\tmix$.

This problem is motivated by the numerous scientific applications and
machine learning tasks in which
the quantity of interest is
the
mean $\vpi(f) = \sum_i \pi_i f(i)$
for some function $f$ of the states of
a Markov chain.
This is the setting of the celebrated Markov Chain Monte Carlo (MCMC)
paradigm~\cite{liu2001monte}, but the problem also arises in
performance prediction involving time-correlated data, as is common in
reinforcement learning~\cite{sutton98}.
Observable bounds on mixing times are useful in the design and
diagnostics of these methods; they yield effective approaches to
assessing the estimation quality, even when \emph{a priori} knowledge
of the mixing time or correlation structure is unavailable.

\paragraph{Main results.}
We develop the first procedure for constructing non-trivial
and fully empirical confidence intervals for Markov mixing time.
Consider a reversible ergodic Markov chain on $d$ states with absolute
spectral gap $\gap$ and stationary distribution minorized by $\pimin$. 
As is well-known \citep[Theorems~12.3 and~12.4]{LePeWi08},
\begin{equation}
  \label{eq:mixing-time-bound}
  \Parens{\mfrac1{\gap} - 1} \ln2
  \ \leq \ \tmix
  \ \leq \ \mfrac1{\gap} \ln \mfrac4{\pimin}
  ,
\end{equation}
and hence it suffices to estimate $\gap$ and $\pimin$. 
Our main results are summarized as follows.
\begin{enumerate}
  \item
    In \cref{sec:rates-lower}, we show that in some problems 
    $n = \Omega\parens{ (d\log d)/\gap + 1/\pimin}$ observations are necessary for any procedure
    to guarantee constant multiplicative accuracy in estimating $\gap$
    (\cref{thm:lb-pimin,thm:lb-gap}).
    Essentially, in some problems \emph{every} state may need to be visited about
    $\log(d)/\gap$ times, on average, before an accurate estimate of
    the mixing time can be provided, regardless of the actual estimation
    procedure used.

  \item
    In \cref{sec:rates-upper}, we give a point-estimator for $\gap$,
    and prove in \cref{thm:err} that it achieves multiplicative
    accuracy from \emph{a single sample path} of length
    $\tilde{O}\parens{1/(\pimin\gap^3)}$.\footnote{The
    $\tilde{O}(\cdot)$ notation suppresses logarithmic factors.}
    We also provide a point-estimator for $\pimin$ that requires a sample
    path of length $\tilde{O}\parens{1/(\pimin\gap)}$.
    This establishes the feasibility of \emph{estimating} the mixing
    time in this setting.
    However, the valid confidence intervals suggested by
    \cref{thm:err} depend on the unknown quantities $\pimin$ and
    $\gap$.
    We also discuss the importance of reversibility, and some possible
    extensions to non-reversible chains.

  \item
    In \cref{sec:empirical}, the construction of
    valid \emph{fully empirical confidence intervals} for $\pimin$ and
    $\gap$ are considered. First, the difficulty of the task is explained,
    i.e., why the standard approach of turning the finite time confidence intervals of
    \cref{thm:err} into a fully empirical one fails.
	Combining several results from perturbation theory in a novel fashion
	we propose a new procedure and prove that it avoids slow convergence
    (\cref{thm:empirical}).
    We also explain how to combine the empirical confidence intervals
    from \cref{alg:empest} with the non-empirical bounds from
    \cref{thm:err} to produce valid empirical confidence intervals.
    We prove in \cref{thm:combined} that the width of these new
    intervals converge to zero asymptotically at least as fast as
    those from either \cref{thm:err} and \cref{thm:empirical}.

\end{enumerate}

\paragraph{Related work.}
There is a vast statistical literature on estimation in Markov chains.
For instance, it is known that under the assumptions on
$(X_t)_t$ from above, the law of large numbers guarantees that
the sample mean $\vpi_n(f) \defeq \frac1n \sum_{t=1}^n f(X_t)$
converges almost surely to $\vpi(f)$~\cite{meyn1993markov}, while the
central limit theorem tells us that as $n\to \infty$, the distribution
of the deviation $\sqrt{n}( \vpi_n(f)-\vpi(f))$ will be normal with
mean zero and asymptotic variance $\lim_{n\to\infty}
n\Var\Parens{\vpi_n(f)}$~\cite{kipnis1986central}.

Although these asymptotic results help us understand the limiting
behavior of the sample mean over a Markov chain, they say little about
the finite-time non-asymptotic behavior, which is often needed for the
prudent evaluation of a method or even its algorithmic design
\cite{
MCMCDiscussion93%
,DBLP:conf/valuetools/KontoyiannisLM06%
,BBL06%
,MniSzeAu08%
,MauPo09%
,LiLiWaSt11:KWIK%
,flegal2011implementing%
,Gyori-paulin15%
,SwaJoa15:LoggedBandit%
}.
To address this need, numerous works have developed Chernoff-type
bounds on $\Pr\parens{ |\vpi_n(f)-\vpi(f)| > \eps }$, thus providing
valuable tools for non-asymptotic probabilistic
analysis~\cite{gillman1998chernoff,leon2004optimal,DBLP:conf/valuetools/KontoyiannisLM06,
paulin15}.
These probability bounds are larger than corresponding bounds for
independent and identically distributed (iid) data due to the temporal
dependence; intuitively, for the Markov chain to yield a fresh draw
$X_{t'}$ that behaves as if it was independent of $X_t$, one must wait
$\Theta(\tmix)$ time steps.
Note that the bounds generally depend on distribution-specific
properties of the Markov chain (e.g., $\vP$, $\tmix$, $\gap$), which are often
unknown \emph{a priori} in practice.
Consequently, much effort has been put towards estimating these
unknown quantities, especially in the context of MCMC diagnostics, in
order to provide data-dependent assessments of estimation
accuracy~\cite[e.g.,][]{MCMCDiscussion93,GaSmi00:eigval,jones2001,flegal2011implementing,1209.0703,Gyori-paulin15}.
However, these approaches generally only provide asymptotic
guarantees, and hence fall short of our goal of empirical bounds that
are valid with any finite-length sample path.

Learning with dependent data is another main motivation to our work.
Many results from statistical learning and empirical process theory
have been extended to sufficiently fast mixing, dependent
data~\citep[e.g.,][]{Yu94,MR1921877,gamarnik03,MoRo08,MoRo09,DBLP:conf/nips/SteinwartC09,Steinwart2009175},
providing learnability assurances (e.g., generalization error bounds).
These results are often given in terms of mixing coefficients, which
can be consistently estimated in some cases~\citet{McDoShaSche11}.
However, the convergence rates of the estimates
from~\citet{McDoShaSche11}, which are needed to derive confidence
bounds, are given in terms of unknown mixing coefficients.
When the data comes from a Markov chain, these mixing coefficients can
often be bounded in terms of mixing times, and hence our main results
provide a way to make them fully empirical, at least in the limited setting we study.

It is possible to eliminate many of the difficulties presented above
when allowed more flexible access
to the Markov chain.
For example, given a sampling oracle that generates
independent transitions from any given state (akin to a ``reset''
device), the mixing time becomes an efficiently testable property in
the sense studied in~\citet{BaFoRuSmiWhi00,BaFoRuSmiWhi13}.
On the other hand, when one only has a circuit-based description of
the transition probabilities of a Markov chain over an
exponentially-large state space, there are complexity-theoretic
barriers for many MCMC diagnostic problems~\citet{BhaBoMo11}.

