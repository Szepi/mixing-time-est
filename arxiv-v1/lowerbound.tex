%!TEX root =  matrix-est.tex

%\subsection{A two-state Markov chain}
%
%Fix $\lambda_2' \in (0,1)$ (and $\gamma' := 1-\lambda_2' \in (0,1)$)
%and $\vpi' \in \Int\Delta^1$.
%Define the stochastic matrix $\vP' \in [0,1]^{2 \times 2}$ where
%\begin{align*}
%  P_{1,2}' & := \frac{\gamma'}{1 + \pi_1'/\pi_2'} , &
%  P_{2,1}' & := \frac{\gamma'}{1 + \pi_2'/\pi_1'} .
%\end{align*}
%Then it can be checked that $\vpi'\vP' = \vpi'$ and the second-largest
%eigenvalue of $\vL' = \Diag(\vpi')^{1/2}\vP'\Diag(\vpi')^{-1/2}$ is
%$\lambda_2'$.

%\subsection{Sequences of length $1/\pimin$ are required}

%We begin with a result showing that
%any algorithm for estimating $\gap$
%requires an observation sequence of length $\Omega(1/\pimin)$.
%
%\begin{theorem}
%  \label{thm:lb-pimin}
%  Pick any $\delta \in (0,1/2)$.
%  Consider any estimator $\hatgap$ of the spectral gap that takes as
%  input a random sample path of length $n < 1/(2\pimin)$ from a
%  reversible Markov chain starting from the stationary distribution.
%  There exists a $2$-state reversible Markov chain distribution with
%  unique stationary distribution such that
%  \[
%    \Pr\Brackets{ |\hatgap - \gap| \geq 1-2\delta } \geq \frac14 .
%  \]
%\end{theorem}

%%\begin{proof}[Proof of Theorem~\ref{thm:lb-pimin}]
%\noindent \textbf{Proof of Theorem~\ref{thm:lb-pimin}.}
%  Fix $\bar\pi \in (0,1/4)$, and consider two Markov chains given by
%  the following stochastic matrices:
%  \[
%    \vP^{(1)} :=
%    \begin{bmatrix}
%      1-\bar\pi & \bar\pi \\
%      1-\bar\pi & \bar\pi
%    \end{bmatrix}
%    , \quad
%    \vP^{(2)} :=
%    \begin{bmatrix}
%      1-\bar\pi & \bar\pi \\
%      \bar\pi & 1-\bar\pi
%    \end{bmatrix}
%    .
%  \]
%  Both Markov chains are ergodic and reversible; their stationary
%  distributions are, respectively, $\vpi^{(1)} = (1-\bar\pi,\bar\pi)$
%  and $\vpi^{(2)} = (1/2,1/2)$.
%  We have $\pimin \geq \bar\pi$ in both cases.
%  For the first Markov chain, $\slem = 0$, and hence the spectral
%  gap is $1$; for the second Markov chain, $\slem = 1-2\bar\pi$, so
%  the spectral gap is $2\bar\pi$.
%
%  In order to guarantee $|\hatgap - \gap| < 1-2\bar\pi$, it must be
%  possible to distinguish the above two Markov chains.
%  In either $\vP^{(1)}$ or $\vP^{(2)}$, with probability at least
%  half, the initial state is $1$; and both chains have the same
%  transition probabilities from state $1$.
%  Therefore, the chains are indistinguishable less the sample path
%  eventually reaches state $2$.
%  But with probability at least $1/4$, a sample path of length $n <
%  1/(2\bar\pi)$ starting from state $1$ always remains in the same
%  state.
%\hfill $\blacksquare$
%%\end{proof}

\begin{proof}[Proof of \cref{thm:lb-pimin}]
  Fix $\bar\pi \in (0,1/3)$, and set $\veps := 2\bar\pi/(1-\bar\pi)$.
  Consider two Markov chains given by the following stochastic
  matrices:
  \[
    \vP^{(1)} :=
    \begin{bmatrix}
      1-\veps & \veps \\
      1-\veps & \veps
    \end{bmatrix}
    , \quad
    \vP^{(2)} :=
    \begin{bmatrix}
      1-\veps & \veps \\
      1/2 & 1/2
    \end{bmatrix}
    .
  \]
  Each Markov chain is ergodic and reversible; their stationary
  distributions are, respectively, $\vpi^{(1)} = (1-\veps,\veps)$
  and $\vpi^{(2)} = (\veps/(2+\veps),2/(2+\veps))$.
  We have $\pimin \geq \bar\pi$ in both cases.
  For the first Markov chain, $\slem = 0$, and hence the spectral gap
  is $1$; for the second Markov chain, $\slem = 1/2-\veps$, so the
  spectral gap is $1/2+\veps$.

  In order to guarantee $|\hatgap - \gap| < 1/6 < 1/2-\veps$, it must
  be possible to distinguish the two Markov chains.
  Assume that the initial state distribution has mass at least $1/2$
  on state $1$.
  (If this is not the case, we swap the roles of states $1$ and $2$ in
  the constructions above.)
  With probability at least half, the initial state is $1$; and both
  chains have the same transition probabilities from state $1$.
  The chains are indistinguishable less the sample path eventually
  reaches state $2$.
  But with probability at least $1/4$, a sample path of length $n <
  c/\bar\pi = c(2+\veps)/\veps$ starting from state $1$ always
  remains in the same state.
\end{proof}


%\subsection{Sequences of length $d\log(d)/\gap$ are required}

%Next, we show that in order to estimate
%$\gap$ to within a constant multiplicative accuracy,
%a sequence of length $\Omega(d\log(d)/\gap)$ is required.
%
%\begin{theorem}
%  \label{thm:lb-gap}
%  There exists absolute constants $C, c>0$ such that the following
%  holds.
%  Pick any positive integer $d \geq C$.
%  Consider any estimator $\hatgap$ of the spectral gap that takes as
%  input a random sample path of length $n < c d\log(d) / \gap$ from a
%  $d$-state reversible Markov chain starting from an arbitrary initial
%  state.
%  For any initial state, there exists a reversible Markov chain
%  distribution with unique stationary distribution such that
%  \[
%    \Pr\Brackets{ |\hatgap - \gap| \geq \frac12\gap} \geq \frac14 .
%  \]
%\end{theorem}

\begin{proof}[Proof of \cref{thm:lb-gap}]
  We consider $d$-state Markov chains of the following form:
  \[
    P_{i,j} =
    \begin{cases}
      1-\veps_i & \text{if $i = j$} ; \\
      \displaystyle\frac{\veps_i}{d-1} & \text{if $i \neq j$}
    \end{cases}
  \]
  for some $\veps_1, \veps_2, \dotsc, \veps_d \in (0,1)$.
  Such a chain is ergodic and reversible, and its unique stationary
  distribution $\vpi$ satisfies
  \[
    \pi_i = \frac{1/\veps_i}{\sum_{j=1}^d 1/\veps_j}
    .
  \]
  We fix $\veps := \frac{d-1}{d/2}\bar\gamma$ and set $\veps' :=
  \frac{d/2-1}{d-1} \veps < \veps$.
  Consider the following $d+1$ different Markov chains of the type
  described above:
  \begin{itemize}
    \item
      $\vP^{(0)}$: $\veps_1 = \dotsb = \veps_d = \veps$.
      For this Markov chain, $\lambda_2 = \lambda_d = \slem =
      1-\frac{d}{d-1}\veps$.

    \item
      $\vP^{(i)}$ for $i \in [d]$: $\veps_j = \veps$ for $j \neq i$,
      and $\veps_i = \veps'$.
      For these Markov chains, $\lambda_2 = 1 - \veps' -
      \frac{1}{d-1}\veps = 1 - \frac{d/2}{d-1} \veps$,
      and $\lambda_d = 1 - \frac{d}{d-1} \veps$.
      So $\slem = 1-\frac{d/2}{d-1} \veps$.

  \end{itemize}
  The spectral gap in each chain satisfies $\gap \in
  [\bar\gamma,2\bar\gamma]$; in $\vP^{(i)}$ for $i \in [d]$, it is half
  of what it is in $\vP^{(0)}$.
  Also $\pi_i \geq 1/(2d)$ for each $i \in [d]$.

  In order to guarantee $|\hatgap - \gap| \leq \bar\gamma$, it must
  be possible to distinguish $\vP^{(0)}$ from each $\vP^{(i)}$,
  $i\in[d]$.
  But $\vP^{(0)}$ is identical to $\vP^{(i)}$ except for the transition
  probabilities from state $i$.
  Therefore, regardless of the initial state, the sample path must
  visit all states in order to distinguish $\vP^{(0)}$ from each
  $\vP^{(i)}$, $i \in [d]$.
  For any of the $d+1$ Markov chains above, the earliest time in which
  a sample path visits all $d$ states
  stochastically dominates a generalized coupon collection time $T = 1 +
  \sum_{i=1}^{d-1} T_i$, where $T_i$ is the number of steps required to
  see the $(i+1)$-th distinct state in the sequence beyond the first
  $i$.
  The random variables $T_1,T_2,\dotsc,T_{d-1}$ are independent, and are
  geometrically distributed, $T_i \sim \Geom(\veps -
  (i-1)\veps/(d-1))$.
  We have that
  \[
    \bbE[T_i] = \frac{d-1}{\veps(d-i)} , \quad
    \var(T_i) = \frac{1 - \veps \frac{d-i}{d-1}}{%
      \Parens{ \veps \frac{d-i}{d-1} }^2
    }
    .
  \]
  Therefore
  \[
    \bbE[T] = 1 + \frac{d-1}{\veps} H_{d-1} , \quad
    \var(T) \leq \Parens{ \frac{d-1}{\veps} }^2 \frac{\pi^2}{6} 
  \]
  where $H_{d-1} = 1 + 1/2 + 1/3 + \dotsb + 1/(d-1)$.
  Since $n < c d\log(d) / \bar\gamma$, with probability at least $1/4$,
  the sample path does not visit all $d$ states.
\end{proof}


