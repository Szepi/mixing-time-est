%!TEX root =  matrix-est.tex
\label{sec:setting}
Let $\vP \in (\Delta^{d-1})^d \subset [0,1]^{d \times d}$ be a $d
\times d$ row-stochastic matrix for an ergodic (i.e., irreducible and
aperiodic) Markov chain.
This implies there is a unique stationary distribution $\vpi \in
\Delta^{d-1}$ with $\pi_i > 0$ for all $i \in [d]$~\citep[Corollary
1.17]{LePeWi08}.
We also assume that $\vP$ is \emph{reversible} (with respect to
$\vpi$):
\begin{align}
\label{eq:reversibility}
  \pi_i P_{i,j} = \pi_j P_{j,i} ,
  \quad i,j \in [d] .
\end{align}
The minimum stationary probability is denoted by $\pimin := \min_{i
\in [d]} \pi_i$.

Define the matrices
\begin{align*}
\vM := \Diag(\vpi) \vP \quad \text{and} \quad
\vL := \Diag(\vpi)^{-1/2} \vM \Diag(\vpi)^{-1/2}\,.
\end{align*}
The $(i,j)$th entry of the matrix $M_{i,j}$ contains the \emph{doublet
probabilities} associated with $\vP$: $M_{i,j} = \pi_i P_{i,j}$ is the
probability of seeing state $i$ followed by state $j$ when the chain
is started from its stationary distribution.
The matrix $\vM$ is symmetric on account of the reversibility of
$\vP$, and hence it follows that $\vL$ is also symmetric.
Further, $\vL = \Diag(\vpi)^{1/2} \vP \Diag(\vpi)^{-1/2}$, hence $\vL$
and $\vP$ are similar and thus their eigenvalue systems are identical.
Ergodicity and reversibility imply that the eigenvalues of $\vL$ are
contained in the interval $\opencloseint{-1,1}$, and that $1$ is an
eigenvalue of $\vL$ with multiplicity $1$~\citep[Lemmas 12.1 and
12.2]{LePeWi08}.
Denote and order the eigenvalues of $\vL$ as
\[
  1 = \lambda_1 > \lambda_2 \geq \dotsb \geq \lambda_d > -1 .
\]
Let $\slem := \max\{ \lambda_2,\, |\lambda_d| \}$, and define the
(absolute) spectral gap to be $\gap := 1-\slem$, which is strictly positive.
%Throughout the paper, both $\pimin$ and $\gap$ are assumed
%to be positive.

Let $(X_t)_{t\in\bbN}$ be a Markov chain whose transitions are governed by
$\vP$.
For each $t \in \bbN$, let $\vpi^{(t)} \in \Delta^{d-1}$ denote the
marginal distribution of $X_t$, so
\[
  \vpi^{(t+1)} = \vpi^{(t)} \vP ,
  \quad t \in \bbN .
\]
Note that the initial distribution $\vpi^{(1)}$ is arbitrary,
and need not be the stationary distribution $\vpi$.

The goal is to estimate $\pimin$ and $\gap$ from the length $n$ sample
path $(X_t)_{t \in [n]}$, and also to construct fully empirical
confidence intervals that $\pimin$ and $\gap$ with high probability;
in particular, the construction of the intervals should not depend on
any unobservable quantities, including $\pimin$ and $\gap$ themselves.
It is well-known that the \emph{mixing time} of the Markov chain $\tmix$ (defined in the introduction)
\hide{,
\[
  \tmix
  :=
  \min\Braces{
    t \in \bbN :
    \sup_{\vq \in \Delta^{d-1}}
    \Norm{
      \vq \vP^t - \vpi
    }_{\tv} \leq 1/4
  }
  ,
\]}
can be bounded in terms of $\pimin$ and $\gap$~\citep[Theorems~12.3
and~12.4]{LePeWi08}:
\[
  \Parens{\frac1{\gap} - 1} \ln2
  \ \leq \ \tmix
  \ \leq \ \frac1{\gap} \log \frac4{\pimin}
  .
\]
Moreover, convergence rates for empirical processes on Markov chain
sequences are also often given in terms of mixing coefficients that
can ultimately be bounded in terms of $\pimin$ and
$\gap$~\citep[e.g.,][]{MR1921877,MoRo08,MoRo09}. \todoc{Cite one? Btw, do these really have these results?}
Therefore, valid confidence intervals for $\pimin$ and $\gap$ can be
used to make these rates fully observable.

%Therefore, confidence intervals for $\pimin$ and $\gap$ can be used to
%make these rates data-dependent and hence usable in procedures such
%as Structural Risk Minimization.
%\todoc{citation for SRM, and other applications?}
%
%Additionally, we aim to provide
%asymptotic convergence rates
%as well as data-dependent confidence intervals;
%the latter should not depend on any unknown quantities
%(such as $\pimin$ and $\gap$).
%
%\todoc[inline]{What is the data? What is the problem?}

