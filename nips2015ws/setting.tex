\paragraph{Notations.}
%\label{sec:notation}
%
We denote the set of positive integers by $\bbN$, and
the set of the first $d$ positive integers $\{1,2,\dotsc,d\}$ by $\iset{d}$.
The non-negative part of a real number $x$ is $[x]_+ := \max\{0,x\}$.
We use $\ln(\cdot)$ in expressions with exact constants, while we use $\log(\cdot)$ elsewhere.
Boldface symbols are used for vectors and matrices (e.g., $\vv$,
$\vM$), and their entries are referenced by subindexing (e.g., $v_i$,
$M_{i,j}$).
For a vector $\vv$, $\norm{\vv}$ denotes its Euclidean norm; for a
matrix $\vM$, $\norm{\vM}$ denotes its spectral norm.
We use $\Diag(\vv)$ to denote the diagonal matrix whose $(i,i)$-th
entry is $v_i$.
The probability simplex is denoted by $\Delta^{d-1} = \{ \vp
\in [0,1]^d : \sum_{i=1}^d p_i = 1 \}$, and we regard vectors in
$\Delta^{d-1}$ as row vectors.

\paragraph{Setting.}
%\label{sec:setting}
%
Let $\vP \in (\Delta^{d-1})^d \subset [0,1]^{d \times d}$ be a $d
\times d$ row-stochastic matrix for an ergodic (i.e., irreducible and
aperiodic) Markov chain.
This implies there is a unique stationary distribution $\vpi \in
\Delta^{d-1}$ with $\pi_i > 0$ for all $i \in [d]$~\citep[Corollary
1.17]{LePeWi08}.
We also assume that $\vP$ is \emph{reversible} (with respect to
$\vpi$):
$\pi_i P_{i,j} = \pi_j P_{j,i}$ for all $i,j \in [d]$.
%\begin{align}
%\label{eq:reversibility}
%  \pi_i P_{i,j} = \pi_j P_{j,i} ,
%  \quad i,j \in [d] .
%\end{align}
The minimum stationary probability is denoted by $\pimin := \min_{i
\in [d]} \pi_i$.

Define the matrices
$\vM := \Diag(\vpi) \vP$ and
$\vL := \Diag(\vpi)^{-1/2} \vM \Diag(\vpi)^{-1/2}$.
%\begin{align*}
%\vM := \Diag(\vpi) \vP \quad \text{and} \quad
%\vL := \Diag(\vpi)^{-1/2} \vM \Diag(\vpi)^{-1/2}\,.
%\end{align*}
The $(i,j)$th entry of the matrix $M_{i,j}$ contains the \emph{doublet
probabilities} associated with $\vP$: $M_{i,j} = \pi_i P_{i,j}$ is the
probability of seeing state $i$ followed by state $j$ when the chain
is started from its stationary distribution.
The matrix $\vM$ is symmetric on account of the reversibility of
$\vP$, and hence it follows that $\vL$ is also symmetric.
%(We will strongly exploit the symmetry in our results.)
Further, $\vL = \Diag(\vpi)^{1/2} \vP \Diag(\vpi)^{-1/2}$, hence $\vL$
and $\vP$ are similar and thus their eigenvalue systems are identical.
Ergodicity and reversibility imply that the eigenvalues of $\vL$ are
contained in the interval $\opencloseint{-1,1}$, and that $1$ is an
eigenvalue of $\vL$ with multiplicity $1$~\citep[Lemmas 12.1 and
12.2]{LePeWi08}.
Denote and order the eigenvalues of $\vL$ as
$1 = \lambda_1 > \lambda_2 \geq \dotsb \geq \lambda_d > -1$.
%\[
%  1 = \lambda_1 > \lambda_2 \geq \dotsb \geq \lambda_d > -1 .
%\]
Let $\slem := \max\{ \lambda_2,\, |\lambda_d| \}$, and define the
(absolute) spectral gap to be $\gap := 1-\slem$, which is strictly
positive on account of ergodicity.

Let $(X_t)_{t\in\bbN}$ be a Markov chain whose transition
probabilities are governed by $\vP$.
For each $t \in \bbN$, let $\vpi^{(t)} \in \Delta^{d-1}$ denote the
marginal distribution of $X_t$, so
$\vpi^{(t+1)} = \vpi^{(t)} \vP$ for all $t \in \bbN$.
%\[
%  \vpi^{(t+1)} = \vpi^{(t)} \vP ,
%  \quad t \in \bbN .
%\]
Note that the initial distribution $\vpi^{(1)}$ is arbitrary,
and need not be the stationary distribution $\vpi$.

The goal is to estimate $\pimin$ and $\gap$ from the length $n$ sample
path $(X_t)_{t \in [n]}$, and also to construct fully empirical
confidence intervals that $\pimin$ and $\gap$ with high probability;
in particular, the construction of the intervals should not depend on
any unobservable quantities, including $\pimin$ and $\gap$ themselves.
As mentioned in the introduction,
it is well-known that the \emph{mixing time} of the Markov chain
$\tmix$ (defined in Eq.~\ref{eq:mixtimedef})
is bounded in terms of $\pimin$ and $\gap$, as shown in
\cref{eq:mixing-time-bound}.
Moreover, convergence rates for empirical processes on Markov chain
sequences are also often given in terms of mixing coefficients that
can ultimately be bounded in terms of $\pimin$ and
$\gap$ (as we will show in the proof of our first result).
Therefore, valid confidence intervals for $\pimin$ and $\gap$ can be
used to make these rates fully observable.

