%!TEX root =  matrix-est.tex
We denote the set of positive integers by $\bbN$, and
the set of the first $d$ positive integers $\{1,2,\dotsc,d\}$ by $\iset{d}$.
The non-negative part of a real number $x$ is $[x]_+ := \max\{0,x\}$.
We will use $\ln(x)$, to denote the natural logarithm of $x$
in expressions with exact numerical constants,
while we use $\log(x)$ elsewhere.
Boldface symbols are used for vectors and matrices (e.g., $\vv$,
$\vM$), and their entries are referenced by subindexing (e.g., $v_i$,
$M_{i,j}$).
%The $i$-th coordinate basis vector is denoted by $\ve_i$.
For a vector $\vv$, $\norm{\vv}$ denotes its Euclidean norm; for a
matrix $\vM$, $\norm{\vM}$ denotes its spectral norm.
We use $\Diag(\vv)$ to denote the diagonal matrix whose $(i,i)$-th
entry is $v_i$.
The probability simplex is denoted by $\Delta^{d-1} = \{ \vp
\in [0,1]^d : \sum_{i=1}^d p_i = 1 \}$, and we regard vectors in
$\Delta^{d-1}$ as row vectors.
%Finally, the total variation between two probability distributions
%$\vp, \vq \in \Delta^{d-1}$ is denoted by $\norm{\vp-\vq}_{\tv}$.
