%!TEX root =  matrix-est.tex
This article provides the first efficient, non-heuristic procedure for computing a fully data dependent interval
that traps the mixing time of a finite reversible ergodic Markov chain with a prescribed probability.
The interval is computed from a single finite-length sample
path from the Markov chain, and does not require the 
knowledge of any parameters of the chain.
%bounds on the mixing time itself.
This stands in contrast to previous approaches, which either 
only provide point estimates, or require a
reset mechanism, or additional prior knowledge.
%Most notably, 
%the procedure produces a fully empirical confidence
%interval (i.e., it depends only on the sample path and a given
%confidence parameter $\delta$) that is always valid with probability
%$1-\delta$; and 
The width of the interval converges to zero at a
$\sqrt{n}$ rate, where $n$ is the length of the sample path.
Upper and lower bounds are given on the number of samples required
to achieve a constant accuracy estimate in terms of the chain's parameters,
in particular, based on the absolute spectral gap $\gap$.
\todoc{We may need to add some explanation to connect $\gap$ estimates
to $\tmix$ estimates.}
Our lower bounds indicate that, 
unless further restrictions are placed on the chain,
no procedure can achieve a constant
accuracy before seeing each state at least  $\Omega(1/\gap)$ times.
Future directions of research are identified.
\todoc{Such as: closing the gap. Extension to large state spaces, 
based on additional assumptions. Even if $\gap$ is large, we need many samples.
Maybe we need even stronger assumptions, like the decay rate of eigenvalues.
}
%If $\gap$ denotes the absolute spectral gap of the unknown Markov chain
%and the chain's stationary distribution is minorized by $\pimin$,
%our procedure requires $\tilde{O}( \frac{1}{\pimin\gap^3})$ observations
% to estimate the mixing time up to a constant multiplicative accuracy, 
%while we also establish that any algorithm to achieve the same
%goal would need $\Omega( d\log(d)/\gap + 1/\pimin)$ samples. 

%and the dependence on the main  chain's 

%We provide confidence guarantees that are fully data-dependent in the sense that they yield non-trivial
%finite-sample deviation bounds without any knowledge of the mixing coefficients.
%
%
%The novelty of our approach lies in several key aspects.
%Crucially,
%we do not assume any a priori bounds
%on the mixing time,
%which
%sets our approach apart from
%most others that do
%require some advance knowledge of this quantity.
%Another
%advantage of our result over
%common practice is the sufficiency of a single long sampling sequence,
%without access to a restart mechanism.
%Perhaps most importantly, we overcome an obstacle inherent in the chicken-and-egg nature of
%estimating mixing coefficients: to give confidence bounds one typically needs to know the mixing time.
%We provide confidence guarantees that are fully data-dependent in the sense that they yield non-trivial
%finite-sample deviation bounds without any knowledge of the mixing coefficients.
%%Upper and lower sample complexity bounds are provided and challenging open problems are posed.
%%\todod{I would remove the last sentence, at least the part about
%%``challenging'' open problems.}
%%%test
