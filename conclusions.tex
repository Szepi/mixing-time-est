%!TEX root =  matrix-est.tex
\todoc{Theorem 5 more generally, log}

The algorithm from Section~\ref{sec:empirical} demonstrates the
feasibility of fully empirical finite-sample confidence bounds on
Markov chain mixing time.
This makes it possible to apply learning methods that require
empirical bounds to many non-iid settings that are commonplace in many
applications.
%One limitation of our techniques is that they are limited to Markov
%chains over finite state spaces.
We expect that future results will generalize and tighten our
analysis, possibly extend it to large (or infinite) state spaces by
adding extra assumptions.

A salient limitation of our approach is the reversibility assumption.
Without the latter, one could still obtain non-trivial finite-sample
estimates.
One possible starting point is to work with a pointwise estimate $\wh
P_{ij}$ and invoke the Ostrowski-Elsner theorem
\citep{stewart1990matrix} to bound the deviation between the empirical
$\hatgap$ and the true $\gap$.
Unfortunately, this bound has an exponential dependence on $d$.
Another problem we leave open for now is to close the gap between the
upper bound in Theorem~\ref{thm:err} and the lower bounds in
Theorems~\ref{thm:lb-pimin} and \ref{thm:lb-gap}. 
\todoc{Reduce space complexity? Assume more structure of $\vP$?
Sometimes $\gap$ is known, but $\pimin$ is not. Does this help?
How about the other way around?
}
%\todoc{We need to move the references needed in the appendix to after the appendix. 
%This should be possible with some latex package for splitting references..}
%and investigate
%whether the width of the fully empirical confidence interval can be
%shortened to match the width of the theoretical confidence interval.

%The upshot is that now the various Markov PAC learning results stated in
%terms of unknown mixing coefficients
%\citep{gamarnik03,MoRo09}
%can be applied to give nontrivial data-dependent bounds,
%albeit in a limited setting.
