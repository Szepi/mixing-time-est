\documentclass[11pt]{article}

\usepackage[destlabel]{hyperref}
\usepackage[margin=1in,letterpaper]{geometry}

\usepackage{amsmath,amsbsy,amsfonts,amssymb,amsthm,color,dsfont,mleftright,commath}

\def\ddefloop#1{\ifx\ddefloop#1\else\ddef{#1}\expandafter\ddefloop\fi}

% \bfA, \bfB, ...
\def\ddef#1{\expandafter\def\csname bf#1\endcsname{\ensuremath{\mathbf{#1}}}}
\ddefloop ABCDEFGHIJKLMNOPQRSTUVWXYZabcdefghijklmnopqrstuvwxyz\ddefloop

% \bfalpha, \bfbeta, ...,  \bfGamma, \bfDelta, ...,
\def\ddef#1{\expandafter\def\csname bf#1\endcsname{\ensuremath{\pmb{\csname #1\endcsname}}}}
\ddefloop {alpha}{beta}{gamma}{delta}{epsilon}{varepsilon}{zeta}{eta}{theta}{vartheta}{iota}{kappa}{lambda}{mu}{nu}{xi}{pi}{varpi}{rho}{varrho}{sigma}{varsigma}{tau}{upsilon}{phi}{varphi}{chi}{psi}{omega}{Gamma}{Delta}{Theta}{Lambda}{Xi}{Pi}{Sigma}{varSigma}{Upsilon}{Phi}{Psi}{Omega}{ell}\ddefloop

% \bbA, \bbB, ...
\def\ddef#1{\expandafter\def\csname bb#1\endcsname{\ensuremath{\mathbb{#1}}}}
\ddefloop ABCDEFGHIJKLMNOPQRSTUVWXYZ\ddefloop

% \cA, \cB, ...
\def\ddef#1{\expandafter\def\csname c#1\endcsname{\ensuremath{\mathcal{#1}}}}
\ddefloop ABCDEFGHIJKLMNOPQRSTUVWXYZ\ddefloop

% \vA, \vB, ..., \va, \vb, ...
\def\ddef#1{\expandafter\def\csname v#1\endcsname{\ensuremath{\boldsymbol{#1}}}}
\ddefloop ABCDEFGHIJKLMNOPQRSTUVWXYZabcdefghijklmnopqrstuvwxyz\ddefloop

% \valpha, \vbeta, ...,  \vGamma, \vDelta, ...,
\def\ddef#1{\expandafter\def\csname v#1\endcsname{\ensuremath{\boldsymbol{\csname #1\endcsname}}}}
\ddefloop {alpha}{beta}{gamma}{delta}{epsilon}{varepsilon}{zeta}{eta}{theta}{vartheta}{iota}{kappa}{lambda}{mu}{nu}{xi}{pi}{varpi}{rho}{varrho}{sigma}{varsigma}{tau}{upsilon}{phi}{varphi}{chi}{psi}{omega}{Gamma}{Delta}{Theta}{Lambda}{Xi}{Pi}{Sigma}{varSigma}{Upsilon}{Phi}{Psi}{Omega}{ell}\ddefloop

\newcommand\Sig{\ensuremath{\varSigma}}
\newcommand\vSig{\ensuremath{\vvarSigma}}
\newcommand\veps{\ensuremath{\varepsilon}}
\newcommand\vveps{\ensuremath{\vvarepsilon}}
\newcommand\eps{\ensuremath{\epsilon}}

\renewcommand\t{{\ensuremath{\scriptscriptstyle{\top}}}}

\DeclareMathOperator{\tr}{tr}
\DeclareMathOperator{\diag}{diag}
\DeclareMathOperator{\Diag}{Diag}
\DeclareMathOperator{\rank}{rank}
\DeclareMathOperator{\sign}{sign}
\DeclareMathOperator{\supp}{supp}
\DeclareMathOperator{\vol}{vol}

\DeclareMathOperator{\E}{\mathbb{E}}
\renewcommand{\P}{\ensuremath{\mathbb{P}}}
\DeclareMathOperator{\var}{var}
\DeclareMathOperator{\stddev}{stddev}
\DeclareMathOperator{\cov}{cov}

\DeclareMathOperator{\Bd}{bd}
\DeclareMathOperator{\Cl}{cl}
\DeclareMathOperator{\Conv}{conv}
\DeclareMathOperator{\Dom}{dom}
\DeclareMathOperator{\Epi}{epi}
\DeclareMathOperator{\Int}{int}
\DeclareMathOperator{\Null}{ker}
\DeclareMathOperator{\Span}{span}
\DeclareMathOperator{\Range}{ran}
\DeclareMathOperator{\Diam}{diam}
\DeclareMathOperator{\dist}{dist}
\DeclareMathOperator{\cost}{cost}
\DeclareMathOperator{\mean}{mean}
\DeclareMathOperator{\nnz}{nnz}

\DeclareMathOperator*{\argmin}{arg\,min}
\DeclareMathOperator*{\argmax}{arg\,max}

\newcommand\wt{\ensuremath{\widetilde}}
\newcommand\wh{\ensuremath{\widehat}}
\newcommand\mbf{\ensuremath{\mathbf}}
\renewcommand\v{\ensuremath{\boldsymbol}}

\newcommand\comment[1]{{\color{blue}\{\emph{Comment}:} #1{\color{blue}\}}}

\newtheorem{lemma}{Lemma}
\newtheorem{claim}{Claim}
\newtheorem{fact}{Fact}
\newtheorem{proposition}{Proposition}
\newtheorem{theorem}{Theorem}
\newtheorem{corollary}{Corollary}
\theoremstyle{remark}
\newtheorem{remark}{Remark}
\theoremstyle{definition}
\newtheorem{definition}{Definition}
\newtheorem{condition}{Condition}

\newcommand\parens[1]{(#1)}
\renewcommand\norm[1]{\|#1\|} % already defined in commath
\newcommand\braces[1]{\{#1\}}
\newcommand\brackets[1]{[#1]}
\newcommand\ceil[1]{\lceil#1\rceil}
\renewcommand\abs[1]{|#1|} % already defined in commath
\newcommand\ind[1]{\ensuremath{\mathds{1}\{#1\}}}
\newcommand\dotp[1]{\langle #1 \rangle}

\newcommand\Parens[1]{\mleft(#1\mright)}
\newcommand\Norm[1]{\mleft\|#1\mright\|}
\newcommand\Braces[1]{\mleft\{#1\mright\}}
\newcommand\Brackets[1]{\mleft[#1\mright]}
\newcommand\Ceil[1]{\mleft\lceil#1\mright\rceil}
\newcommand\Abs[1]{\mleft|#1\mright|}
\newcommand\Ind[1]{\mathds{1}\mleft\{#1\mright\}}
\newcommand\Dotp[1]{\mleft\langle#1\mright\rangle}

\begin{document}

\begin{theorem}
  There exists absolute constants $C, C', C'' > 0$ such that the
  following holds.
  Let $X_1, X_2, \dotsc, X_n$ be a sequence of real-valued random
  variables, where $\cF_i$ is the $\sigma$-field generated by $X_1, X_2,
  \dotsc, X_i$.
  Assume $\E(X_i|\cF_{i-1}) = 0$ and $\abs{X_i} \leq 1$ almost surely.
  Define $V := \sum_{i=1}^n \E(X_i^2|\cF_{i-1})$ and $S := \sum_{i=1}^n
  X_i$.
  For any $t>0$,
  \[
    \P\Parens{
      |S| >
      C t +
      C' \sqrt{
        V \Parens{
          \ln\Parens{
            \ln^2\max\Braces{ C'' \,,\, \frac{V}{|S|}}
          } + t
        }
      }
    }
    \ \leq \ 3e^{-t}
    \,.
  \]
\end{theorem}
\begin{proof}
  Define the events $E_A := \braces{ |S| \leq c_1 t + c_2 V }$ and
  $E_B := \braces{ V \leq c_3 t }$.
  It is easy to show that for any fixed $\lambda \in \intcc{-1,1}$,
  \begin{equation}
    \E\exp(\lambda S - \lambda^2 (e-2) V) \ \leq \ 1
    \,.
    \label{eq:freedman}
  \end{equation}
  See, for example, Beygelzimer et al (2011; arXiv:1002.4058).
  Therefore, $\P(\neg E_A) \leq 2e^{-t}$ for suitable $c_1, c_2 > 0$.
  Next, observe that on the event $E_A \wedge E_B$, we have
  \[
    |S|
    \ \leq \
    (c_1 + c_2c_3) t \leq c_4 t
  \]
  for a suitable value of $c_4 > 0$; while on the event $E_A \wedge
  \neg E_B$, we have
  \[
    |S|
    \ \leq \
    \Parens{
      \frac{c_1}{c_3} + c_2
    } V
    \ \leq \
    \frac{c_7V}{(1+\delta)e^2}
    \qquad\text{and}\qquad
    \frac{c_7V}{|S|(1-\delta)} \ \geq \ e^3
  \]
  for a suitable choice of $c_7>0$ and $\delta := (e-1)/(e+1) \approx
  0.46212 \in (0,1)$.
  Therefore,
  \begin{multline*}
    \P\Parens{
      E_A
      \ \wedge \
      |S| >
      c_4 t +
      c_5 \sqrt{
        V \Parens{
          \ln\Parens{
            \ln^2\max\Braces{ e^3 \,,\, \frac{c_6V}{|S|} }
          } + t
        }
      }
    }
    \\
    \ \leq \
    \P\Parens{
      E_A \ \wedge \
      \neg E_B \ \wedge \
      |S| >
      c_5 \sqrt{
        V \Parens{
          \ln\Parens{
            \ln^2\frac{c_6V}{|S|}
          } + t
        }
      }
    }
    \,.
  \end{multline*}
  To bound this latter probability, we first monotonically transform
  both sides of the inequality in the probability statement using the
  exponential function:
  \begin{multline*}
    \P\Parens{
      E_A \wedge \neg E_B \wedge
      \frac{
        \exp\Parens{\frac{S^2}{c_5^2V}}
      }{
        \ln^2\frac{c_6V}{|S|}
      }
      > e^t
    }
    \ \leq \
    \E\Parens{
      \ind{E_A \wedge \neg E_B}
      \frac{
        \exp\Parens{\frac{S^2}{c_5^2V}}
      }{
        \ln^2\frac{c_6V}{|S|}
      }
    }
    e^{-t}
    \\
    \ \leq \
    \int
    \E\Parens{
      \ind{E_A \wedge \neg E_B}
      \exp\Parens{ \lambda S - \lambda^2 (e-2) V }
    }
    f(\lambda) \dif\lambda
    e^{-t}
    \ \leq \
    e^{-t}
    \,.
  \end{multline*}
  Above, $f(t) := \ind{t \in \intco{-1/e^2,0} \cup \intoc{0,1/e^2}}
  1/(\abs{t}\ln^2(1/\abs{t}))$ can be interpreted as a probability
  density function.
  The first inequality is by Markov's inequality; the second
  inequality is proved below; and the third inequality above follows
  from~\eqref{eq:freedman}.
  Combining this with $\P(\neg E_A) \leq 2e^{-t}$ proves the theorem.

  Let $\Lambda$ be a random variable with density $f$.
  We need to prove
  \[
    \E\Parens{
      \ind{E_A \wedge \neg E_B}
      \frac{
        \exp\Parens{\frac{(1-\delta^2)S^2}{2c_7V}}
      }{
        \ln^2\Parens{\frac{c_7V}{|S|(1-\delta)}}
      }
    }
    \ \leq \
    \E\Parens{
      \exp\Parens{ \Lambda S - \Lambda^2 (e-2) V }
    }
    \,.
  \]
  We bound the conditional expectation $\E\parens{ \exp\parens{
  \Lambda S - \Lambda^2 (e-2) V } \mid S,V}$ on the event $E_A \wedge
  \neg E_B$:
  \begin{align*}
    \lefteqn{
      \E\Parens{
        \exp\Parens{ \Lambda S - \Lambda^2 (e-2) V }
        \mid S,V
      }
    } \\
    & \ = \
    \int_{-1/e^2}^{1/e^2}
    \exp\Parens{ S t - (e-2) V t^2 }
    f(t) \dif{t}
    \\
    & \ = \
    \exp\Parens{\frac{S^2}{2c_7V}}
    \int_{-1/e^2}^{1/e^2}
    \exp\Parens{
      -\frac{S^2}{2c_7V}
      + S t - (e-2) V t^2
    }
    f(t) \dif{t}
    \\
    & \ \geq \
    \exp\Parens{\frac{S^2}{2c_7V}}
    \int_{-1/e^2}^{1/e^2}
    \exp\Parens{
      -\frac{c_7V}{2}
      \Parens{ t - \frac{S}{c_7V} }^2
    }
    f(t) \dif{t}
    \qquad \text{(for $c_7 \geq 2(e-2)$)}
    \\
    & \ \geq \
    \exp\Parens{\frac{S^2}{2c_7V}}
    \int_0^{1/e^2}
    \exp\Parens{
      -\frac{c_7V}{2}
      \Parens{ t - \frac{|S|}{c_7V} }^2
    }
    f(t) \dif{t}
    \qquad \text{(since $f(t) = f(-t)$)}
    \\
    & \ \geq \
    \exp\Parens{\frac{S^2}{2c_7V}}
    \int_{\frac{|S|(1-\delta)}{c_7V}}^{\frac{|S|(1+\delta)}{c_7V}}
    \exp\Parens{
      -\frac{c_7V}{2}
      \Parens{ t - \frac{|S|}{c_7V} }^2
    }
    f(t) \dif{t}
    \qquad \text{(since $|S|(1+\delta)/(c_7V) \leq 1/e^2)$}
    \\
    & \ \geq \
    \exp\Parens{\frac{(1-\delta)S^2}{2c_7V}}
    \int_{\frac{|S|(1-\delta)}{c_7V}}^{\frac{|S|(1+\delta)}{c_7V}}
    f(t) \dif{t}
    \\
    & \ = \
    \exp\Parens{\frac{(1-\delta^2)S^2}{2c_7V}}
    \left.
    \Parens{
      \frac{1}{\ln(1/t)}
    }
    \right\vert_{\frac{|S|(1-\delta)}{c_7V}}^{\frac{|S|(1+\delta)}{c_7V}}
    \\
    & \ = \
    \exp\Parens{\frac{(1-\delta^2)S^2}{2c_7V}}
    \frac{\ln\frac{1+\delta}{1-\delta}}
    {\ln\Parens{\frac{c_7V}{|S|(1-\delta)}}\ln\Parens{\frac{c_7V}{|S|(1+\delta)}}}
    \\
    & \ \geq \
    \exp\Parens{\frac{(1-\delta^2)S^2}{2c_7V}}
    \frac{\ln\frac{1+\delta}{1-\delta}}
    {\ln^2\Parens{\frac{c_7V}{|S|(1-\delta)}}}
    \\
    & \ = \
    \exp\Parens{\frac{(1-\delta^2)S^2}{2c_7V}}
    \frac{1}{\ln^2\Parens{\frac{c_7V}{|S|(1-\delta)}}}
    \,.
    \qedhere
  \end{align*}
\end{proof}

\end{document}

