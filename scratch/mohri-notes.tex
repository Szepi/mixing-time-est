\documentclass[11pt]{article}
\usepackage{todonotes}
\newcommand{\todoc}[2][]{\todo[size=\scriptsize,color=green!10!white,#1]{Cs: #2}} % Csaba's comments
\newcommand{\todot}[2][]{\todo[size=\scriptsize,inline,color=blue!20!white,#1]{D: #2}} % Daniel's comments
\newcommand{\todor}[2][]{\todo[size=\scriptsize,color=orange!20!white,#1]{A: #2}} % Aryeh's comments

\usepackage{amsmath,amsbsy,amsfonts,amssymb,amsthm,color,dsfont}

%%%%%%%%%%%%%%%%%%%%%%%%%%%%%%%%
% PACKAGES
%%%%%%%%%%%%%%%%%%%%%%%%%%%%%%%%
\usepackage{latexsym}
\usepackage{fullpage}
\usepackage{amsmath}
\usepackage{amssymb}
\usepackage{mathtools}
\usepackage{enumerate}
\usepackage{accents}
\usepackage{tikz}
\usepackage{pgfplots}
\usepackage[ruled]{algorithm}
\usepackage{algpseudocode}
\usepackage{dsfont}
\usepackage[bf]{caption}
\usepackage{hyperref}
\hypersetup{
    bookmarks=true,         % show bookmarks bar?
    unicode=false,          % non-Latin characters in AcrobatÕs bookmarks
    pdftoolbar=true,        % show AcrobatÕs toolbar?
    pdfmenubar=true,        % show AcrobatÕs menu?
    pdffitwindow=false,     % window fit to page when opened
    pdfstartview={FitH},    % fits the width of the page to the window
    pdftitle={My title},    % title
    pdfauthor={Author},     % author
    pdfsubject={Subject},   % subject of the document
    pdfcreator={Creator},   % creator of the document
    pdfproducer={Producer}, % producer of the document
    pdfkeywords={keyword1} {key2} {key3}, % list of keywords
    pdfnewwindow=true,      % links in new window
    colorlinks=true,       % false: boxed links; true: colored links
    linkcolor=red,          % color of internal links (change box color with linkbordercolor)
    citecolor=blue,        % color of links to bibliography
    filecolor=magenta,      % color of file links
    urlcolor=cyan           % color of external links
}
\usepackage{amsthm}
\usepackage{times}
\usepackage{natbib}
\usepackage{nicefrac}
\usepackage{wrapfig}
\usepackage[capitalize]{cleveref}



%%%%%%%%%%%%%%%%%%%%%%%%%%%%%%%%
% MACROS
%%%%%%%%%%%%%%%%%%%%%%%%%%%%%%%%
\newcommand{\defined}{\vcentcolon =}
\newcommand{\rdefined}{=\vcentcolon}
%\newcommand{\E}{\mathbb E}
\newcommand{\Var}{\operatorname{Var}}
\newcommand{\calF}{\mathcal F}
\newcommand{\calR}{\mathcal R}
\newcommand{\sr}[1]{\stackrel{#1}}
\newcommand{\set}[1]{\left\{#1\right\}}
\newcommand{\ind}[1]{\mathds{1}\!\!\set{#1}}
\newcommand{\argmax}{\operatornamewithlimits{arg\,max}}
\newcommand{\argmin}{\operatornamewithlimits{arg\,min}}
\newcommand{\floor}[1]{\left \lfloor {#1} \right\rfloor}
\newcommand{\ceil}[1]{\left \lceil {#1} \right\rceil}

\def\subsubsect#1{\vspace{1ex plus 0.5ex minus 0.5ex}\noindent{\bf\boldmath{#1.}}}

\renewcommand{\P}[1]{\mathbb{P}\left\{#1\right\}}
\newcommand{\Prob}[1]{\mathbb{P}\left\{#1\right\}}
\newcommand{\EE}[1]{\mathbb{E}\left[#1\right]}

\let\temp\epsilon
\let\epsilon\varepsilon
\newcommand{\eps}{\varepsilon}


%%%%%%%%%%%%%%%%%%%%%%%%%%%%%%%%
% THEOREMS
%%%%%%%%%%%%%%%%%%%%%%%%%%%%%%%%
\theoremstyle{plain}
\newtheorem{theorem}{Theorem}
\newtheorem{proposition}[theorem]{Proposition}
\newtheorem{lemma}[theorem]{Lemma}
\newtheorem{corollary}[theorem]{Corollary}
\theoremstyle{definition}
\newtheorem{definition}[theorem]{Definition}
\newtheorem{assumption}[theorem]{Assumption}
\newtheorem{remark}[theorem]{Remark}
\newtheorem{example}[theorem]{Example}
\theoremstyle{remark}

\newcommand{\R}{\mathds{R}}

\newcommand\wh{\ensuremath{\widehat}}
\newcommand\wt{\ensuremath{\widetilde}}
\newcommand\norm[1]{\left\| #1 \right\|}
\newcommand\tvnorm[1]{\left\| #1 \right\|_{\mathrm{TV}}}
\newcommand\parens[1]{\left( #1 \right)}
\DeclareMathOperator{\Diag}{Diag}
\DeclareMathOperator{\Sym}{Sym}
\DeclareMathOperator*{\E}{\text{\bf E}}
\renewcommand\t{{\ensuremath{\scriptscriptstyle{\top}}}}
\newcommand\be{\ensuremath{\mathbf{e}}}
\newcommand\tmix{\ensuremath{t_{\mathrm{mix}}}}
\newcommand\htmix{\ensuremath{\hat{t}_{\mathrm{mix}}}}
\newcommand{\od}{\bar{d}}
\newcommand{\tcouple}{\tau_{\mathrm{couple}}}
\newcommand{\ZZ}{\mathcal{Z}}
\newcommand\trel{\ensuremath{t_{\mathrm{rel}}}}
\newcommand{\ip}[1]{\langle#1\rangle}
\newcommand{\DF}{\mathcal{E}}
\newcommand\ttmix{\ensuremath{t_{\mathrm{mix},2}}}

\newcommand{\paren}[1]{\left( #1 \right)}
\newcommand{\sqprn}[1]{\left[ #1 \right]}
\newcommand{\tlprn}[1]{\left\{ #1 \right\}}
%\newcommand{\set}[1]{\left\{ #1 \right\}}
\newcommand{\oo}[1]{\frac{1}{#1}}
\newcommand{\nrm}[1]{\left\Vert #1 \right\Vert}
\newcommand{\abs}[1]{\left| #1 \right|}
\newcommand{\trn}{^\intercal} %operator transpose

\newcommand{\beq}{\begin{eqnarray*}}
\newcommand{\eeq}{\end{eqnarray*}}
\newcommand{\beqn}{\begin{eqnarray}}
\newcommand{\eeqn}{\end{eqnarray}}


\def\ddefloop#1{\ifx\ddefloop#1\else\ddef{#1}\expandafter\ddefloop\fi}
% \bbA, \bbB, ...
\def\ddef#1{\expandafter\def\csname bb#1\endcsname{\ensuremath{\mathbb{#1}}}}
\ddefloop ABCDEFGHIJKLMNOPQRSTUVWXYZ\ddefloop

\title{Notes on the Mohri-Rostamizadeh approach}
%\author{Daniel, Aryeh, Csaba}


\begin{document}
\maketitle


%\begin{abstract}
%An abstract
%\end{abstract}

Let $T$ be the true transition matrix and $\hat T$ the estimated one
(leaving the estimation method presently unspecified). 
Recall that for any matrix $A$, its Euclidean norm $\nrm{A}_2$ is given by
$$ \nrm{A}_2 = \sup_{\nrm{x}\le1}\nrm{Ax} 
= \sup_{\nrm{x},\nrm{y}\le1}x\trn A y,
$$
where all vector norms are $\ell_2$ unless stated otherwise.

We are interested
in bounding
$$ \nrm{T-\hat T}_2 = \sup_{\nrm{x},\nrm{y}\le1} {x\trn(T-\hat T)y} .
$$

Trying to stay notationally close to \cite{mohri-rosta08},
let $H=B\times B$,
where $B$ is the Euclidean unit ball, and put
$R(x,y)=x\trn T y$, $\hat R_S(x,y)=x\trn \hat T y$.
Then
$$ \nrm{T-\hat T}_2 = \sup_{(x,y)\in H} R(x,y)-\hat R_S(x,y)=:\Phi(S),$$
where $S$ is the observed sample sequence.

We will need an analogue of Prop. 1 in \cite{mohri-rosta08}:
$$R(x,y) = \E_{S}[\hat R_S(x,y)].$$
I believe that for $S$ sampled from a stationary process
and $\hat T_{ij}$ taken to be the sample mean, this indeed holds.
\todoc{I don't think so. There is a bias because maybe $i$ is never visited.}
I don't think it holds for the median-of-means estimate.

[Note: there appears to be something wrong with the indexing
of the blocks in (6)-(7) of \cite{mohri-rosta08}. 
The numbering in 
\cite{MR1258867}, p.101 makes more sense, and we will follow it.]
Partition the sequence $S=(z_1,\ldots,z_n)$,
$n=2a\mu$, into $2a\mu$ blocks of length $a$.
Let $S_0$ be the sequence of odd blocks
and $\tilde S_0$ a sequence of independent blocks of length
$a$, where each block in $\tilde S_0$ is distributed identically to its
corresponding block $S_0$.

Since $\abs{\Phi(\cdot)}\le2$, Lemma 4.1 of
\cite{MR1258867} implies
$$ \abs{ \E_{S_0}[\Phi]-\E_{\tilde S_0}[\Phi]}\le 2(\mu-1)\beta(a),$$
where $\beta(\cdot)$ is the beta-mixing coefficient.

By Lemma 1 of \cite{mohri-rosta08}, we have,
for all $a,\mu,\eps>0$ with $2\mu a=n$ 
%and 
%$\eps>\E_{\tilde S_0}[\Phi(\tilde S_0)]$,
that
$$ \P{\Phi(S)>\eps}
\le 2\P{ \Phi(\tilde S_0)
%-\E_{\tilde S_0}[\Phi(\tilde S_0}]
>\eps
}
+
2(\mu-1)\beta(a).
$$

Proposition 2 ibid. yields, for all
$\eps>\E_{\tilde S_0}[\Phi(\tilde S_0)]$,


$$ \P{\Phi(S)>\eps} \le
2\exp(-\mu{\eps'}^2/2)+2(\mu-1)\beta(a),
$$
where $\eps'=\eps-\E_{\tilde S_0}[\Phi(\tilde S_0)]$.


Now invoke Lemma 2 ibid. to bound the expectation over iid blocks:
$$ \E_{\tilde S_0}[\Phi(\tilde S_0)] \le \calR_\mu(H),$$
where $\cal_\mu(\cdot)$ is the ordinary (iid) Rademacher complexity of
$H$.
Now it is well-known \cite{MR1965359} that the $\ell_2$
$t$-covering number of the $d$-dimensional Euclidean unit ball is
bounded by $(3/t)^d$. 
To bound the $\ell_2$ covering number of $H=B_d\times B_d$,
fix a radius $t$ and
%recall that
%$$ 
%%\frac{\sqrt{a}+\sqrt{b}}{\sqrt 2}
%%\le 
%\sqrt{a+b}\le \sqrt a+\sqrt b$$
%for all $a,b>0$.
%Hence
observe that
\beq
\nrm{(x,y)}_2 &=& \sqrt{ \sum_i x_i^2 + \sum_j y_j^2} \\
&\le& \sqrt2 t.
\eeq
Hence,
$$ B_d\times B_d \subset \sqrt2 B_{2d} $$
and $H$
can be covered by
$(\sqrt 6/t)^{2d}$ balls of radius $t$.
Using Dudley's chaining integral to estimate
the Rademacher complexity, we have

\beq
\calR_\mu(H) &\le& \frac{12}{\sqrt\mu}\int_0^{\sqrt2}\sqrt{d\ln(\sqrt 6/t)}dt \\
&\le& 25\sqrt{d/\mu}.
\eeq
Note that $d$ is the number of states, consistent with our other notes.

Adapting Theorem 1 in \cite{mohri-rosta08} to our needs, we have
that for any sample $S$ of length $n=2\mu a$
and $\delta>2(\mu-1)\beta(a)$,
the following
holds with probability at least $1-\delta$:
\beq
%\Phi(S) 
\nrm{T-\hat T}_2
\le 25\sqrt{\frac{d}{\mu}} + 2\sqrt{\frac{\ln(2/\delta')}{2\mu}},
\eeq
where $\delta'=\delta-2(\mu-1)\beta(a)$.

Recall that our other calculation (``beta-mixing'') gave
$$ \beta(a) \le \oo{\pi_*}e^{-a(1-\lambda_2)}.$$

\bibliographystyle{plainnat}
\bibliography{all}

\end{document}

