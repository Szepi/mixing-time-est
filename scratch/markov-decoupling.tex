\documentclass[11pt]{article}
\usepackage{todonotes}
\newcommand{\todoc}[2][]{\todo[size=\scriptsize,color=green!10!white,#1]{Cs: #2}} % Csaba's comments
\newcommand{\todot}[2][]{\todo[size=\scriptsize,inline,color=blue!20!white,#1]{D: #2}} % Daniel's comments
\newcommand{\todor}[2][]{\todo[size=\scriptsize,color=orange!20!white,#1]{A: #2}} % Aryeh's comments

\usepackage{amsmath,amsbsy,amsfonts,amssymb,amsthm,color,dsfont}

%%%%%%%%%%%%%%%%%%%%%%%%%%%%%%%%
% PACKAGES
%%%%%%%%%%%%%%%%%%%%%%%%%%%%%%%%
\usepackage{latexsym}
\usepackage{fullpage}
\usepackage{amsmath}
\usepackage{amssymb}
\usepackage{mathtools}
\usepackage{enumerate}
\usepackage{accents}
\usepackage{tikz}
\usepackage{pgfplots}
\usepackage[ruled]{algorithm}
\usepackage{algpseudocode}
\usepackage{dsfont}
\usepackage[bf]{caption}
\usepackage{hyperref}
\hypersetup{
    bookmarks=true,         % show bookmarks bar?
    unicode=false,          % non-Latin characters in AcrobatÕs bookmarks
    pdftoolbar=true,        % show AcrobatÕs toolbar?
    pdfmenubar=true,        % show AcrobatÕs menu?
    pdffitwindow=false,     % window fit to page when opened
    pdfstartview={FitH},    % fits the width of the page to the window
    pdftitle={My title},    % title
    pdfauthor={Author},     % author
    pdfsubject={Subject},   % subject of the document
    pdfcreator={Creator},   % creator of the document
    pdfproducer={Producer}, % producer of the document
    pdfkeywords={keyword1} {key2} {key3}, % list of keywords
    pdfnewwindow=true,      % links in new window
    colorlinks=true,       % false: boxed links; true: colored links
    linkcolor=red,          % color of internal links (change box color with linkbordercolor)
    citecolor=blue,        % color of links to bibliography
    filecolor=magenta,      % color of file links
    urlcolor=cyan           % color of external links
}
\usepackage{amsthm}
\usepackage{times}
\usepackage{natbib}
\usepackage{nicefrac}
\usepackage{wrapfig}
\usepackage[capitalize]{cleveref}



%%%%%%%%%%%%%%%%%%%%%%%%%%%%%%%%
% MACROS
%%%%%%%%%%%%%%%%%%%%%%%%%%%%%%%%
\newcommand{\defined}{\vcentcolon =}
\newcommand{\rdefined}{=\vcentcolon}
\newcommand{\E}{\mathbb E}
\newcommand{\Var}{\operatorname{Var}}
\newcommand{\calF}{\mathcal F}
\newcommand{\sr}[1]{\stackrel{#1}}
\newcommand{\set}[1]{\left\{#1\right\}}
\newcommand{\ind}[1]{\mathds{1}\!\!\set{#1}}
\newcommand{\argmax}{\operatornamewithlimits{arg\,max}}
\newcommand{\argmin}{\operatornamewithlimits{arg\,min}}
\newcommand{\floor}[1]{\left \lfloor {#1} \right\rfloor}
\newcommand{\ceil}[1]{\left \lceil {#1} \right\rceil}

\def\subsubsect#1{\vspace{1ex plus 0.5ex minus 0.5ex}\noindent{\bf\boldmath{#1.}}}

\renewcommand{\P}[1]{\mathbb{P}\left\{#1\right\}}
\newcommand{\Prob}[1]{\mathbb{P}\left\{#1\right\}}
\newcommand{\EE}[1]{\mathbb{E}\left[#1\right]}

\let\temp\epsilon
\let\epsilon\varepsilon
\newcommand{\eps}{\varepsilon}


%%%%%%%%%%%%%%%%%%%%%%%%%%%%%%%%
% THEOREMS
%%%%%%%%%%%%%%%%%%%%%%%%%%%%%%%%
\theoremstyle{plain}
\newtheorem{theorem}{Theorem}
\newtheorem{proposition}[theorem]{Proposition}
\newtheorem{lemma}[theorem]{Lemma}
\newtheorem{corollary}[theorem]{Corollary}
\theoremstyle{definition}
\newtheorem{definition}[theorem]{Definition}
\newtheorem{assumption}[theorem]{Assumption}
\newtheorem{remark}[theorem]{Remark}
\newtheorem{example}[theorem]{Example}
\theoremstyle{remark}

\newcommand{\R}{\mathds{R}}

\newcommand\wh{\ensuremath{\widehat}}
\newcommand\wt{\ensuremath{\widetilde}}
\newcommand\norm[1]{\left\| #1 \right\|}
\newcommand\tvnorm[1]{\left\| #1 \right\|_{\mathrm{TV}}}
\newcommand\parens[1]{\left( #1 \right)}
\DeclareMathOperator{\Diag}{Diag}
\DeclareMathOperator{\Sym}{Sym}
\renewcommand\t{{\ensuremath{\scriptscriptstyle{\top}}}}
\newcommand\be{\ensuremath{\mathbf{e}}}
\newcommand\tmix{\ensuremath{t_{\mathrm{mix}}}}
\newcommand\htmix{\ensuremath{\hat{t}_{\mathrm{mix}}}}
\newcommand{\od}{\bar{d}}
\newcommand{\tcouple}{\tau_{\mathrm{couple}}}
\newcommand{\ZZ}{\mathcal{Z}}
\newcommand\trel{\ensuremath{t_{\mathrm{rel}}}}
\newcommand{\ip}[1]{\langle#1\rangle}
\newcommand{\DF}{\mathcal{E}}
\newcommand\ttmix{\ensuremath{t_{\mathrm{mix},2}}}

\newcommand{\paren}[1]{\left( #1 \right)}
\newcommand{\sqprn}[1]{\left[ #1 \right]}
\newcommand{\tlprn}[1]{\left\{ #1 \right\}}
%\newcommand{\set}[1]{\left\{ #1 \right\}}
\newcommand{\oo}[1]{\frac{1}{#1}}

\newcommand{\N}{\mathbb{N}}
\newcommand{\nrm}[1]{\left\Vert #1 \right\Vert}
\newcommand{\TV}[1]{\nrm{#1}_{\textrm{{\tiny \textup{TV}}}}}

\def\ddefloop#1{\ifx\ddefloop#1\else\ddef{#1}\expandafter\ddefloop\fi}
% \bbA, \bbB, ...
\def\ddef#1{\expandafter\def\csname bb#1\endcsname{\ensuremath{\mathbb{#1}}}}
\ddefloop ABCDEFGHIJKLMNOPQRSTUVWXYZ\ddefloop

\title{Statistical Estimation of Mixing Times}
%\author{Daniel, Aryeh, Csaba}


\begin{document}
\maketitle




\section{Markov decoupling}

The following convention will be used for discontiguous indices. For 
$I
=\set{i_1,i_2,\ldots,i_{|I|}}
\subset\N
$ and
any sequence $u=u_1,u_2,\ldots$,
we write
$$
(u_i)_{i\in I}:=(u_{i_1},u_{i_2},\ldots,u_{i_{|I|}}).
$$
Also, for any random variable, $\cal L(\cdot)$ will denote its
law (distribution).


Let $X=(X_1,X_2,\ldots)$ be a 
stationary
$(G,\theta)$-geometrically ergodic
(in the sense of \cite{meroi2012-jap})
Markov chain.
Define
$\tilde X=(\tilde X_1,\tilde X_2,\ldots)$ be a 
the iid sequence
distributed according to 
the stationary distribution of $X$.


{\bf Theorem.}
For $I=\set{s,2s,\ldots,ts}$ for some $s,t\in\N$, we have
$$
\TV{{\cal L}(X_{i\in I}) - {\cal L}(\tilde X_{i\in I})}
\le 
2(t-1)G\theta^{s-1}
.
$$





\bibliographystyle{plainnat}
\bibliography{all}

\end{document}




and
$\tilde Y=\tilde Y_1,\tilde Y_2,\ldots$ 
be the iid sequence
of $\N$-valued random variables
distributed according to 
the stationary distribution of $Y$.
% (given by
%$\rho=B\pi$).
Define 
the index set
$I=\set{s,2s,\ldots,ts}$ for some $s,t\in\N$.
Then
\beq
\TV{\calL(Y_{i\in I}) - \calL(\tilde Y_{i\in I})}
\le 
2(t-1)G\tha^{s-1}
%2G\tha^{s-1}
.
\eeq
\enthn




The following convention will be used for discontiguous indices. For 
$I
=\set{i_1,i_2,\ldots,i_{|I|}}
\subset\N
$ and
$u\in\N^I$, the coordinates of $u$ are indexed as 
\beqn
\label{eq:index}
u=(u_i)_{i\in I}=(u_{i_1},u_{i_2},\ldots,u_{i_{|I|}})
\eeqn
and {\bf not} as $u=(u_i)_{1\le i\le|I|}=(u_{1},u_{2},\ldots,u_{|I|})$.

\bethn
\label{thm:hmm-decoup}
%Let 
%$(\pi,A,B)$ be a stationary $(G,\tha)$-geometrically ergodic Markov kernel
%with stationary distribution $\pi$ and $B$ an emission probability matrix.
Let $Y=Y_1,Y_2,\ldots$ be a 
stationary
$(G,\tha)$-geometrically ergodic
hidden Markov chain 
%induced by $(\pi,A,B)$,
and
$\tilde Y=\tilde Y_1,\tilde Y_2,\ldots$ be the iid sequence
of $\N$-valued random variables
distributed according to 
the stationary distribution of $Y$.
% (given by
%$\rho=B\pi$).
Define 
the index set
$I=\set{s,2s,\ldots,ts}$ for some $s,t\in\N$.
Then
\beq
\TV{\calL(Y_{i\in I}) - \calL(\tilde Y_{i\in I})}
\le 
2(t-1)G\tha^{s-1}
%2G\tha^{s-1}
.
\eeq
\enthn
%{\em Remark}: 
%Note that the bound is {\em dimension-free} in that it does not 
%depend on $T=|I|$.

The proof of Theorem \ref{thm:hmm-decoup} will proceed via several lemmata. 
First, we reduce
HMM decoupling to the simpler Markov decoupling:
\belen
\label{lem:hmm<markov-decoup}
Let $Y$
be the Markov chain induced by $(\pi,A,B)$
and
$X$ be the Markov chain underlying $Y$.
Define
$\tilde X$ and $\tilde Y$ to be the iid $\N$-valued processes distributed according to 
$\pi$ and
$\rho=B\pi$, respectively,
and
%
%, $\tilde Y$, $A$, $B$, and $\pi$ be as in the statement of 
%Theorem \ref{thm:hmm-decoup} and let $X=X_1,X_2,\ldots$ be the 
%Markov chain underlying $Y$.
%Define $\tilde X=\tilde X_1,\tilde X_2,\ldots$ be an 
%iid sequence of $\N$-valued random variables
%distributed according to $\pi$ and
choose any
$I\subset\set{1,\ldots,n}$, $n\in\N$.
Then
\beq
\TV{\calL(Y_{i\in I}) - \calL(\tilde Y_{i\in I})}
\le
\TV{\calL(X_{i\in I}) - \calL(\tilde X_{i\in I})} 
.
\eeq
\enlen
\bepf
Let $Q$ and $\tilde Q$ be the probability measures 
induced 
on $\N^{ts}$
by $X$ and ${\tilde X}$, respectively,
and let
$P$ and $\tilde P$ 
be the probability measures induced by
 $Y$ and ${\tilde Y}$.
Put $J=\set{1,\ldots,n}\setminus I$.

For $w\in\N^I$ and $z\in\N^J$, define $x[w,z]\in\N^{I\cup J}$
to be such that $x_i=w_i$ for $i\in I$ and $x_j=z_j$ for $j\in J$ 
(recall our indexing convention (\ref{eq:index})). 
For $u\in\N^I,v\in\N^J$, we define $y[u,v]\in\N^{I\cup J}$ 
analogously.
Then
\beq
\TV{\calL(Y_{i\in I}) - \calL(\tilde Y_{i\in I})}
&=& 
\oo2\sum_{u\in\N^I}\abs{ \sum_{v\in\N^J} \paren{P(y[u,v])-\tilde P(y[u,v])}} \\
\hide{
&=&
\oo2\sum_{u\in\N^I}\abs{ 
\sum_{v\in\N^J }
\sum_{x\in\N^{I\cup J}}
\prod_{i\in I}B_{u_i,x_i} \prod_{j\in J}B_{v_j,x_j}
\paren{
Q(x)
-
\tilde Q(x)
}
} 
\\
}
&=&
\oo2\sum_{u\in\N^I}\abs{ 
\sum_{v\in\N^J }
%\sum_{x\in\N^{I\cup J}
\sum_{w\in\N^{I}}
\sum_{z\in\N^{J}}
\prod_{i\in I}B_{u_i,w_i} \prod_{j\in J}B_{v_j,z_j}
\paren{
Q(x[w,z])
-
\tilde Q(x[w,z])
}
}
\\
&=&
\oo2\sum_{u\in\N^I}\abs{ 
\sum_{w\in\N^{I}}
\sum_{z\in\N^{J}}
\prod_{i\in I}B_{u_i,w_i}
\paren{
Q(x[w,z])
-
\tilde Q(x[w,z])
}
}
\\
&\le&
\oo2\sum_{u\in\N^I}
\sum_{w\in\N^{I}}
\prod_{i\in I}B_{u_i,w_i}
\abs{ 
\sum_{z\in\N^{J}}
\paren{
Q(x[w,z])
-
\tilde Q(x[w,z])
}
}
\\
&=&
\oo2
\sum_{w\in\N^{I}}
\abs{ 
\sum_{z\in\N^{J}}
\paren{
Q(x[w,z])
-
\tilde Q(x[w,z])
}
}\\
&=&
\TV{\calL(X_{i\in I}) - \calL(\tilde X_{i\in I})} 
.
\eeq
\enpf

Next, we 
establish a Markov decoupling result, which may be of independent interest.
The result relies on the following lemma, whose proof was sketched in
\cite[Lemma 2.2.5, Remark 2.2.7]{kont07-thesis}, and which we give in the Appendix for completeness.
\belen
\label{lem:2kernels}
Consider two at most countable sets $\calX,\calY$.
Let 
$p_0$ 
be
a distribution on $\calX$, and
$p_1(\cdot\gn x)$, $x\in\calX$, a (conditional probability) 
%transition
kernel from $\calX$ to $\calY$,
%We will 
and
write 
$
\mu=
p_0\tp p_1
%(\cdot\gn x)
$ for the distribution on
$\calX\times\calY$ defined by
\beq
\mu(x,y) &=& p_0(x)p_1(y\gn x),
\qquad x,y\in\calX\times\calY.
\eeq
Similarly, let $q_0$ be a distribution on $\calX$ and $q_1$ a kernel from $\calX$
to $\calY$; define
$\nu=q_0\tp q_1
%(\cdot\gn x)
$. 
Then
\beqn
\label{eq:tvtensker}
\TV{\mu-\nu} &\leq &
d_0+d_1-d_0d_1,
\eeqn
where $d_0=\TV{p_0-q_0}$ and
\beq
d_1 &=& \sup_{x\in\calX}\TV{p_1(\cdot\gn x)-q_1(\cdot\gn x)}.
\eeq
\enlen

%prove a 
%dimension-free
%Markov contraction inequality:
\belen
\label{lem:markov-decoup}
Let $X=(X_1,\ldots,X_n)$
be a 
stationary
Markov chain 
%Let $A$ be 
%Markov kernel 
with 
stationary distribution $\pi$
and
contraction coefficient $\kappa$,
%induced by $(\xi,A)$
and
$\tilde X=(\tilde X_1,\ldots,\tilde X_n)$ 
be the iid process with distribution $\pi$.
Then
\beq
\TV{\calL(X)-\calL(\tilde X)}\le(n-1)\kappa
.
\eeq
\enlen
\bepf
We will prove the claim by recursive applications of 
(\ref{eq:tvtensker}). First, take 
$\calX=\N$, $\calY=\N^{n-1}$,
$p_0=\calL(X_1)$, $p_1=\calL(\sseq X2n\gn X_1)$,
$q_0=\pi$, $q_1=\pi^{n-1}$.
Clearly, $d_0=0$, and hence
\beq
\TV{\calL(X)-\calL(\tilde X)}\le \sup_{x_1\in\calX}
\TV{\calL(\sseq X2n\gn X_1=x_1) - \pi^{n-1}}.
\eeq
For any fixed $x_1\in\calX$, we bound 
$\TV{\calL(\sseq X2n\gn X_1=x_1) - \pi^{n-1}}$ 
again via (\ref{eq:tvtensker}),
this time taking $\calX=\N$, $\calY=\N^{n-2}$,
$p_0=\calL(X_2\gn X_1=x_1)$, $p_1=\calL(\sseq X3n\gn X_2)$,
$q_0=\pi$, $q_1=\pi^{n-2}$. Now
$d_0=\TV{ \calL(X_2\gn X_1=x_1)-\pi}\le\kappa$, and so 
(\ref{eq:tvtensker}) implies
\beq
\TV{\calL(\sseq X2n\gn X_1=x_1) - \pi^{n-1}}
\le \kappa + \sup_{x_2\in\calX}
\TV{\calL(\sseq X3n\gn X_2=x_2) - \pi^{n-2}}.
\eeq
Continuing this process yields the claim.
%ALSO WRONG
\hide{
\beq
\TV{\calL(X)-\calL(X')}
&=&
\oo2
\sum_{x\in\N^n}
\abs{
\paren{\xi_{x_1}-\xi'_{x_1}}
A_{x_2,x_1}\ldots A_{x_n,x_{n-1}}
}\\
&=&
\oo2
\sum_{x\in\N^n}
A_{x_2,x_1}\ldots A_{x_n,x_{n-1}}
\abs{
{\xi_{x_1}-\xi'_{x_1}}
}\\
&=&
\oo2
\sum_{x_1\in\N}
\abs{
{\xi_{x_1}-\xi'_{x_1}}
}=\TV{\xi-\xi'}
\hide{
%WRONG, keeping for posterity
&\le&
\oo2
\sum_{x_2^n\in\N^{n-1}}
A_{x_2,x_2}\ldots A_{x_n,x_{n-1}}
\abs{
\sum_{x_1\in\N}
A_{x_2,x_1}
\paren{
\xi_{x_1}-\xi'_{x_1}
}
}\\
&=&
\oo2
\sum_{x_2\in\N }
\abs{
\sum_{x_1\in\N }
A_{x_2,x_1}
\paren{
\xi_{x_1}
-
\xi'_{x_1}
}
}\\
&=&
\TV{A(\xi-\psi)} \le \kappa\TV{\xi-\psi}
}
\eeq
where the last inequality follows from Lemma \ref{lem:markov-contraction}.
}
\enpf
{\em Remark.} A more careful application of (\ref{eq:tvtensker}),
together with \cite[Lemma 12]{mark-conc}, yields the sharper estimate
\beq
\TV{\calL(X)-\calL(\tilde X)}\le 1-(1-\kappa)^{n-1}.
\eeq

We proceed with the proof of the HMM decoupling theorem:
\bepf[Proof of Theorem \ref{thm:hmm-decoup}]
Let $X$ be the Markov chain underlying $Y$
induced by $(\xi,A)$ for some stochastic vector $\xi$,
and let $\tilde X$ be its 
corresponding iid version, distributed according to $\pi$,
the stationary distribution of $A$.
Then Lemma \ref{lem:hmm<markov-decoup} implies that, for any $I\subset\N$,
\beq
\TV{\calL(Y_{i\in I}) - \calL(\tilde Y_{i\in I})}
\le
\TV{\calL(X_{i\in I}) - \calL(\tilde X_{i\in I})} .
\eeq
Now for $I=\set{s,2s,\ldots,ts}$, we have
\beq
\calL(X_{i\in I}) = \calL(Z_1,\ldots,Z_t),
\eeq
where $Z=\seq Z1t$ is the Markov chain induced by $(\xi,A^{s-1})$.
Now by Corollay \ref{cor:ergo-contr}, $A^{s-1}$ has contraction coefficient
$\kappa\le2G\tha^{s-1}$, and so Lemma \ref{lem:markov-decoup} implies
\beq
\TV{\calL(X_{i\in I}) - \calL(\tilde X_{i\in I})}
&=&
\TV{\calL\seq Z1t - \calL\seq{\tilde X}{1}{t}} \\
&\le& 
2(t-1)G\tha^{s-1}
%\kappa\TV{\xi-\pi}\le
%2G\tha^{s-1}
.
\eeq
\enpf



\bibliographystyle{plainnat}
\bibliography{all}

\end{document}

