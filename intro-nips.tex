%!TEX root =  matrix-est.tex

%Performance guarantees play a key role in the analysis and design of machine learning algorithms.

This work
tackles the challenge of constructing
fully empirical bounds on the
mixing time of Markov chains based on a single sample path.
Let $(X_t)_{t=1,2,\dotsc}$ be an irreducible, aperiodic
time-homogeneous Markov chain on a finite state space $[d] :=
\{1,2,\dotsc,d\}$ with transition matrix $\vP$.
Under this assumption, the chain converges to its unique stationary
distribution $\vpi =
(\pi_i)_{i=1}^d$ regardless of the initial state distribution $\vq$: 
\[
  \lim_{t\to\infty} {\Pr}_{\vq}\Parens{X_t = i}
  = \lim_{t\to\infty} (\vq \vP^t)_i = \pi_i
  \quad \text{for each $i \in [d]$} .
\]
The \emph{mixing time} $\tmix$ of the Markov chain is the
number of time steps required
%time that it
%takes
for the chain to
be within a fixed threshold of
%(approximately) reach
its stationary
distribution:
\todoa{slightly changed wording}
\begin{align}
\label{eq:mixtimedef}
  \tmix
  :=
  \min\Braces{
    t \in \bbN :
    \sup_{\vq}
    \max_{A\subset [d]}
    \Abs{
      \textstyle\Pr_{\vq}\Parens{ X_t \in A } - \vpi(A)
    }
%    \Norm{
%      \vq \vP^t - \vpi
%    }_{\tv} 
    \leq 1/4
  }\,.
\end{align}
Here, $\vpi(A) = \sum_{i\in A} \pi_i$ is the probability assigned to
set $A$ by $\vpi$, and the supremum is over all possible initial
distributions $\vq$.
The problem studied in this work is the construction of a non-trivial
confidence interval $C_n = C_n(X_1,X_2,\dotsc,X_n,\delta) \subset
[0,\infty]$, based only on the observed sample path
$(X_1,X_2,\dotsc,X_n)$ and $\delta \in (0,1)$, that
succeeds with probability $1-\delta$
in trapping
the value of the
mixing time $\tmix$.
\todoa{changed wording}

This problem is motivated by the numerous scientific applications and
machine learning tasks in which
the quantity of interest is
%one is interested in estimating
the
mean $\vpi(f) = \sum_i \pi_i f(i)$
for some function $f$ of the states of
a Markov chain.
\todoa{changed wording} \todoc{I am not sure that the new wording is better. It
fails to express that there is an estimation problem. "Quantity of interest"
is way to vague for me.} 
This is the setting of the celebrated Markov Chain Monte Carlo (MCMC)
paradigm~\cite{liu2001monte}, but the problem also arises in
performance prediction involving time-correlated data, as is common in
reinforcement learning~\cite{sutton98}.
Observable bounds on mixing times are useful in the design and
diagnostics of these methods; they yield effective approaches to
assessing the estimation quality, even when \emph{a priori} knowledge
of the mixing time or correlation structure is unavailable.

\paragraph{Main results.}
We develop the first procedure for constructing non-trivial
and fully empirical confidence intervals for Markov mixing time.
Consider a reversible ergodic Markov chain on $d$ states with absolute
spectral gap $\gap$ and stationary distribution minorized by $\pimin$. 
As is well-known \citep[Theorems~12.3 and~12.4]{LePeWi08},
\begin{equation}
  \label{eq:mixing-time-bound}
  \Parens{\frac1{\gap} - 1} \ln2
  \ \leq \ \tmix
  \ \leq \ \frac1{\gap} \ln \frac4{\pimin}
  ,
\end{equation}
%$(\frac{1}\gap  - 1)\ln(2)  \le \tmix \le \frac{1}\gap \ln \frac4\pimin$,
and hence it suffices to estimate $\gap$ and $\pimin$. 
%these quantities $\gap$ and $\pimin$ are well-known to govern
%mixing time.
Our main results are summarized as follows.
\begin{enumerate}
  \item
    In \cref{sec:rates-upper}, we give a point-estimator for $\gap$,
    and prove in \cref{thm:err} that it achieves multiplicative
    accuracy from \emph{a single sample path} of length
    $\tilde{O}\parens{1/(\pimin\gap^3)}$.\footnote{The
    $\tilde{O}(\cdot)$ notation suppresses logarithmic factors.}
    We also provide a point-estimator for $\pimin$ that requires a sample
    path of length $\tilde{O}\parens{1/(\pimin\gap)}$.
    This establishes the feasibility of \emph{estimating} the mixing
    time in this setting.
    However, the valid confidence intervals suggested by
    \cref{thm:err} depend on the unknown quantities $\pimin$ and
    $\gap$.
    We also discuss the importance of reversibility, and some possible
    extensions to non-reversible chains.

  \item
    In \cref{sec:rates-lower}, we show that $n = \Omega\parens{ (d\log
    d)/\gap + 1/\pimin}$ observations are necessary for any procedure
    to guarantee constant multiplicative accuracy in estimating $\gap$
    (\cref{thm:lb-pimin,thm:lb-gap}).
    Essentially, \emph{every} state may need to be visited about
    $\log(d)/\gap$ times, on average, before an accurate estimate of
    the mixing time can be provided, regardless of the estimation
    procedure used.

  \item
    In \cref{sec:empirical}, we develop a procedure for computing
    \emph{valid empirical confidence intervals} for $\pimin$ and
    $\gap$ (\cref{alg:empest}).
    The procedure uses just a single sample path and does not require
    any prior knowledge of the chain's parameters, such as $\pimin$ or
    $\gap$.
    We quantify the rate at which the width of the confidence
    intervals shrink as a function of the length of the sample path
    (\cref{thm:empirical}).
    We also explain how to combine the empirical confidence intervals
    from \cref{alg:empest} with the non-empirical bounds from
    \cref{thm:err} to produce valid empirical confidence intervals.
    We prove in \cref{thm:combined} that the width of these new
    intervals converge to zero asymptotically at least as fast as
    those from either \cref{thm:err} and \cref{thm:empirical}.

%    which is
%    $     O\Parens{
%      \frac1{\pimin}\sqrt{\frac{d\log(d/\delta)}{ n}}
%      +
%      \frac{d}{\pimin\gap} \sqrt{\frac{\log(d/\delta)}{\pimin n}}
%    }.$


    
%    \todoc{please add rate}

\end{enumerate}

\paragraph{Related work.}
There is a vast statistical literature on estimation in Markov chains.
For instance, it is known that under the assumptions on
$(X_t)_t$ from above, the law of large numbers guarantees that
the sample mean $\vpi_n(f) \defeq \frac1n \sum_{t=1}^n f(X_t)$
converges almost surely to $\vpi(f)$~\cite{meyn1993markov}, while the
central limit theorem tells us that as $n\to \infty$, the distribution
of the deviation $\sqrt{n}( \vpi_n(f)-\vpi(f))$ will be normal with
mean zero and asymptotic variance $\lim_{n\to\infty}
n\Var\Parens{\vpi_n(f)}$~\cite{kipnis1986central}.

Although these asymptotic results help us understand the limiting
behavior of the sample mean over a Markov chain, they say little about
the finite-time non-asymptotic behavior, which is often needed for the
prudent evaluation of a method or even its algorithmic design
\cite{
MCMCDiscussion93%
,DBLP:conf/valuetools/KontoyiannisLM06%
,BBL06%
,MniSzeAu08%
,MauPo09%
,LiLiWaSt11:KWIK%
,flegal2011implementing%
,Gyori-paulin15%
,SwaJoa15:LoggedBandit%
}. %BaHaVa10:activelearning,ShaVo08:conformalpred
To address this need, numerous works have developed Chernoff-type
bounds on $\Pr\parens{ |\vpi_n(f)-\vpi(f)| > \eps }$, thus providing
valuable tools for non-asymptotic probabilistic
analysis~\cite{gillman1998chernoff,leon2004optimal,DBLP:conf/valuetools/KontoyiannisLM06,
%meroi2012-jap,
paulin15}.
These probability bounds are larger than corresponding bounds for
independent and identically distributed (iid) data due to the temporal
dependence; intuitively, for the Markov chain to yield a fresh draw
$X_{t'}$ that behaves as if it was independent of $X_t$, one must wait
$\Theta(\tmix)$ time steps.
Note that the bounds generally depend on distribution-specific
properties of the Markov chain (e.g., $\vP$, $\tmix$, $\gap$), which are often
unknown \emph{a priori} in practice.
Consequently, much effort has been put towards estimating these
unknown quantities, especially in the context of MCMC diagnostics, in
order to provide data-dependent assessments of estimation
accuracy~\cite[e.g.,][]{MCMCDiscussion93,GaSmi00:eigval,jones2001,flegal2011implementing,1209.0703,Gyori-paulin15}.
However, these approaches generally only provide asymptotic
guarantees, and hence fall short of our goal of empirical bounds that
are valid with any finite-length sample path.

Learning with dependent data is another main motivation to our work.
Many results from statistical learning and empirical process theory
have been extended to sufficiently fast mixing, dependent
data~\citep[e.g.,][]{Yu94,MR1921877,gamarnik03,MoRo08,MoRo09,DBLP:conf/nips/SteinwartC09,Steinwart2009175},
providing learnability assurances (e.g., generalization error bounds).
These results are often given in terms of mixing coefficients, which
can be consistently estimated in some cases~\citet{McDoShaSche11}.
However, the convergence rates of the estimates
from~\citet{McDoShaSche11}, which are needed to derive confidence
bounds, are given in terms of unknown mixing coefficients.
When the data comes from a Markov chain, these mixing coefficients can
often be bounded in terms of mixing times, and hence our main results
provide a way to make them fully empirical, at least in the limited setting we study.

It is possible to eliminate many of the difficulties presented above
%with additional access to the Markov chain.
when allowed more flexible access
to the Markov chain.
For example, given a sampling oracle that generates
independent transitions from any given state (akin to a ``reset''
device), the mixing time becomes an efficiently testable property in
the sense studied in~\citet{BaFoRuSmiWhi00,BaFoRuSmiWhi13}.
On the other hand, when one only has a circuit-based description of
the transition probabilities of a Markov chain over an
exponentially-large state space, there are complexity-theoretic
barriers for many MCMC diagnostic problems~\citet{BhaBoMo11}.
%%Performance guarantees play a key role in the analysis and design of machine learning algorithms.
%
%In this paper we study the construction of fully empirical bounds on the mixing time of Markov chains based on a single sample path.
%Let $(X_t)_{t=1,2,\dotsc}$ be an irreducible, aperiodic time-homogeneous Markov chain on a finite state space $[d]$ of $d$ states numbered by positive integers with transition matrix $\vP$. Under this assumption, the chain converges to its unique stationary \todoc{Maybe steady state is better?}
%distribution $\vpi = (\pi_i)_{i=1}^d$ regardless of the distribution $\vq$ of $X_1$: 
%$\lim_{t\to\infty} \Pr_{\vq}\Parens{X_t = i} = \lim_{t\to\infty} (\vq \vP^t)_i = \pi_i$, $1\le i \le d$.
%The mixing time $\tmix$ of the Markov chain is the time that it takes for the chain to reach its stationary distribution, say,
%\[
%  \tmix
%  :=
%  \min\Braces{
%    t \in \bbN :
%    \sup_{\vq}
%    \max_{A\subset [d]} |\textstyle\Pr_{\vq}\Parens{ X_t \in A } - \vpi(A)|
%%    \Norm{
%%      \vq \vP^t - \vpi
%%    }_{\tv} 
%    \leq 1/4
%  }\,.
%\]
%Here $\vpi(A) = \sum_{i\in A} \pi_i$ is the total probability assigned
%to set $A$ by $\vpi$, and the supremum is over
%all possible initial distributions $\vq$.
%The problem studied in the paper is the construction of a confidence
%set $C_n = C_n(X_1,X_2,\dotsc,X_n,\delta) \subset [0,\infty]$ based on
%the knowledge of $X_1,X_2,\dotsc,X_n$ only that captures the mixing
%time $\tmix$ with a pre-specified probability $1-\delta \in (0,1)$. 
%In the remainder of the introduction, we describe our motivation of studying this problem, followed by a 
%brief description of our main results.
%\todod{``might be interesting'' sounds weak.} \todoc{better now?}
%
%In numerous scientific applications and some machine learning tasks
%one is interested in estimating the mean $\vpi(f) = \sum_i \pi_i f(i)$ of some function $f$ of the states of a Markov chain. 
%For example, this is the realms of the celebrated Markov Chain Monte Carlo (MCMC) paradigm, but 
%the problem also arises in performance prediction when working with time correlated data, as is common in,
%for example, in reinforcement learning.
%%as 
%% (cf. \cref{sec:mot}) 
%Under our assumptions,
%the law of large numbers guarantees that the sample mean $\vpi_n(f) \defeq \frac1n \sum_{t=1}^n f(X_t)$ 
%converges almost surely to $\vpi(f)$, while the central limit theorem tells us that as $n\to \infty$, 
%the distribution of the deviation $\sqrt{n}( \vpi_n(f)-\vpi(f))$ will be normal with zero mean and asymptotic
%variance $\sigma^2_{\mathrm{as}}(f) \defeq \lim_{n\to\infty} n \Var\Parens{ \vpi_n(f) }$. \todoc{shall we add a citation here?}
%While these results help us to understand how the sample mean over a Markov chain behaves in the limit,
%they tell us little about the finite time, non-asymptotic behavior of the sample mean,
%which are needed for prudent evaluation of one's method or even for algorithm design 
%\cite{
%MCMCDiscussion93%
%,DBLP:conf/valuetools/KontoyiannisLM06%
%,BBL06%
%,MniSzeAu08%
%,MauPo09%
%,LiLiWaSt11:KWIK%
%,flegal2011implementing%
%,Gyori-paulin15%
%,SwaJoa15:LoggedBandit%
%}. %BaHaVa10:activelearning,ShaVo08:conformalpred
%To address this need, numerous works have been devoted to deriving
%Chernoff-type, exponential bounds for upper bounding the tail probabilities $p(r) = \Pr\Parens{ |\vpi_n(f)-\vpi(f)|> r }$,
%also known as ``concentration bounds'', e.g., 
%\cite{gillman1998chernoff,leon2004optimal,DBLP:conf/valuetools/KontoyiannisLM06,paulin15}.
%Compared with the case when $(X_t)_t$ is an independent, identically distributed (iid) sequence of
%random variables, these tail probabilities must be larger due to the temporal dependence of the sequence $(X_t)_t$.
%Intuitively, for the Markov chain to forget the state where it is at time $t$, or in other words to get a 
%``fresh'' sample $X_{t'}$ that is approximately ``independent'' from $X_t$, one must wait $\Theta(\tmix)$ time steps.
%Thus, in the bounds on $p(r)$, one expects to see that the sample size $n$ 
%gets appropriately reduced as $\tmix$ grows and this is indeed the case in bounds mentioned above.
%is replaced by $n/\tmix$, which one
%might think of as an approximate sample size. 
%Indeed, all the above mentioned works get bounds on $p(r)$ that depend on various functions of $\vP$.
%Namely, Kontoyiannis et al. obtain bounds that depend on certain information-theoretic quantities defined implicitly
%as a function of $\vP$ and $f$, which are after simplification is bounded based on quantities related to 
%a regeneration-type quantity derived from $\vP$
%\cite{DBLP:conf/valuetools/KontoyiannisLM06}, while the other works mentioned above, in the case of reversible chains,
%derive bounds that depend on the spectral gap of $\vP$.
%Note that the spectral gap for such chains is essentially inversely related to the mixing time of the chain (see \cref{sec:setting}).
%%\cite{gillman1998chernoff,leon2004optimal,,paulin15}.
%Much research have been devoted to finding appropriate \emph{a priori} bounds on the various quantities that appear 
%in these bounds, mostly in the theory community (e.g., \cite{saloff2004random}). 
%To address the lack or looseness of such a priori bounds, a significant effort also went into developing methods
%that estimate these quantities based on data, mostly in the MCMC community in an effort to develop
%effective methods that are able to diagnose whether a chain has reached its stationary state 
%(e.g., \cite{MCMCDiscussion93,flegal2011implementing,Gyori-paulin15}).
%While the typical results state the asymptotic consistency of these estimators, or a central limit theorem (which 
%are of little help to achieve a fully empirical, non-asymptotic bound on $p(r)$),
%Gyori and Paulin give a finite time error bound for these estimates. 
%Unsurprisingly, these error bounds themselves
%depend on other unknown quantities of the Markov chain (which can be bounded in terms of the mixing time, again)
%\cite{Gyori-paulin15}.
%
%This situation is similar to what appears when one attempts to use a Bernstein-type deviation bound to 
%control the deviation between the sample mean of a sequence of iid random variables from their common mean
%with a known range, but unknown variance: For a tight bound, one wishes to use the unknown variance, but in the deviation
%bound for the error of a variance estimate, the variance appears again.
%While in the case of Bernstein-bounds this circular reference leads to a nontrivial bound
%\cite{audibert2009}, it appears that the situation is different in the Markov setting.
%Focusing on the spectral gap, we will see that if one attempts to use a similar technique,
%the set of feasible values based on the two error bounds always will never rule out that the spectral gap is zero,
%or that the mixing time is infinite. Can one even hope to obtain 
%after finitely many samples a nontrivial upper bound on the mixing time?
%Obviously, for any finite sample size, if the mixing time of the chain is larger than the sample size than a sound method
%must return infinity as the upper bound on the mixing time. Hence, the question is whether one can design a method
%that never fails and if the sample size is appropriately large, returns a nontrivial bound on the mixing time, adapting the time it needs to trap the mixing time in a fixed size confidence interval to the unknown mixing time.
%
%\hide{
%\subsection{Motivation}
%\label{sec:mot}
%One key competence in statistical learning is the ability to predict one's performance on future data.
%To make performance prediction possible, one starts by posing some assumptions on how the data arises.
%Simpler assumptions lead to simpler algorithms and analysis, while they may also limit applicability of the
%obtained results and algorithms.
%The most common assumption employed postulates that all data points, including past and future data points,
%are sampled independently of each other from a common underlying distribution (e.g., \cite{SSBD14:UnderstandingML}).
%This is helpful on two accounts: 
%The lack of dependence between the data points means that sample averages of bounded functions of the data 
%will converge at a uniform rate (irrespective of the distribution), while the assumption that all data points share the same
%distribution means that one can learn about the distribution of future data points based on past data.
%}
%
%\hide{
%It is well-known that contrary to the idealized assumptions of the standard
%learning model, real data is almost never iid.
%Fortunately, many classical results of statistics and statistical
%learning theory extend to the dependent data in a more or less
%seamless fashion, at least when the data is sufficiently ``mixing''.
%Such results include the central limit theorem, 
%exponential tail bounds for empirical processes, 
%performance guarantees for complexity regularization
%using sieves or roughness penalties, as well as 
%results based on stability arguments or Rademacher complexity 
%\citep{Woo86,Dou94,Yu94,ZheChu96,dlPenaGine99,Mei00,
%  MR1921877,vid2003,gamarnik03,LoKuScha05,MoRo08,MoRo09,
%  DBLP:conf/nips/SteinwartC09,Steinwart2009175,
%  FaSze11,
%  london:nips12asalsn,london:icml13,
%  DBLP:conf/nips/ShaliziK13}.
%
%Unfortunately, a distribution-free theory of non-iid learning is simply
%impossible, when arbitrary dependence among the sample points is allowed.
%Hence, one must quantify this dependence via the various mixing coefficients,
%which appear in the non-iid generalization bounds mentioned above.
%A frequently raised concern is that now the bounds depend on {\em
%unknown} properties of the distribution, which precludes their
%actual use in algorithmic applications (e.g., active
%learning~\citep{BBL06,DHM07}).
%%, and any algorithm aiming at a fully empirical error bound must
%%also estimate the mixing coefficients.
%In addition to learning-theoretic generalization bounds, the need for
%fully empirical confidence bounds arises in Markov Chain Monte Carlo
%(MCMC) simulations---so as to halt the run when sufficiently close to
%the stationary distribution \citep{galin06}, and in Markov Reward
%Processes \citep{sutton98}, as this will affect the learner's policy.
%As \citet{Mei00} notes, in the absence of accurate estimates of the
%mixing time, such fully empirical bounds remain ``a theoretician's
%dream''.
%
%Our goal here is to develop an algorithm that, given a single sample
%path of a Markov chain, computes non-trivial bounds on the mixing time
%of the Markov chain.
%We remark on the inherent chicken-and-egg nature of the challenge:
%confidence intervals derived from a Markov chain---say, on the the
%chain's mixing time---are typically given in terms of the chain's
%mixing time.
%Our way out of this conundrum is to give {\em empirical} (i.e.,
%\emph{data-dependent}) as opposed to {\em a priori} guarantees.
%To illustrate the intuition, consider a Markov chain on two states,
%$\braces{0,1}$.
%A learner who is stuck in state $0$ for a long time learns very little
%about the chain and hence cannot be expected to give meaningful upper
%bounds on its mixing time.
%Consider, however, a learner who is lucky enough
%to be in a chain that switches frequently between the two states.
%Surely at some finite time, the learner will possess meaningful and nontrivial
%information about the chain's mixing time --- crucially, without any
%{\em prior} knowledge regarding this quantity. This simple example captures
%the nature of our results: in some cases, we will not be able to say
%anything meaningful (but the learner will recognize that this is the case),
%while in ``luckier'' situations, we will be able to provide nontrivial
%upper estimates on the mixing time.
%}
%
%\paragraph{Main results.}
%We propose what is apparently the first efficient and fully empirical
%estimator of Markov mixing time.
%\todod{``apparently'' sounds weak.}
%Consider a reversible ergodic Markov chain on $d$ states with spectral
%gap $\gap$ (which is well-known to govern mixing time)
%and stationary distribution minorized by $\pimin$.
%Our main results are summarized as follows.
%\begin{enumerate}
%  \item
%    In \cref{thm:err}, we give an estimator for $\gap$ that
%    achieves multiplicative accuracy from \emph{a single sample path}
%    of length $\tilde{O}\parens{1/(\pimin\gap^3)}$.\footnote{The
%    $\tilde{O}(\cdot)$ notation suppresses logarithmic factors.}
%    We also provide an estimator for $\pimin$ that requires a sample
%    path of length $\tilde{O}\parens{1/(\pimin\gap)}$.
%    However, the valid confidence intervals suggested by
%    \cref{thm:err} depend on the unknown quantities $\pimin$ and
%    $\gap$.
%
%  \item
%    In \cref{thm:lb-pimin,thm:lb-gap}, we show that $\Omega\parens{
%    (d\log d)/\gap + 1/\pimin }$ observations are necessary for
%    constant multiplicative accuracy in estimating $\gap$.
%
%  \item
%    In \cref{thm:empirical}, we develop a procedure for computing
%    \emph{valid empirical confidence intervals} for $\pimin$ and
%    $\gap$.
%    The procedure uses just a single sample path and does not require
%    any prior knowledge about $\pimin$ or $\gap$.
%    We also quantify the rate at which the width of the confidence
%    intervals shrink as a function of the length of the sample path.
%
%    We show in \cref{thm:combined} how to combine these empirical
%    confidence intervals with the non-empirical bounds from
%    \cref{thm:err} to produce valid empirical confidence intervals;
%    the width of these intervals converge to zero asymptotically as
%    fast as those from \cref{thm:err}.
%
%%    which is
%%    $     O\Parens{
%%      \frac1{\pimin}\sqrt{\frac{d\log(d/\delta)}{ n}}
%%      +
%%      \frac{d}{\pimin\gap} \sqrt{\frac{\log(d/\delta)}{\pimin n}}
%%    }.$
%
%
%    
%%    \todoc{please add rate}
%
%\end{enumerate}
%
%
%
%\paragraph{Related work.} \todoc{Move this later? It also needs to be updated, to 
%adjust it to the current intro.}
%Owing to the wide applicability of MCMC methods to various computational
%problems, much effort has been devoted to developing effective convergence
%diagnostic methods the aim of deriving stopping rules or stationarity
%diagnoses. While a large number of heuristic techniques have been developed
%\citep[e.g.,][]{MCMCDiscussion93,CoCa96}, it is widely recognized that all
%these methods may fail to meet the guarantees that they are designed for.
%Some recent advance in this area \citep{1209.0703,Gyori-paulin15} have
%yielded practical algorithms for estimating the rates of convergence
%of various MCMC algorithms, but they fall short of providing intervals
%that are valid with any finite length sequence.
%
%In the theory community, \citet{BaFoRuSmiWhi00,BaFoRuSmiWhi13}
%developed algorithms for testing whether a given Markov chain over a
%finite state space of size $d$ is rapidly mixing.
%In particular, assuming access to a sampling oracle that can generate
%independent transitions from any given state, their algorithm can
%distinguish between chains that in $t$ steps for a fixed $t$ reach a
%distribution close to the stationary distribution using $\tilde{O}(t
%d^{5/3})$ oracle calls.
%It is remarkable that when $t$ is small enough, the procedure requires
%sublinearly many samples in the total number of transitions in the
%Markov chain.
%For Markov chains with exponentially-large state spaces (with a
%circuit-based description of the transition probability structure),
%\citet{BhaBoMo11} establishes complexity-theoretic barriers for
%similar MCMC diagnostic problems.
%%A similar question was studied by \citet{BhaBoMo11} in the context of
%%Markov chains with very large state spaces.
%%They assume an circuit based description of the
%%transition probability structure of the matrix and derive several hardness
%%results for Markov chains whose state space consists of all binary strings of
%%a given length. For example,
%%they prove that given a Markov chain that mixes
%%rapidly, it is $\mathsf{SZK}$-hard to distinguish whether the chain is close to
%%stationarity by time $t$ or far from stationarity at time $ct$ for a
%%constant $c$ when started from a given initial state ($\mathsf{SZK}
%%\subset \mathsf{AM} \cap \mathsf{coAM}$, which is conjectured to
%%extend beyond $\mathsf{BPP}$ \citep{GoVa11}).\footnote{%
%%  $\mathsf{SZK}$ stands for statistical zero-knowledge, $\mathsf{BPP}$
%%  stands for probabilistic polynomial-time, $\mathsf{AM}$ stands for
%%  ``Arthur-Merlin protocol''.%
%%}
%
%Other works include that of \citet{McDoShaSche11} who propose to estimate
%the $\beta$-mixing coefficients of a stationary process possessing a density
%based on histogram estimates of the underlying joint density. Their main
%result is a consistency result for a fixed ``lag''. For fast mixing processes
%(e.g., for $k$-step Markov processes) their result yields a convergence rate,
%which unfortunately is given in terms of the unknown mixing coefficients.
%%Catch-22 problem remains: The rate of convergence of their estimates depends on the unknown mixing coefficients
%%(Theorem 2.4)
%% $\exp(W(\log(n)))$ is the window length, where $W$ is Lambert's $W$-function (i.e., this is between $\log \log(n)$ and $\log n$).
%The earlier study of \citet{Nobel06} considers the generalization of
%existing discernibility results for iid processes to weakly uniformly
%ergodic families, and shows how this can be used to study---and in
%some cases estimate---polynomial mixing rates for dependent processes.
%
%
