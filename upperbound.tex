
%!TEX root =  matrix-est.tex

In this section, we prove Theorem~\ref{thm:err}.
We start by proving the deviation bound on $\pimin-\hatpimin$.

%\subsection{Accuracy of singleton frequency estimates and bounding $\pimin-\hatpimin$}
%\label{sec:singleton}

%{\bf Aryeh: I'm going to do this without blocking}
%First, 
Theorem~3.8 of \citet{paulin15} implies
\[
\bbP_{\vpi}
\Braces{
\abs{ \hat\pi_i-\pi_i} > \eps
} \le
2\exp\Parens{-\frac{
    n\gap\eps^2
  }{
4\pi_i(1-\pi_i)+10\eps
}},
\qquad i\in[d],
\]
where $\bbP_{\vpi}$ denotes the probability with respect to the stationary chain.
Moreover, by Proposition~3.14 of \citet{paulin15}, we have
\begin{eqnarray*}
\bbP
\Braces{
\abs{ \hat\pi_i-\pi_i} > \eps
}
&\le&
\sqrt{
%\Parens{\sum_{i\in[d]}}
\frac1\pimin
\bbP_{\vpi}
\Braces{
\abs{ \hat\pi_i-\pi_i} > \eps
}} 
\le
\sqrt{\frac2\pimin}\exp\Parens{-\frac{
    n\gap\eps^2
  }{
8\pi_i(1-\pi_i)+20\eps
}}.
\end{eqnarray*}

% \ln\Parens{\frac{d}\delta\sqrt{\frac{2}{\pimin}}}

Hence, with probability $1-\delta$, we have for all $i\in [d]$,
%\todoc{Is it possible to have $\pi_i$ here instead of $\pimin$ in the RHS? Probably not..}
\begin{align}
\label{eq:singletonboundsimple}
\Abs{\hat{\pi}_i-\pi_i}
\le
\sqrt{ 8 \pi_i(1-\pi_i) 
\,\frac{
\ln\Parens{\frac{d}\delta\sqrt{\frac{2}{\pimin}}}
  }{\gap n}}
+
\frac{20
  \ln\Parens{\frac{d}\delta\sqrt{\frac{2}{\pimin}}}
}{\gap n}
%\frac1{(n-1)\gap\sqrt\pimin}
%+
%\sqrt{\frac2{\pimin\gap^2n}\ln(2d/\delta)}
\le
23 \sqrt{  \pi_i 
	\frac{\ln\Parens{\frac{d}\delta\sqrt{\frac{2}{\pimin}}}
           }{\gap n}
       }\,,
\end{align}
where the second inequality holds as soon as 
\[
n \ge
	\frac{\ln\Parens{\frac{d}\delta\sqrt{\frac{2}{\pimin}}}
           }{\gap \pimin}\,.
\]
Consider now the following lemma:
\begin{lemma}
Let  $\pi_i,\wh\pi_i$, $i\in [2]$, $C>0$ be
such that $|\pi_i - \wh\pi_i| \le C \sqrt{\pi_i}$, $i\in [2]$.
Assume that $\pi_1\le \pi_2$.
Then, if $C\le 2\sqrt{\pi_1}$ then $\Abs{\pi_1 - \min(\wh\pi_1,\wh\pi_2) } \le C \sqrt{\pi_1}$.
\end{lemma}
\begin{proof}
Let $\hatpimin = \min(\wh\pi_1,\wh\pi_2)$.
We have
\begin{align*}
\hatpimin \le \hat{\pi}_1 \le  \pi_1 + C \sqrt{\pi_1}
\end{align*}
so it remains to show that
\begin{align*}
\pi_1 - C \sqrt{\pi_1} \le \hatpimin\,.
\end{align*}
Note that $\hat\pi_2 \ge \pi_2- C \sqrt{\pi_2 } $. Now,
our condition on $C$ implies that $C \le 2 \sqrt{\pi_1} = \sqrt{\pi_1}+\sqrt{\pi_2} = \frac{\pi_2 - \pi_1}{\sqrt{\pi_2}-\sqrt{\pi_1}}$. 
\begin{align*}
\pi_2 - C \sqrt{\pi_2  } \ge \pi_1 - C \sqrt{\pi_1 }
\end{align*}
holds if and only if
%\begin{align*}
%\pi_i - \pi_1 \ge C (\sqrt{\pi_i} - \sqrt{\pi_1}) \sqrt{\eps}\,,
%\end{align*}
%which in turn is equivalent to 
\begin{align*}
\sqrt{\pi_2} + \sqrt{\pi_1} = \frac{\pi_2 - \pi_1}{\sqrt{\pi_2} -\sqrt{\pi_1}} \ge C \,,
\end{align*}
which is implied by our condition on $\eps$. Thus, under our condition $\hat\pi_2 \ge \pi_2 - C  \sqrt{\pi_2 } \ge \pi_1 - C \sqrt{\pi_1}$, finishing the proof.
\end{proof}
By this lemma, as soon as $n\ge (11.5)^2 \frac{\ln( \frac{d}{\delta} )+\frac12\ln(\frac2\pimin) }{ \gap\pimin}$,
then
\[
\Abs{\pimin - \hatpimin} \le 23 \sqrt{ \pimin \frac{\ln( \frac{d}{\delta} )+\frac12\ln(\frac2\pimin) }{ \gap n} }\,,
\]
For $n$ smaller than the above critical value, the second term in \eqref{eq:piminbound} is lower bounded by one.
Noting that $\Abs{\pimin-\hatpimin}\le 1$ thus shows that \eqref{eq:piminbound} holds for any $n$.


%%%%%%%%%%%%%%%%%%%%%%%%%%%%%%%%%%%%%%%%%%%%%%%%%%%%%%
%%%%%%%%%%%%%%%%%%%%%%%%%%%%%%%%%%%%%%%%%%%%%%%%%%%%%%
%%%%%%%%%%%%%%%%%%%%%%%%%%%%%%%%%%%%%%%%%%%%%%%%%%%%%%
%%%%%%%%%%%%%%%%%%%%%%%%%%%%%%%%%%%%%%%%%%%%%%%%%%%%%%

Let us now turn to proving \eqref{eq:gapbound}, i.e., 
the bound on the error of the spectral gap estimate $\hatgap$.
The accuracy of $\hatgap$ is based on the accuracy of $\Sym(\wh\vL)$
in approximating $\vL$ via Weyl's inequality:
\begin{align*}%\label{eq:Weyl}
  |\hat\lambda_i - \lambda_i|
  \leq \norm{\Sym(\wh\vL) - \vL}
  \quad \text{for all }i \in [d] .
\end{align*}
Moreover, the triangle inequality implies that symmetrizing $\wh\vL$
can only help:
\begin{align*}
%\label{eq:symm}
  \norm{\Sym(\wh\vL) - \vL} \leq \norm{\wh\vL - \vL} .
\end{align*}
Therefore, the bound on the spectral gap in \cref{thm:err} is readily obtained from the
following lemma which bounds $\norm{\wh\vL - \vL}$ as well as $\abs{
\hat\pi_i/\pi_i - 1 }$ for each $i \in [d]$ with high probability.
\begin{lemma}
  \label{lem:err}
  There exists an absolute constant $C>0$ such that the following
  holds.
  For any $\delta \in (0,1)$, with probability at least $1-\delta$,
  \begin{equation*}
    \norm{\wh\vL - \vL}
    \leq
    \frac{C}{\pimin}
      \sqrt{\frac{\ln(d/\delta)\ln\frac{n}{\delta\pimin}}{n\pimin\gap}} \,.
  \end{equation*}
	 \todoc{Note that the bound that was here essentially got multiplied by $1/\pimin$!}
\end{lemma}
The remainder of this section is devoted to proving this \lcnamecref{lem:err}.

The error $\wh\vL - \vL$ may be written as
\[
  \wh\vL - \vL
  = \errm + \errp \vL + \vL \errp + \errp \vL \errp
  + \errp \errm + \errm \errp + \errp \errm \errp\,,
\]
where
\begin{align*}
  \errp & := \Diag(\hat\vpi)^{-1/2} \Diag(\vpi)^{1/2} - \vI , \\
  \errm & := \Diag(\vpi)^{-1/2} \Parens{
    \wh\vM - \vM
  } \Diag(\vpi)^{-1/2} .
\end{align*}
Therefore
\begin{align*}%\label{eq:vlbound}
  \norm{\wh\vL - \vL}
  \leq \norm{\errm} +
  \Parens{
    \norm{\errm} + \norm{\vL}
  }
  \Parens{
    2\norm{\errp} + \norm{\errp}^2
  }
  .
\end{align*}
When  $\norm{\errm}\le 1$ and $\norm{\errp}\le 1$ both hold, the previous inequality implies that
\begin{align}\label{eq:vlsimple}
\norm{\wh\vL - \vL} \le C(\norm{\errm}+\norm{\errp})\,.
\end{align}
Here and in what follows we use $C>0$ to denote some universal constant, whose value does not depend on the problem parameters, but may change from line to line.
Since in addition to $\norm{\vL}\le 1$,  $\norm{\wh\vL}\le 1$ also holds (this again follows from the Perron-Frobenius theorem), in fact  \eqref{eq:vlsimple}  always holds as long as $C\ge 2$.

\subsection{A bound on $\norm{\errp}$}
We have
\begin{align*}
  \norm{\errp}
  = \max_{i\in[d]} \Abs{
    \sqrt{\frac{\pi_i}{\hat\pi_i}} - 1\,.
  }
\end{align*}
%The purpose of this section is to develop a bound on the RHS of \eqref{eq:singleton} so as to bound $\norm{\errp}$.
First note that for $x$ real, $\Abs{x-1}\le 1/2$ implies that $\max(\Abs{x-1},\Abs{1/x-1}) \le 2 \Abs{x-1}$.
Hence, it suffices to develop a bound on $ \Abs{ \frac{\hat\pi_i}{\pi_i} - 1 }$;
the final bound will follow under the condition that the developed bound is below $1/2$.
The required bound is readily obtained from~\eqref{eq:singletonboundsimple}:
As long as $n = \Omega \Parens{
	\frac{\ln\Parens{\frac{d}\delta\sqrt{\frac{2}{\pimin}}}
           }{\gap \pimin} }$, \todoc{First condition on $n$.}
$
\Abs{\frac{\hat{\pi}_i}{\pi_i}-1}
\le
23 \sqrt{  
	\frac{\ln\Parens{\frac{d}\delta\sqrt{\frac{2}{\pimin}}}
           }{\pimin\gap n}
       }\,,
$
and thus
\begin{align}
\label{eq:singletonbound}
\norm{\errp} \le C \sqrt{ \frac{\ln\Parens{\frac{d}\delta} + \ln \frac{1}{\pimin} }{\pimin\gap n}} \,.
\end{align}
We now turn to bounding  $\norm{\errm}$.
%Note that although the entries of $\wh{\vM}$ and $\hat\vpi$ are simply
%sample averages, they are formed using a \emph{single sample path}
%from a (possibly) non-stationary Markov chain.
%Applying Chernoff-Hoeffding bounds for Markov chains to each entry of
%$\wh{\vM}$ would result in a
%significantly worse sample complexity bound.
%%vastly worse sequence length requirement.
%Instead, we adapt probability tail bounds for sums of independent
%random matrices to our setting.


\subsection{Accuracy of doublet frequency estimates (bounding $\norm{\errm}$)}
\label{sec:pairwise}
In this section we prove a bound on $\norm{\errm}$.
For this, we decompose
$\errm = \Diag(\vpi)^{-1/2}(\wh\vM-\vM)\Diag(\vpi)^{-1/2} $
into $\EE{\errm}$ and 
$
\errm-\EE{\errm}% =  \Diag(\vpi)^{-1/2}(\wh\vM - \EE{\wh\vM}) \Diag(\vpi)^{-1/2} ,
$, the first measuring the effect of a non-stationary start of the chain,
while the second measuring the variation due to randomness.
\if0
into the sum of
\[
  \Diag(\vpi)^{-1/2}
  \left(\bbE( \wh\vM )- \vM\right)
  \Diag(\vpi)^{-1/2}
\]
and
\begin{multline*}
  \Diag(\vpi)^{-1/2} \left(\wh\vM- \bbE(\wh\vM)\right) \Diag(\vpi)^{-1/2}
  \\
  =
  \frac{n_H}{n-1}
  \Diag(\vpi)^{-1/2}\Parens{
    \wh{\vM}_H - \bbE(\wh{\vM}_H)
  }\Diag(\vpi)^{-1/2}
  \\
  + \frac{n_T}{n-1}
  \Diag(\vpi)^{-1/2}\Parens{
    \wh{\vM}_T - \bbE(\wh{\vM}_T)
  }\Diag(\vpi)^{-1/2}
  .
\end{multline*}
\fi
\subsubsection{Bounding $\norm{\EE{\errm}}$: The price of a non-stationary start.}
Let $\vpi^{(t)}$ be the distribution of states at time step $t$.
We will make use of the following proposition, which can be derived by following 
Proposition 1.12 of \citet{MoTe06}:
\begin{proposition}
For $t\ge 1$, let $\vUpsilon^{(t)}$ be the vector with $\Upsilon^{(t)}_i= \frac{\pi_i^{(t)}}{\pi_i}$ and let
$\norm{\cdot}_{2,\vpi}$ denote the $\vpi$-weighted $2$-norm
\begin{align}
\label{eq:vv}
  \norm{\vv}_{2,\vpi} := \Parens{ \sum_{i=1}^d \pi_i v_i^2 }^{1/2}
  .
\end{align}
%\todoc{Move up to notation?}
 Then,
\begin{equation}
  \label{eq:L2-contraction}
  \norm{ \vUpsilon^{(t)} - \v1 }_{2,\vpi} \leq
  \frac{(1-\gap)^{t-1}}{\sqrt{\pimin}} \,.
\end{equation}
\end{proposition}
%Corollary 1.15:
An immediate corollary of this result is that 
\begin{align}
\label{eq:dvpi-ratio-bound}
\Norm{  \Dvpit \Dvpi^{-1} - \vI  } \le \frac{(1-\gap)^{t-1}}{\sqrt{\pimin}}\,.
\end{align}
Now note that
\begin{align*}
\bbE (\wh\vM) = \frac{1}{n-1} \sum_{t=1}^{n-1} \Dvpit \vP\,
\end{align*}
and thus
\begin{align*}
\EE{\errm} &=
  \Dvpi^{-1/2}
  \left(\bbE( \wh\vM )- \vM\right)
  \Dvpi^{-1/2}\\
& =  
\frac{1}{n-1} \sum_{t=1}^n
  \Dvpi^{-1/2} (\Dvpit  - \Dvpi) \vP \Dvpi^{-1/2} \\
& =  
\frac{1}{n-1} \sum_{t=1}^n
  \Dvpi^{-1/2} (\Dvpit \Dvpi^{-1} - \vI) \vM \Dvpi^{-1/2} \\
& =  
\frac{1}{n-1} \sum_{t=1}^n
   (\Dvpit \Dvpi^{-1} - \vI) \vL \,.
\end{align*}
Combining this, $\norm{\vL}\le 1$ and~\eqref{eq:dvpi-ratio-bound}, we get
\begin{align}
\label{eq:biasbound}
\norm{\bbE(\errm)} \le \frac{1}{(n-1)\sqrt{\pimin}}\, \sum_{t=1}^n (1-\gap)^{t-1} \le  \frac{1}{(n-1)\gap\sqrt{\pimin}}\, .
\end{align}
\if0
There is a bias in the estimates that arises when the initial state
distribution is not the stationary distribution.
The first block $F$ is most subject to non-stationarity effects, while
the subsequent blocks $H_s$ and $T_s$ are much less affected as long
as the block lengths are long enough.
Fortunately, the impact of the first block is small when there are
enough blocks.
Choosing an appropriate block length will balance these effects.
(The block length also plays a role in the dependence between blocks,
but this is dealt with later.)

Define
\begin{align*}
  \vpi^{(H_s)} & := \frac1{|H_s|} \sum_{t \in H_s} \vpi^{(t)} ,
  & \vpi^{(T_s)} & := \frac1{|T_s|} \sum_{t \in T_s} \vpi^{(t)} , \\
  \vpi^{(H)} & := \sum_{s=1}^{\mu_H} w_{H,s} \vpi^{(H_s)} ,
  & \vpi^{(T)} & := \sum_{s=1}^{\mu_T} w_{T,s} \vpi^{(T_s)} ,
\end{align*}
and
\begin{align*}
  \vpi^{(H,T)} & :=
  \frac{n_H}{n_H + n_T} \vpi^{(H)}
  + \frac{n_T}{n_H + n_T} \vpi^{(T)}
  .
\end{align*}
Then
\begin{multline*}
  \Diag(\vpi)^{-1/2}
  \bbE\Parens{ \wh\vM \,\Big\vert\, \wh{\vM}_F }
  \Diag(\vpi)^{-1/2} - \vL
  \\
  =
  \frac{|F|}{n-1}\Parens{
    \Diag(\vpi)^{-1/2} \wh{\vM}_F \Diag(\vpi)^{-1/2} - \vL
  }
  \\
  + \Parens{ 1 - \frac{|F|}{n-1} }
  \Parens{
    \Diag(\vpi)^{-1/2} \Diag(\vpi^{(H,T)}) \Diag(\vpi)^{-1/2} - \vI
  } \vL
  .
\end{multline*}
Since $\norm{\wh{\vM}_F} \leq 1$ and $\norm{\vL} \leq 1$, it follows
that
\begin{multline*}
  \Norm{
    \Diag(\vpi)^{-1/2}
    \bbE\Parens{ \wh\vM \,\Big\vert\, \wh{\vM}_F }
    \Diag(\vpi)^{-1/2} - \vL
  }
  \\
  \leq
  \frac{|F|}{n-1}
  \Parens{ \frac1{\pimin} + 1 }
  +
  \Parens{ 1 - \frac{|F|}{n-1} }
  \max_{i \in [d]} \Abs{ \frac{\pi_i^{(H,T)}}{\pi_i} - 1 }
  .
\end{multline*}

For any block $H_s$,
\begin{align*}
  \Norm{ \vUpsilon^{(H_s)} - \v1 }_{2,\vpi}
  & =
  \Norm{
    \frac1{|H_s|} \sum_{t \in H_s} \vUpsilon^{(t)} - \v1
  }_{2,\vpi}
  \\
  & \leq
  \frac1{|H_s|} \sum_{t \in H_s}
  \Norm{
    \vUpsilon^{(t)} - \v1
  }_{2,\vpi}
  \\
  & \leq \frac1{|H_s|} \cdot \frac{(1-\gap)^{\min H_s -
  1}}{\sqrt{\pimin}\gap}
  .
\end{align*}

Therefore, for each $i \in [d]$,
\begin{align*}
  \Abs{ \frac{\pi_i^{(H_s)}}{\pi_i} - 1 }
  & \leq
  \frac1{\sqrt{\pi_i}}
  \Norm{ \vUpsilon^{(H_s)} - \v1 }_{2,\vpi}
  \\
  & \leq
  \frac1{|H_s|}
  \frac{(1-\gap)^{\min H_s - 1}}{\sqrt{\pi_i\pimin}\gap}
  .
\end{align*}
(The same, of course, applies to $T_s$ blocks.)
By the same argument, for all $i \in [d]$,
\begin{align*}
  \Abs{ \frac{\pi_i^{(H,T)}}{\pi_i} - 1 }
  & \leq \frac1{n-1} \cdot \frac{(1-\gap)^{\min H_1 -
  1}}{\sqrt{\pi_i\pimin}\gap}
  .
\end{align*}

Suppose the first block $F$ is of length
\begin{equation}
  \label{eq:block-length}
  |F| = a \geq 1 + \frac{1}{\gap} \ln \frac1{\pimin\gap}
  .
\end{equation}
Then
\[
  \frac{(1-\gap)^{\min H_s - 1}}{\pimin\gap}
  \leq 1
\]
for all blocks $H_s$ (and same for the $T_s$ blocks), so
\begin{align}
  \max \Braces{
    \Abs{ \frac{\pi_i^{(H_s)}}{\pi_i} } ,\,
    \Abs{ \frac{\pi_i^{(T_s)}}{\pi_i} } ,\,
    \Abs{ \frac{\pi_i^{(H,T)}}{\pi_i} }
    : i \in [d] , s
  }
  & \leq 2
  .
  \label{eq:near-stationary}
\end{align}
This means that as long as the blocks are sufficiently long, then all
but the first block $F$ can effectively be thought of as being started
from the stationary distribution.
We shall henceforth assume that~\eqref{eq:block-length} holds.
\fi
\subsubsection{
Bounding $\norm{\errm-\EE{\errm}}$: 
Application of a matrix tail inequality}
\label{sec:pairwise-tailbound}

In this section we analyze the deviations of $\errm-\EE{\errm}$. By the definition of $\errm$,
\begin{align}
\label{eq:errmdev1}
\norm{\errm-\EE{\errm}}
 =  \norm{\Diag(\vpi)^{-1/2}\Parens{\wh\vM - \bbE \wh\vM } \Diag(\vpi)^{-1/2}}
 \le \frac{1}{\pimin} \norm{\wh\vM - \bbE \wh\vM }
 \,.
\end{align}
The matrix $\wh\vM - \EE{\wh\vM}$  is defined 
as a sum of dependent centered random matrices.
We will use the blocking technique of  \citet{Bernstein27} 
to relate the likely deviations of this matrix to that of
a sum of independent centered random matrices.
The deviations of these will then bounded
with the help of a Bernstein-type matrix tail inequality due to \citet{tropp2015intro}.

%We divide $(X_t)_{t\in[n]}$ into $2\mu$ blocks each of length
%$a\geq2$.
%Define
%\begin{align*}
%  H_j & := \{ t \in \bbN : 2(j-1)a + 1 \leq t \leq (2j-1)a \}
%  ,
%  \\
%  T_j & := \{ t \in \bbN : (2j-1)a + 1 \leq t \leq 2ja \}
%  ,
%  \qquad j \in [\mu] .
%\end{align*}
%Now define the corresponding blocks of random variables:
%\begin{equation*}
%  X(H_j) := ( X_t : t \in H_j )
%  , \quad
%  X(T_j) := ( X_t : t \in T_j )
%  ,
%  \qquad j \in [\mu] .
%\end{equation*}
%Let $\Xi(H_j)$ for $j \in [\mu]$ be independent copies of $X(H_j)$ for
%$j \in [\mu]$, and similarly define $\Xi(T_j)$ for $j \in [\mu]$.
%Let the distribution of $X(H_j)$ be denoted by $\cP_j$, and let the
%distribution of $X(T_j)$ be denoted by $\cQ_j$.
%Let the distributions of $(X(H_j) : j \in [\mu])$ and $(\Xi(H_j) : j
%\in [\mu])$ be denoted by $\cP$ and $\tilde{\cP} = \prod_{j=1}^{\mu}
%\cP_j$, respectively.
%Similarly define $\cQ$ and $\tilde{\cQ} = \prod_{j=1}^{\mu} \cQ_j$.
%Then for any $\veps>0$,
%\begin{align*}
%\end{align*}

Divide $[n-1]$ into several blocks each of length $a \leq n/2$ (except \todoc{Needed $a\le n/2$}
possibly the final block):
\begin{align*}
  F & := [a] , \\
  H_s & := \{ t \in [n-1] : (2s-1)a + 1 \leq t \leq 2sa \} , \\
  T_s & := \{ t \in [n-1] : 2sa + 1 \leq t \leq (2s+1) a \} ,
\end{align*}
where $s=1,2,\dots$.
Let $\mu_H$ (resp., $\mu_T$) be the number of non-empty $H_s$ (resp., $T_s$)
blocks.
Let $n_H := \sum_{s=1}^{\mu_H} |H_s|$ (resp., $n_T := \sum_{s=1}^{\mu_T}|T_s|$)
be the number of time steps in $\cup_s H_s$ (resp., in $\cup_s T_s$) and let
 $w_{H,s} := |H_s|/n_H$ (resp., $w_{T,s} := |T_s|/n_T$) be the relative weight of data points in block $H_s$
 (resp., in block $T_s$).

We have
\begin{align*}
  \wh\vM
  & = \frac1{n-1} \sum_{t=1}^{n-1} \ve_{X_t} \ve_{X_{t+1}}^\t \\
  & = \frac{|F|}{n-1} 
    \underbrace{
      \frac{1}{|F|} \sum_{t\in F} \ve_{X_t} \ve_{X_{t+1}}^\t
     }_{\wh\vM_F} +
  \frac{n_H}{n-1}
  \underbrace{
    \sum_{s=1}^{\mu_H} w_{H,s}
    \Parens{
      \frac1{|H_s|} \sum_{t \in H_s} \ve_{X_t} \ve_{X_{t+1}}^\t
    }
  }_{\wh{\vM}_H}\\
  & \qquad
  + \frac{n_T}{n-1}
  \underbrace{
    \sum_{s=1}^{\mu_T} w_{T,s}
    \Parens{
      \frac1{|T_s|} \sum_{t \in T_s} \ve_{X_t} \ve_{X_{t+1}}^\t
    }
  }_{\wh{\vM}_T}
  .
\end{align*}

Note that $\norm{\wh{\vM}_F},\norm{\EE{\vM}}\le 1$, so the first term's effect on the deviation of $\wh \vM$ from $\vM$ is bounded by $2/(n-1)$.
We now present a bound on $\wh{\vM}_H - \bbE(\wh{\vM}_H)$; the bound on (and analysis of)
 $\wh{\vM}_T - \bbE(\wh{\vM}_T)$ is identical.

Let
\[
  \vY_s := \frac1{|H_s|} \sum_{t \in H_s} \ve_{X_t} \ve_{X_{t+1}}^\t ,
  \quad s \in [\mu_H] ,
\]
so
\[
  \wh{\vM}_H = \sum_{s=1}^{\mu_H} w_{H,s} \vY_s ,
\]
a weighted average of the random matrices $\vY_s$.
For each $s \in [\mu_H]$, the random matrix $\vY_s$ is a function of
\[ (X_t : (2s-1)a + 1 \leq t \leq 2sa + 1) \]
(note the $+1$ in the upper limit of $t$).
So $\vY_s$ and $\vY_{s+1}$ are separated by $a-1$ time steps.
When $a$ is sufficiently large, we will be able to effectively treat
the random matrices $\vY_s$ as if they were independent.
In the sequel, we shall always assume that the block length $a$
satisfies
\begin{equation}
  \label{eq:block-length}
  a \geq
   1 + \frac{1}{2\gap} \ln \frac1{\pimin}\,.
%   \geq
%   1 + \frac{1}{\gap} \ln \frac1{|H_s|\sqrt{\pimin}\gap}
\end{equation}
\todoc{By using the geometric series bound,
we can derive that it is enough
if $a\ge 1 + \frac{1}{\gap} \ln \frac1{|H_s|\sqrt{\pimin}\gap}$,
which is better when $|H_s|$ is bigger than $\gap$.
}
\todoc{As we need $a\le n/2$, we will need $n = \Omega(1/\gap\ln\frac1{\pimin})$.}

Define%\todoc{Do we want this abuse?}
\begin{align*}
  \vpi^{(H_s)}  := \frac1{|H_s|} \sum_{t \in H_s} \vpi^{(t)} , \qquad \qquad
  \vpi^{(H)}  := \sum_{s=1}^{\mu_H} w_{H,s} \vpi^{(H_s)} .
\end{align*}
Observe that
\[
  \bbE(\vY_s)
  = \Diag(\vpi^{(H_s)}) \vP
\]
so
\[
  \bbE\Parens{
    \sum_{s=1}^{\mu_H} w_{H,s} \vY_s
  } = \Diag(\vpi^{(H)}) \vP
  .
\]

Define
\[
  \vZ_s
  := \Diag(\vpi)^{-1/2} \left( \vY_s 
  - \bbE(\vY_s) \right)\Diag(\vpi)^{-1/2}
  .
\]
We apply a matrix tail inequality to the
$w_{H,s}$-weighted sum of \emph{independent} copies of the $\vZ_s$'s.
More precisely, we will apply the tail inequality to independent copies $\wt{\vZ}_s$, $s\in \iset{\mu_H}$ 
of the random variables $\vZ_s$ and then relate the weighted sum of $\wt{\vZ}_s$ to that of $\vZ_s$.
The matrix tail inequality we use is as follows:
\begin{theorem}[Matrix Bernstein inequality, Theorem~6.1.1 of \citet{tropp2015intro}]
\label{thm:mxbernstein}
Let $\vX_1,\dots,\vX_n$ be a sequence of independent, random matrices with common dimension $d_1\times d_2$.
Assume that $\EE{\vX_i} = 0$ and $\Norm{\vX_i}\le R$ for each $1\le i \le n$. Let $\vS = \sum_{i=1}^n \vX_i$ and let 
\begin{align*}
v = \max\left\{ \norm{\bbE \textstyle\sum_i \vX_i \vX_i^\top  }, 
						      \norm{\bbE \textstyle\sum_i \vX_i^\top \vX_i  }
			\right\}\,.
\end{align*}
Then, for all $t\ge 0$, 
\[
\bbP\left( \Norm{\vS} \ge t \right) \le (d_1 + d_2) \exp\left( -\frac{t^2/2}{v+R t/3}\right)\,.
\]
\end{theorem}
By inverting the bound, we get that for any $0\le \delta \le 1$, with probability $1-\delta$, 
\[
\norm{\vS} \le \sqrt{2 v \ln \frac{d_1+d_2}{\delta}} + \frac{2R}{3} \ln\frac{d_1+d_2}{\delta}\,.
\]

We see, that in order to apply this result we need to bound the range of $\vZ_s$ and 
the spectral norms of the matrices $\sum_{s=1}^{\mu_H}
w_{H,s}^2 \bbE(\vZ_s \vZ_s^\t)$ and $\sum_{s=1}^{\mu_H} w_{H,s}^2
\bbE(\vZ_s^\t \vZ_s)$ (which we call variance bounds).

\paragraph{Range bound.}
%We first determine a bound on the spectral norm of $\vZ_s$.
By the triangle inequality,
\[
  \norm{\vZ_s}
  \leq \norm{\Diag(\vpi)^{-1/2} \vY_s \Diag(\vpi)^{-1/2}}
  + \norm{\Diag(\vpi)^{-1/2} \bbE(\vY_s) \Diag(\vpi)^{-1/2}}
  \,.
\]
Clearly,
\begin{align}
\label{eq:vysbound}
  \norm{\Diag(\vpi)^{-1/2} \vY_s \Diag(\vpi)^{-1/2}} \leq \frac1{\pimin} .
\end{align}
Similarly to \eqref{eq:biasbound}, using $\norm{\vL}\le 1$,
we get 
\begin{align*}
\MoveEqLeft  \norm{\Diag(\vpi)^{-1/2} (\bbE(\vY_s) -\vM)\Diag(\vpi)^{-1/2}}
= 
\norm{ \bigl(\Diag(\vpi^{(H_s)}) \Diag(\vpi)^{-1} - \vI  \bigr) \vL } \nonumber \\
&\qquad\qquad\qquad \le
\norm{ \Diag(\vpi^{(H_s)}) \Diag(\vpi)^{-1} - \vI }\,.
\end{align*}
Then, using~\eqref{eq:dvpi-ratio-bound},
\begin{align}
\norm{ \Diag(\vpi^{(H_s)}) \Diag(\vpi)^{-1} - \vI } \le
 \frac{|H_s|(1-\gap)^{(2s-1)a}}{|H_s|\sqrt{\pimin}}
\le
 \frac{(1-\gap)^{a}}{\sqrt{\pimin}} \le 1\,,
 \label{eq:ysbias}
\end{align}
where the last inequality follows from the assumption that the
block length $a$ satisfies~\eqref{eq:block-length}.

Combining this with
$\norm{\Diag(\vpi)^{-1/2} \vM\Diag(\vpi)^{-1/2}} = \norm{\vL}\le 1$,
%\if0
%
%Since
%\[
%  \Diag(\vpi)^{-1/2} \bbE(\vY_s) \Diag(\vpi)^{-1/2}
%  =
%  \Diag(\vpi)^{-1/2}
%  \Diag(\vpi^{(H_s)})
%  \Diag(\vpi)^{-1/2}
%  \vL
%\]
%and
%\[
%  \norm{\vL} \leq 1
%  ,
%\]
%it follows that
%\[
%  \norm{\Diag(\vpi)^{-1/2} \bbE(\vY_s) \Diag(\vpi)^{-1/2}}
%  \leq \max_{i \in [d]} \frac{\pi_i^{(H_s)}}{\pi_i}\,.
%%  \leq 2
%\]
%
%Suppose the first block $F$ is of length
%\begin{equation}
%  \label{eq:block-length}
%  |F| \geq 1 + \frac{1}{\gap} \ln \frac1{\pimin\gap}
%  .
%\end{equation}
%Then
%\[
%  \frac{(1-\gap)^{\min H_s - 1}}{\pimin\gap}
%  \leq 1
%\]
%for all blocks $H_s$ (and same for the $T_s$ blocks), so
%\begin{align}
%  \max \Braces{
%    \Abs{ \frac{\pi_i^{(H_s)}}{\pi_i} } ,\,
%    \Abs{ \frac{\pi_i^{(T_s)}}{\pi_i} } ,\,
%    \Abs{ \frac{\pi_i^{(H,T)}}{\pi_i} }
%    : i \in [d] , s
%  }
%  & \leq 2
%  .
%  \label{eq:near-stationary}
%\end{align}
%
%
%
%(using the assumption that~\eqref{eq:block-length} holds).
%Therefore
%\fi
it follows that
\begin{align}
\label{eq:evysbound}
\norm{\Diag(\vpi)^{-1/2} \bbE(\vY_s) \Diag(\vpi)^{-1/2}} \le 2
\end{align}
which, together with~\eqref{eq:vysbound}, gives
\[
  \norm{\vZ_s} \leq \frac1{\pimin} + 2
%  \max_{i\in[d]} \frac{\pi_i^{(H_s)}}{\pi_i}
  .
\]

\paragraph{Variance bound.}
We now determine bounds on the spectral norm of $\sum_{s=1}^{\mu_H}
w_{H,s}^2 \bbE(\vZ_s \vZ_s^\t)$ and of $\sum_{s=1}^{\mu_H} w_{H,s}^2
\bbE(\vZ_s^\t \vZ_s)$.
Observe that
\begin{align}
  \lefteqn{
    \bbE(\vZ_s\vZ_s^\t)
  } \notag \\
  & =
  \frac1{|H_s|^2} \sum_{t \in H_s}
  \bbE\Parens{
    \Diag(\vpi)^{-1/2}
    \ve_{X_t} \ve_{X_{t+1}}^\t 
    \Diag(\vpi)^{-1}
    \ve_{X_{t+1}} \ve_{X_t}^\t
    \Diag(\vpi)^{-1/2}
  }
  \label{eq:var1}
  \\
  & \quad
  + \frac1{|H_s|^2} \sum_{ \substack{ t \neq t' \\ t,t'\in H_s} } 
  \bbE\Parens{
    \Diag(\vpi)^{-1/2}
    \ve_{X_t} \ve_{X_{t+1}}^\t 
    \Diag(\vpi)^{-1}
    \ve_{X_{t'+1}} \ve_{X_{t'}}^\t
    \Diag(\vpi)^{-1/2}
  }
  \label{eq:var2}
  \\
  & \quad
  - \Diag(\vpi)^{-1/2}
  \bbE(\vY_s)
  \Diag(\vpi)^{-1} 
  \bbE(\vY_s^\t)
  \Diag(\vpi)^{-1/2}
  .
  \label{eq:var3}
\end{align}
The first sum~\eqref{eq:var1} easily simplifies to the diagonal matrix
\begin{align*}
  \lefteqn{
    \frac1{|H_s|^2}
    \sum_{t \in H_s}
    \sum_{i=1}^d \sum_{j=1}^d \Pr(X_t = i, X_{t+1} = j) \cdot
    \frac{1}{\pi_i\pi_j} \ve_i\ve_j^\t\ve_j\ve_i^\t
  } \\
  & =
  \frac1{|H_s|^2}
  \sum_{t \in H_s}
  \sum_{i=1}^d \sum_{j=1}^d \pi_i^{(t)} P_{i,j} \cdot
  \frac{1}{\pi_i\pi_j} \ve_i\ve_i^\t
  =
  \frac1{|H_s|}
  \sum_{i=1}^d \frac{\pi_i^{(H_s)}}{\pi_i}
  \Parens{ \sum_{j=1}^d \frac{P_{i,j}}{\pi_j} }
   \ve_i \ve_i^\t
  .
\end{align*}
For the second sum~\eqref{eq:var2},  a symmetric matrix, consider
\[
  \vu^\t
  \Parens{
    \frac1{|H_s|^2} \sum_{ \substack{ t \neq t' \\ t,t'\in H_s} } 
    \bbE\Parens{
      \Diag(\vpi)^{-1/2}
      \ve_{X_t} \ve_{X_{t+1}}^\t 
      \Diag(\vpi)^{-1}
      \ve_{X_{t'+1}} \ve_{X_{t'}}^\t
      \Diag(\vpi)^{-1/2}
    }
  } \vu
\]
for an arbitrary unit vector $\vu$.
By Cauchy-Schwarz and AM/GM, this is bounded from above by
\begin{multline*}
  \frac1{2|H_s|^2}
  \sum_{ \substack{ t \neq t' \\ t,t'\in H_s} }
  \biggl[
  \bbE\Parens{
    \vu^\t
    \Diag(\vpi)^{-1/2} \ve_{X_t} \ve_{X_{t+1}}^\t \Diag(\vpi)^{-1}
    \ve_{X_{t+1}} \ve_{X_t}^\t \Diag(\vpi)^{-1/2}
    \vu
  }
  \\
  +
  \bbE\Parens{
    \vu^\t
    \Diag(\vpi)^{-1/2} \ve_{X_{t'}} \ve_{X_{t'+1}}^\t \Diag(\vpi)^{-1}
    \ve_{X_{t'+1}} \ve_{X_{t'}}^\t \Diag(\vpi)^{-1/2}
    \vu
  }
  \biggr]
  ,
\end{multline*}
which simplifies to %the sum of
\[
  \frac{|H_s|-1}{|H_s|^2}
  \vu^\t \bbE\Parens{
    \sum_{t \in H_s}
    \Diag(\vpi)^{-1/2} \ve_{X_t} \ve_{X_{t+1}}^\t \Diag(\vpi)^{-1}
    \ve_{X_{t+1}} \ve_{X_t}^\t \Diag(\vpi)^{-1/2}
  } \vu\,.
\]
%and the same expression with $\vu$ replaced by $\vv$.
The expectation is the same as that from the first
term~\eqref{eq:var1}.

Finally, the spectral norm of the third term~\eqref{eq:var3} is bounded by
\[
  \norm{\Diag(\vpi)^{-1/2} \bbE(\vY_s) \Diag(\vpi)^{-1/2}}^2
  \leq 4
  %\Parens{ \max_{i \in [d]} \frac{\pi_i^{(H_s)}}{\pi_i} }^2
\]
where we used~\eqref{eq:evysbound}, which holds when~\eqref{eq:block-length} is satisfied.
%\comment{possibly improve this to depend only on $\pi^{(H)}$?}

Therefore, by the triangle inequality and some simplifications, 
\begin{align*}
  \Norm{
    \sum_{s=1}^{\mu_H} w_{H,s}^2 \bbE(\vZ_s \vZ_s^\t)
  }
  &\leq
  \Parens{ \max_s w_{H,s} } \left[
  \max_{i\in[d]}
  \Parens{ \sum_{j=1}^d \frac{P_{i,j}}{\pi_j} }
  \frac{\pi_i^{(H)}}{\pi_i}
  + 4 \right]
  \\
  &\leq
  \Parens{ \max_s w_{H,s} } \left[
  2\max_{i\in[d]}
  \Parens{ \sum_{j=1}^d \frac{P_{i,j}}{\pi_j} }
  + 4 \right]
\end{align*}
using \eqref{eq:ysbias}, which holds 
under the assumption that~\eqref{eq:block-length} is satisfied.

We can also bound $\sum_{s=1}^{\mu_H} w_{H,s}^2 \bbE(\vZ_s^\t \vZ_s)$
in a similar way; the only difference is that the reversibility needs
to be used at one place to simplify an expectation:
\begin{align*}
\MoveEqLeft
    \frac1{|H_s|^2} \sum_{t \in H_s}
    \bbE\Parens{
      \Diag(\vpi)^{-1/2}
      \ve_{X_{t+1}} \ve_{X_t}^\t
      \Diag(\vpi)^{-1}
      \ve_{X_t}\ve_{X_{t+1}}^\t
      \Diag(\vpi)^{-1/2}
    }
	\\
  & =
  \frac1{|H_s|^2} \sum_{t \in H_s}
  \sum_{i=1}^d \sum_{j=1}^d \Pr(X_t = i, X_{t+1} = j) \cdot
  \frac{1}{\pi_i\pi_j} \ve_j\ve_j^\t
  \\
  & =
  \frac1{|H_s|^2} \sum_{t \in H_s}
  \sum_{i=1}^d \sum_{j=1}^d \pi_i^{(t)} P_{i,j} \cdot
  \frac{1}{\pi_i\pi_j} \ve_j\ve_j^\t
  \\
  & =
  \frac1{|H_s|^2} \sum_{t \in H_s}
  \sum_{j=1}^d \Parens{
    \sum_{i=1}^d \frac{\pi_i^{(t)}}{\pi_i} \cdot \frac{P_{j,i}}{\pi_i}
  } \ve_j\ve_j^\t & (\text{by }\eqref{eq:reversibility})
  .
\end{align*}
As before, we get
\begin{align*}
  \Norm{
    \sum_{s=1}^{\mu_H} w_{H,s}^2 \bbE(\vZ_s^\t\vZ_s)
  }
  &\leq
  \Parens{ \max_s w_{H,s} } \left[
  \max_{i\in[d]}
  \Parens{
    \sum_{j=1}^d \frac{P_{i,j}}{\pi_j}
    \cdot \frac{\pi_j^{(H)}}{\pi_j}
  }
  + 4 \right]
  \\
&  \leq
  \Parens{ \max_s w_{H,s} }
  \left[ 2 \max_{i\in[d]}
  \Parens{
    \sum_{j=1}^d \frac{P_{i,j}}{\pi_j}
  }
  + 4\right]
\end{align*}
using again \eqref{eq:ysbias}.
%(using the assumption that~\eqref{eq:block-length} holds).

\paragraph{Independent copies bound.}
Let $\wt\vZ_s$ for $s \in [\mu_H]$ be independent copies of
$\vZ_s$ for $s \in [\mu_H]$.
Applying \cref{thm:mxbernstein} to the weighted sum of these random variables, we  get that 
with probability at least $1-\delta$, the spectral norm of
$\sum_{s=1}^{\mu_H} w_{H,s} \wt\vZ_s$ is at most
\begin{align*}
\MoveEqLeft  \Norm{\sum_{s=1}^{\mu_H} w_{H,s} \wt\vZ_s}
  \leq
  \sqrt{
    4\Parens{ \max_s w_{H,s} } \Parens{
      d_{\vP}
      + 2
    }
    \ln(4d/\delta)
  }
  \\
& \qquad\qquad\qquad  +
  \frac23 \max_s w_{H,s}
  \Parens{
    \frac1{\pimin} + 2
  }
  \ln(4d/\delta)
\end{align*}
where
\begin{align*}
  d_{\vP}
  & := \max_{i \in [d]} \sum_{j=1}^d \frac{P_{i,j}}{\pi_j}
  \leq
  \frac{1}{\pimin}
  \,.
\end{align*}

For simplicity, we assume that all of the blocks have length exactly $a$.
Under this assumption, 
the above bound (also with looser constants) gives that
with probability at least $1-\delta$:
\begin{equation}
\label{eq:indbound}
  \Norm{\frac1{\mu_H} \sum_{s=1}^{\mu_H} \wt\vZ_s}
  \leq
  \sqrt{
    \frac{12\ln(4d/\delta)}{\mu_H\pimin}
  }
  +
  \frac{2\ln(4d/\delta)}{\mu_H\pimin}
  \,.
\end{equation}

\paragraph{The actual bound.}
To bound the probability that $\norm{\sum_{s=1}^{\mu_H} w_{H,s}
\vZ_s}$ is large, we appeal to the following result; a consequence of
Corollary 2.7 of~\citet{Yu94}.
Let $\bbP$ denote the joint distribution of $(\vZ_s : s \in [\mu_H])$;
let $\bbP_s$ be its marginal over $\vZ_s$, and let $\bbP_{1:s+1}$ be
its marginal over $(\vZ_1,\vZ_2,\dotsc,\vZ_{s+1})$.
Let $\wt\bbP$ be the product distribution formed from the marginals
$\bbP_1,\bbP_2,\dotsc,\bbP_{\mu_H}$, so $\wt\bbP$ is the joint
distribution of $(\wt\vZ_s : s \in [\mu_H])$.
For any event $E$,
\[
  |\bbP(E) - \wt\bbP(E)| \leq (\mu_H-1) \beta(\bbP)
\]
where
\[
  \beta(\bbP)
  := \max_{1 \leq s \leq \mu_H-1}
  \bbE\Parens{
    \Norm{
      \bbP_{1:s+1}(\cdot\,|\vZ_1,\vZ_2,\dotsc,\vZ_s) - \bbP_{s+1}
    }_{\tv}
  }
  \,.
\]
Here, $\norm{\mu}_{\tv}$ denotes the total variation norm.
Recalling the definition of beta mixing,
the number $\beta(\bbP)$ can be recognized to be the first beta mixing coefficient of the stochastic process $\{\vZ_s\}$.
This result implies that the bound stated above for
$\norm{\sum_{s=1}^{\mu_H} w_{H,s} \wt\vZ_s}$ also holds for
$\norm{\sum_{s=1}^{\mu_H} w_{H,s} \vZ_s}$, except the failure
probability increases from $\delta$ to $\delta +
(\mu_H-1)\beta(\bbP)$.

It is not hard to see that $\beta(\bbP)$ in turn is upper bounded by 
the $(a-1)$-th beta mixing coefficient of the Markov chain $\{X_t\}$ (because the first
index in $H_{s+1}$ and the last index in $H_s$ are at a distance of $(a-1)$).
Hence, 
%of a signed measure $\mu$ over a measurable space 
%$(\Omega,\mathcal{A})$, defined by 
%$\Norm{\mu}_{\tv} = \frac12(\mu^+(\Omega)+\mu^-(\Omega))$, 
%where $\mu^+$ and $\mu^-$ are the positive and negative parts of measure $\mu$.
by a well-known result 
\[
  \beta(\bbP)
  \leq
  \frac{1}{\pimin} \exp(-(a-1)\gap)
\]
\citep[see, for instance,][]{bradley05}. \todoc{Which theorem? I did not find this result there. Can anyone please refine this citation?}
It follows that if the block length $a$ satisfies
\begin{equation}
  \label{eq:block-length2}
  a \geq 1 + \frac{1}{\gap}\ln\frac{n}{\delta\pimin}
  ,
\end{equation}
then the failure probability increases from $\delta$ to at most
$2\delta$.
%Note that this condition implies \eqref{eq:block-length}.

Combining \eqref{eq:biasbound},\eqref{eq:indbound}, we get that if
\[
  a := \left\lceil 1 + \frac{1}{\gap}\ln\frac{2n}{\pimin\delta} \right\rceil
\]
then for any $0\le \delta \le 1$, with probability at least $1-\delta$, with $\mu := \mu_H = \mu_T=n/(2a)$,
\begin{align*}
%\errm-\EE{\errm}
%& =  \Diag(\vpi)^{-1/2}\Parens{\wh\vM - \bbE[\wh\vM]} \Diag(\vpi)^{-1/2} 
\norm{\wh\vM - \bbE \wh\vM }
& \le 
  2\sqrt{
    \frac{12\ln(16d/\delta)}{\mu\pimin}
  }
  +
  \frac{4\ln(16d/\delta)}{\mu\pimin}
  + 
  \frac{2}{n-1}\,,
%  +
%  \frac1{(n-1)\gap \sqrt{\pimin}} 
\end{align*}
which, after the definition of $a$ is plugged in,
together with \eqref{eq:errmdev1},
 gives that as long as $n$ is big enough so that $2a\le n$ holds, \todoc{Second condition on $n$.}
\begin{align*}
\norm{\errm-\EE{\errm}}
\le \frac{C}{\pimin}
\sqrt{
    \frac{\ln(d/\delta)\ln(\frac{n}{\delta\pimin})}{\pimin\gap n}
  }
  \,.
\end{align*}
\subsubsection{The bound on $\norm{\errm}$}
As the RHS of the previous bound dominates the RHS of \eqref{eq:biasbound},
we get 
\begin{align}
\label{eq:doubletbound}
\norm{\errm}
\le \frac{C}{\pimin}
\sqrt{
    \frac{\ln(d/\delta)\ln(\frac{n}{\delta\pimin})}{\pimin\gap n}
  }\,.
\end{align}

\subsection{Overall error bound}
Consider an event when both~\eqref{eq:singletonbound} and \eqref{eq:doubletbound} hold.
Then, using that $\ln(d/\delta) + \ln(1/\pimin) \le \ln(d/\delta)\ln(\frac{n}{\delta\pimin})$, we get
\begin{align*}
\norm{\wh\vL - \vL} \le 
\frac{C}{\pimin}
\sqrt{
    \frac{\ln(d/\delta)\ln(\frac{n}{\delta\pimin})}{\pimin\gap n}
  }\,,
\end{align*}
which together with \eqref{eq:singletonboundsimple},
finishes the proof of \cref{lem:err}.
\hfill $\blacksquare$

\if0

Plugging the previous bound and~\eqref{eq:singletonbound} into~\eqref{eq:vlbound}, we see that
\[
  a := \left\lceil 1 + \frac{1}{\gap}\ln\frac{n}{\pimin\delta} \right\rceil
\]
so that~\eqref{eq:block-length2} holds (which implies~\eqref{eq:block-length}).
Then, we have the following error bound: for some absolute constant
$C>0$, with probability at least $1-\delta$, \todoc{From
$\norm{\errp}$, we get $(\pimin\gap^2n)^{-1/2}$! Where is the cubic term coming from? Why is it not dominated by the other term? All the subsequent error bounds will need to be updated.}
\begin{equation}
  \norm{\wh\vL - \vL}
  \leq
  C
  \sqrt{\frac{\ln(d/\delta)\ln\frac{n}{\delta\pimin}}{n\pimin\gap}}
  +
  C
  \Parens{
    \frac{\ln(d/\delta)\ln\frac{n}{\delta\pimin}}{n\pimin\gap}
  }^3
\end{equation}
and for all $i \in [d]$,
\begin{equation}
  \Abs{ \frac{\hat\pi_i}{\pi_i} - 1 }
  \leq 
  C
  \sqrt{\frac{\ln(d/\delta)\ln\frac{n}{\delta\pimin}}{n\pimin\gap}}
  +
  C
  \frac{\ln(d/\delta)\ln\frac{n}{\delta\pimin}}{n\pimin\gap}
  .
\end{equation}
In this event, we have
\begin{equation}
  |\hatgap - \gap|
  \leq
  C'
  \max\Braces{
    \sqrt{\frac{\ln(d/\delta)\ln\frac{n}{\delta\pimin}}{n\pimin\gap}}
    ,\,
    \Parens{
      \frac{\ln(d/\delta)\ln\frac{n}{\delta\pimin}}{n\pimin\gap}
    }^3
  }
\end{equation}
for some other absolute constant $C'>0$.
This completes the proof of Theorem~\ref{thm:err}.
\hfill $\blacksquare$
\fi

%In terms of matching our lower bounds for estimating $\gap$, we are
%pretty close: ignoring log factors, our sample complexity is
%\[
%  \tilde{O}\Parens{ \frac1{\pimin\gap} }
%\]
%whereas the lower bound is
%\[
%  \tilde{\Omega}\Parens{ \frac{d}{\gap} + \frac1{\pimin} }
%  .
%\]
%So when $\vpi$ is not too far from uniform---meaning that $\pimin =
%\Omega(1/d)$---we are matching the lower bound up to logarithmic
%factors.

\if0
\subsection{Empirical upper confidence bound}

Now, we actually want to be able to provide an empirical upper
confidence bound on $\gap$, which is where we had to do that
calculation from NIPS that introduced extra factors of $1/\gap$ and
$1/\pimin$.
I haven't had time to write out these details, but needless to say, it
will be messy.

We know that
\[
  \pimin \geq \hatpimin - \sqrt{\frac{C\hatpimin}{n\gap}} -
  \frac{C}{n\gap}
  \geq \frac{\hatpimin}{2} - \frac{3C}{n\gap}
  .
\]
for some suitable ``constant'' (log term) $C>0$.
Therefore, ignoring the higher-order forms, we have
\[
  |\hatgap-\gap|
  \leq \sqrt{\frac{C'}{n\hatpimin\gap/2-C''}}
  .
\]
Unfortunately, I don't think we can get an upper bound on $\gap$ this
way, since we can't really know when the denominator on the RHS is
positive. \todoc{Why not? We have to add the constraint that this is positive
and we could solve the inequalities numerically.}

Our non-rigorous approach is to use a first-order approximation.
Let $x := \gap - \hatgap$ and $y := \pimin - \hatpimin$, and use a
linear apporximation to
\[
  b(x,y) = \sqrt{\frac{C}{n(\hatgap-x)(\hatpimin-y)}}
\]
as a bound on $|x|$ and $|y|$.
The first-order approximation is
\[
  b(x,y)
  \approx \tilde{b}(x,y)
  :=
  \sqrt{\frac{C}{n\hatgap\hatpimin}}
  + \frac12\sqrt{\frac{C'}{n\hatgap^3\hatpimin}}x
  + \frac12\sqrt{\frac{C'}{n\hatgap\hatpimin^3}}y
  .
\]
The approximation is a valid upper-bound when $|x| \leq
\frac{\hatgap}2$ and $|y| \leq \frac{\hatpimin}2$; such will be true
when
\[
  \sqrt{\frac{C}{n\gap\pimin}} \leq \min\Braces{ \frac{\hatgap}{3} ,\,
  \frac{\hatpimin}{3} }
\]
i.e., when
\[
  n = \Omega\Parens{ \frac1{\gap^3\pimin} + \frac1{\gap\pimin^3} } .
\]
Assuming this, we find $\bar{x},\bar{y} \geq 0$ such that
$\tilde{b}(\bar{x},\bar{y}) = \bar{x}$ and $\tilde{b}(\bar{x},\bar{y})
= \bar{y}$:
\begin{align*}
  \Parens{
    1 - \frac12\sqrt{\frac{C}{n\hatgap^3\hatpimin}}
  } \bar{x}
  - \frac12\sqrt{\frac{C}{n\hatgap\hatpimin^3}}
  \bar{y}
  & =
  \sqrt{\frac{C}{n\hatgap\hatpimin}}
  \\
  - \frac12\sqrt{\frac{C}{n\hatgap^3\hatpimin}}
  \bar{x}
  + \Parens{
    1 - \frac12\sqrt{\frac{C}{n\hatgap\hatpimin^3}}
  }
  \bar{y}
  & =
  \sqrt{\frac{C}{n\hatgap\hatpimin}}
  .
\end{align*}
The solutions are
\begin{align*}
  \bar{x} 
  = \bar{y}
  & = \frac{1}{
    1 - \frac12\sqrt{\frac{C}{n\hatgap^3\hatpimin}}
    - \frac12\sqrt{\frac{C}{n\hatgap\hatpimin^3}}
  }
  \sqrt{\frac{C}{n\hatgap\hatpimin}}
  .
\end{align*}
\fi
