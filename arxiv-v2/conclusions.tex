The construction used in \cref{thm:combined} applies more generally:
Given a confidence interval of the form $I_n = I_n(\gap,\pimin,\delta)$ 
for some confidence level $\delta$
and a fully empirical confidence set $E_n(\delta)$ for $(\gap,\pimin)$  for the same level,
$I_n' = E_n(\delta) \cap \cup_{(\gamma,\pi)\in E_n(\delta)} I_n(\gamma,\pi,\delta)$ is a valid
fully empirical $2\delta$-level confidence interval whose asymptotic width
matches that of $I_n$ up to lower order terms under reasonable assumptions on $E_n$ and $I_n$.
In particular, this suggests that future work should focus on 
closing the gap between the lower and upper bounds on the accuracy
of point-estimation. Another interesting direction is to 
reduce the computation cost: The current cubic cost in the number of states
can be too high even when the number of states is only moderately
large.

Perhaps more important, however, is to extend 
our results to large state space Markov chains:
In most practical applications the state space is continuous
or is exponentially large in some natural parameters.
As follows from our lower bounds, without further assumptions,
the problem of fully data dependent estimation of the mixing time
is intractable for information theoretical reasons.
Interesting directions for future work thus must consider Markov
chains with specific structure. Parametric classes of Markov chains,
including but not limited to Markov chains with factored transition kernels
with a few factors, are a promising candidate for such future investigations.
The results presented here are a first step in the ambitious research agenda
outlined above, and we hope that they will
serve as a point of departure
for
further insights
in the area of fully empirical estimation of Markov chain 
parameters based on a single sample path.

