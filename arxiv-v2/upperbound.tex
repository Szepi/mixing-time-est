
In this section, we prove \cref{thm:err}.

\subsection{Accuracy of $\hatpimin$}

We start by proving the deviation bound on $\pimin-\hatpimin$, from
which we may easily deduce \cref{eq:piminbound} in \cref{thm:err}.
\begin{lemma}
  \label{lem:hatpimin}
  Pick any $\delta \in (0,1)$, and let
  \begin{equation}
    \veps_n :=
    \frac{ \ln\Parens{\frac{d}\delta\sqrt{\frac{2}{\pimin}}} }{\gap n}
    .
    \label{eq:singleton-veps}
  \end{equation}
  With probability at least $1-\delta$, the following inequalities
  hold simultaneously:
  \begin{align}
    \Abs{\hat{\pi}_i-\pi_i}
    & \le
    \sqrt{8\pi_i(1-\pi_i) \veps_n}
    + 20 \veps_n
    \quad \text{for all $i \in [d]$}
    ;
    \label{eq:singletonboundsimple}
    \\
    \Abs{\hatpimin - \pimin}
    & \leq
    4\sqrt{\pimin\veps_n}
    + 47\veps_n
    .
    \label{eq:hatpimin}
  \end{align}
\end{lemma}
\begin{proof}
  We use the following Bernstein-type inequality for Markov chains
  from \citet[][Theorem~3.8]{paulin15}: letting $\bbP^{\vpi}$ denote
  the probability with respect to the stationary chain (where the
  marginal distribution of each $X_t$ is $\vpi$), we have for every
  $\eps>0$,
  \[
    \bbP^{\vpi}
    \Parens{
      \abs{ \hat\pi_i-\pi_i} > \eps
    } \le
    2\exp\Parens{
      -\frac{
        n\gap\eps^2
      }{
        4\pi_i(1-\pi_i)+10\eps
      }
    },
    \qquad i\in[d]
    .
  \]
  To handle possibly non-stationary chains, as is our case, we combine
  the above inequality with \citet[][Proposition 3.14]{paulin15}, to
  obtain for any $\eps>0$,
  \begin{align*}
    \bbP
    \Parens{
      \abs{ \hat\pi_i-\pi_i} > \eps
    }
    &\le
    \sqrt{
      \frac1\pimin
      \bbP^{\vpi}
      \Parens{
        \abs{ \hat\pi_i-\pi_i} > \eps
      }
    } 
    \le
    \sqrt{\frac2\pimin}
    \exp\Parens{
      -\frac{
        n\gap\eps^2
      }{
        8\pi_i(1-\pi_i)+20\eps
      }
    }.
  \end{align*}
  Using this tail inequality with $\eps :=
  \sqrt{8\pi_i(1-\pi_i)\veps_n} + 20\veps_n$ and a union bound over
  all $i \in [d]$ implies that the inequalities
  in~\cref{eq:singletonboundsimple} hold with probability at least
  $1-\delta$.

  Now assume this $1-\delta$ probability event holds; it remains to
  prove that \cref{eq:hatpimin} also holds in this event.
  Without loss of generality, we assume that $\pimin = \pi_1 \leq
  \pi_2 \leq \dotsb \leq \pi_d$.
  Let $j \in [d]$ be such that $\hatpimin = \hat\pi_j$.
  By \cref{eq:singletonboundsimple}, we have $|\pi_i - \hat\pi_i| \leq
  \sqrt{8\pi_i\veps_n} + 20\veps_n$ for each $i \in \{1,j\}$.
  Since $\hatpimin \leq \hat\pi_1$,
  \[
    \hatpimin - \pimin
    \leq \hat\pi_1 - \pi_1
    \leq \sqrt{8\pimin\veps_n} + 20\veps_n
    \leq \pimin + 22\veps_n
  \]
  where the last inequality follows by the AM/GM inequality.
  Furthermore, using the fact that $a \leq b\sqrt{a} + c \Rightarrow a
  \leq b^2 + b\sqrt{c} + c$ for nonnegative numbers $a, b, c \geq
  0$~\citep[see, e.g.,][]{BBL04} with the inequality $\pi_j \leq
  \sqrt{8\veps_n} \sqrt{\pi_j} + \parens{\hat\pi_j + 20\veps_n}$ gives
  \[
    \pi_j
    \leq 
    \hat\pi_j
    + \sqrt{8(\hat\pi_j + 20\veps_n)\veps_n}
    + 28\veps_n
    .
  \]
  Therefore
  \[
    \pimin - \hatpimin
    \leq \pi_j - \hat\pi_j
    \leq \sqrt{8(\hatpimin + 20\veps_n)\veps_n} + 28\veps_n
    \leq \sqrt{8(2\pimin + 42\veps_n)\veps_n} + 28\veps_n
    \leq 4\sqrt{\pimin\veps_n} + 47\veps_n
  \]
  where the second-to-last inequality follows from the above bound on
  $\hatpimin - \pimin$, and the last inequality uses $\sqrt{a+b} \leq
  \sqrt{a} + \sqrt{b}$ for nonnegative $a,b \geq 0$.
\end{proof}

%%%%%%%%%%%%%%%%%%%%%%%%%%%%%%%%%%%%%%%%%%%%%%%%%%%%%%
%%%%%%%%%%%%%%%%%%%%%%%%%%%%%%%%%%%%%%%%%%%%%%%%%%%%%%
%%%%%%%%%%%%%%%%%%%%%%%%%%%%%%%%%%%%%%%%%%%%%%%%%%%%%%
%%%%%%%%%%%%%%%%%%%%%%%%%%%%%%%%%%%%%%%%%%%%%%%%%%%%%%

\subsection{Accuracy of $\hatgap$}

Let us now turn to proving \cref{eq:gapbound}, i.e., 
the bound on the error of the spectral gap estimate $\hatgap$.
The accuracy of $\hatgap$ is based on the accuracy of $\Sym(\wh\vL)$
in approximating $\vL$ via Weyl's inequality:
\begin{align*}
  |\hat\lambda_i - \lambda_i|
  \leq \norm{\Sym(\wh\vL) - \vL}
  \quad \text{for all }i \in [d] .
\end{align*}
Moreover, the triangle inequality implies that symmetrizing $\wh\vL$
can only help:
\begin{align*}
  \norm{\Sym(\wh\vL) - \vL} \leq \norm{\wh\vL - \vL} .
\end{align*}
Therefore, we can deduce \cref{eq:gapbound} in \cref{thm:err} from the
following lemma.
\begin{lemma}
  \label{lem:gap}
  There exists an absolute constant $C>0$ such that the following
  holds.
  For any $\delta \in (0,1)$, with probability at least $1-\delta$,
  the bounds from \cref{lem:hatpimin} hold, and
  \begin{equation*}
    \norm{\wh\vL - \vL}
    \leq
    C\,
      \sqrt{
        \frac{
          \log\Parens{\frac{d}{\delta}}
          \log\Parens{\frac{n}{\pimin\delta}}
        }{\pimin\gap n}
      }
      +
      C\,
      \frac{\log\Parens{\frac{1}{\gap}}}{\gap n}
  \end{equation*}
\end{lemma}
The remainder of this section is devoted to proving this
\lcnamecref{lem:gap}.

The error $\wh\vL - \vL$ may be written as
\[
  \wh\vL - \vL
  = \errm + \errp \vL + \vL \errp + \errp \vL \errp
  + \errp \errm + \errm \errp + \errp \errm \errp\,,
\]
where
\begin{align*}
  \errp & := \Diag(\hat\vpi)^{-1/2} \Diag(\vpi)^{1/2} - \vI , \\
  \errm & := \Diag(\vpi)^{-1/2} \Parens{
    \wh\vM - \vM
  } \Diag(\vpi)^{-1/2} .
\end{align*}
Therefore
\begin{align*}
  \norm{\wh\vL - \vL}
  \leq \norm{\errm} +
  \Parens{
    \norm{\errm} + \norm{\vL}
  }
  \Parens{
    2\norm{\errp} + \norm{\errp}^2
  }
  .
\end{align*}
Since $\norm{\vL} \leq 1$ and $\norm{\wh\vL} \leq 1$~\citep[Lemma
12.1]{LePeWi08}, we also have $\norm{\wh\vL - \vL} \leq 2$.
Therefore,
\begin{align}
  \norm{\wh\vL - \vL}
  \leq
  \min\Braces{
    \norm{\errm} +
    \Parens{
      \norm{\errm} + \norm{\vL}
    }
    \Parens{
      2\norm{\errp} + \norm{\errp}^2
    }
    ,\,
    2
  }
  \leq 3(\norm{\errm}+\norm{\errp})
  .
  \label{eq:vlsimple}
\end{align}


\subsection{A bound on $\norm{\errp}$}

Since $\errp$ is diagonal,
\begin{align*}
  \norm{\errp}
  = \max_{i\in[d]}
  \Abs{
    \sqrt{\frac{\pi_i}{\hat\pi_i}} - 1
  }
  .
\end{align*}
Assume that
\begin{equation}
  n \geq \frac{108 \ln\Parens{\frac{d}\delta\sqrt{\frac{2}{\pimin}}}
  }{\pimin\gap}
  ,
  \label{eq:min-n1}
\end{equation}
in which case
\[
  \sqrt{8\pi_i(1-\pi_i) \veps_n} + 20\veps_n
  \leq \frac{\pi_i}{2}
\]
where $\veps_n$ is as defined in \cref{eq:singleton-veps}.
Therefore, in the $1-\delta$ probability event from
\cref{lem:hatpimin}, we have $|\pi_i - \hat\pi_i| \leq \pi_i/2$ for
each $i \in [d]$, and moreover, $2/3 \leq \pi_i/\hat\pi_i \leq 2$ for
each $i \in [d]$.
For this range of $\pi_i/\hat\pi_i$, we have
\[
  \Abs{ \sqrt{\frac{\pi_i}{\hat\pi_i}} - 1 }
  \leq 
  \Abs{ \frac{\hat\pi_i}{\pi_i} - 1 }
  .
\]
We conclude that if $n$ satisfies \cref{eq:min-n1}, then in this
$1-\delta$ probability event from \cref{lem:hatpimin},
\begin{align}
  \norm{\errp}
  & \leq
  \max_{i \in [d]}
  \Abs{
    \frac{\hat\pi_i}{\pi_i} - 1
  }
  \leq
  \max_{i \in [d]}
  \frac{
    \sqrt{8\pi_i(1-\pi_i) \veps_n} + 20 \veps_n
  }{\pi_i}
  \notag \\
  & \leq
  \sqrt{\frac{8\veps_n}{\pimin}} + \frac{20\veps_n}{\pimin}
  =
  \sqrt{
    \frac{
      8\ln\Parens{\frac{d}\delta\sqrt{\frac{2}{\pimin}}}
    }{
      \pimin\gap n
    }
  }
  +
  \frac{
    20\ln\Parens{\frac{d}\delta\sqrt{\frac{2}{\pimin}}}
  }{
    \pimin\gap n
  }
  .
  \label{eq:singletonbound}
\end{align}

\subsection{Accuracy of doublet frequency estimates (bounding $\norm{\errm}$)}
\label{sec:pairwise}
In this section we prove a bound on $\norm{\errm}$.
For this, we decompose
$\errm = \Diag(\vpi)^{-1/2}(\wh\vM-\vM)\Diag(\vpi)^{-1/2} $
into $\EE{\errm}$ and 
$\errm-\EE{\errm}$, the first measuring the effect of a non-stationary start of the chain,
while the second measuring the variation due to randomness.
\if0
into the sum of
\[
  \Diag(\vpi)^{-1/2}
  \left(\bbE( \wh\vM )- \vM\right)
  \Diag(\vpi)^{-1/2}
\]
and
\begin{multline*}
  \Diag(\vpi)^{-1/2} \left(\wh\vM- \bbE(\wh\vM)\right) \Diag(\vpi)^{-1/2}
  \\
  =
  \frac{n_H}{n-1}
  \Diag(\vpi)^{-1/2}\Parens{
    \wh{\vM}_H - \bbE(\wh{\vM}_H)
  }\Diag(\vpi)^{-1/2}
  \\
  + \frac{n_T}{n-1}
  \Diag(\vpi)^{-1/2}\Parens{
    \wh{\vM}_T - \bbE(\wh{\vM}_T)
  }\Diag(\vpi)^{-1/2}
  .
\end{multline*}
\fi

\subsubsection{Bounding $\norm{\EE{\errm}}$: The price of a non-stationary start.}
Let $\vpi^{(t)}$ be the distribution of states at time step $t$.
We will make use of the following proposition, which can be derived by
following \citet[Proposition 1.12]{MoTe06}:
\begin{proposition}
For $t\ge 1$, let $\vUpsilon^{(t)}$ be the vector with $\Upsilon^{(t)}_i= \frac{\pi_i^{(t)}}{\pi_i}$ and let
$\norm{\cdot}_{2,\vpi}$ denote the $\vpi$-weighted $2$-norm
\begin{align}
\label{eq:vv}
  \norm{\vv}_{2,\vpi} := \Parens{ \sum_{i=1}^d \pi_i v_i^2 }^{1/2}
  .
\end{align}
 Then,
\begin{equation}
  \label{eq:L2-contraction}
  \norm{ \vUpsilon^{(t)} - \v1 }_{2,\vpi} \leq
  \frac{(1-\gap)^{t-1}}{\sqrt{\pimin}} \,.
\end{equation}
\end{proposition}
An immediate corollary of this result is that 
\begin{align}
\label{eq:dvpi-ratio-bound}
\Norm{  \Dvpit \Dvpi^{-1} - \vI  } \le \frac{(1-\gap)^{t-1}}{\pimin}\,.
\end{align}
Now note that
\begin{align*}
\bbE (\wh\vM) = \frac{1}{n-1} \sum_{t=1}^{n-1} \Dvpit \vP\,
\end{align*}
and thus
\begin{align*}
\EE{\errm} &=
  \Dvpi^{-1/2}
  \left(\bbE( \wh\vM )- \vM\right)
  \Dvpi^{-1/2}\\
& =  
  \frac{1}{n-1} \sum_{t=1}^{n-1}
  \Dvpi^{-1/2} (\Dvpit  - \Dvpi) \vP \Dvpi^{-1/2} \\
& =  
  \frac{1}{n-1} \sum_{t=1}^{n-1}
  \Dvpi^{-1/2} (\Dvpit \Dvpi^{-1} - \vI) \vM \Dvpi^{-1/2} \\
& =  
  \frac{1}{n-1} \sum_{t=1}^{n-1}
   (\Dvpit \Dvpi^{-1} - \vI) \vL \,.
\end{align*}
Combining this, $\norm{\vL}\le 1$ and~\cref{eq:dvpi-ratio-bound}, we get
\begin{align}
  \norm{\bbE(\errm)}
  \le \frac{1}{(n-1)\pimin} \sum_{t=1}^{n-1} (1-\gap)^{t-1}
  \le \frac{1}{(n-1)\gap\pimin}
  .
  \label{eq:biasbound}
\end{align}

\subsubsection{
Bounding $\norm{\errm-\EE{\errm}}$: 
Application of a matrix tail inequality}
\label{sec:pairwise-tailbound}

In this section we analyze the deviations of $\errm-\EE{\errm}$. By the definition of $\errm$,
\begin{align}
\label{eq:errmdev1}
\norm{\errm-\EE{\errm}}
 =  \norm{\Diag(\vpi)^{-1/2}\Parens{\wh\vM - \bbE \wh\vM } \Diag(\vpi)^{-1/2}}
 \,.
\end{align}
The matrix $\wh\vM - \EE{\wh\vM}$  is defined 
as a sum of dependent centered random matrices.
We will use the blocking technique of \citet{Bernstein27} 
to relate the likely deviations of this matrix to that of
a sum of independent centered random matrices.
The deviations of these will then bounded
with the help of a Bernstein-type matrix tail inequality due to \citet{tropp2015intro}.

We divide $[n-1]$ into contiguous blocks of time steps; each has size
$a \leq n/3$ except possibly the first block, which has size between
$a$ and $2a-1$.
Formally, let $a' := a + ((n-1) \bmod a) \leq 2a-1$, and define
\begin{align*}
  F & := [a'] , \\
  H_s & := \{ t \in [n-1] : a' + 2(s-1)a + 1 \leq t \leq a' + (2s-1)a \} , \\
  T_s & := \{ t \in [n-1] : a' + (2s-1)a + 1 \leq t \leq a' + 2sa \} ,
\end{align*}
for $s=1,2,\dotsc$.
Let $\mu_H$ (resp., $\mu_T$) be the number of non-empty $H_s$ (resp., $T_s$)
blocks.
Let $n_H := a\mu_H$ (resp., $n_T := a\mu_T$) be the number of time
steps in $\cup_s H_s$ (resp., $\cup_s T_s$).
We have
\begin{align}
  \wh\vM
  & = \frac1{n-1} \sum_{t=1}^{n-1} \ve_{X_t} \ve_{X_{t+1}}^\t
  \notag \\
  & = \frac{a'}{n-1} \cdot
    \underbrace{
      \frac{1}{a'} \sum_{t\in F} \ve_{X_t} \ve_{X_{t+1}}^\t
     }_{\wh\vM_F} +
  \frac{n_H}{n-1} \cdot
  \underbrace{
    \frac1{\mu_H}
    \sum_{s=1}^{\mu_H}
    \Parens{
      \frac1a \sum_{t \in H_s} \ve_{X_t} \ve_{X_{t+1}}^\t
    }
  }_{\wh{\vM}_H}
  \notag \\
  & \qquad
  + \frac{n_T}{n-1} \cdot
  \underbrace{
    \frac1{\mu_T}
    \sum_{s=1}^{\mu_T}
    \Parens{
      \frac1a \sum_{t \in T_s} \ve_{X_t} \ve_{X_{t+1}}^\t
    }
  }_{\wh{\vM}_T}
  .
  \label{eq:blocking}
\end{align}
Here, $\ve_i$ is the $i$-th coordinate basis vector, so $\ve_i
\ve_j^\t \in \{0,1\}^{d \times d}$ is a $d \times d$ matrix of all
zeros except for a $1$ in the $(i,j)$-th position.

The contribution of the first block is easily bounded using the
triangle inequality:
\begin{multline}
  \frac{a'}{n-1}
  \Norm{
    \Diag(\vpi)^{-1/2}
    \Parens{
      \wh\vM_F - \bbE(\wh\vM_F)
    }
    \Diag(\vpi)^{-1/2}
  }
  \\
  \leq
  \frac1{n-1} \sum_{t \in F}
  \Braces{
    \Norm{
      \frac{
        \ve_{X_t} \ve_{X_{t+1}}^\t
      }{\sqrt{\pi_{X_t}\pi_{X_{t+1}}}}
    }
    +
    \Norm{
      \bbE\Parens{
        \frac{
          \ve_{X_t} \ve_{X_{t+1}}^\t
        }{\sqrt{\pi_{X_t}\pi_{X_{t+1}}}}
      }
    }
  }
  \leq
  \frac{2a'}{\pimin(n-1)}
  .
  \label{eq:first-block}
\end{multline}

It remains to bound the contributions of the $H_s$ blocks and the
$T_s$ blocks.
We just focus on the the $H_s$ blocks, since the analysis is identical
for the $T_s$ blocks.

Let
\[
  \vY_s := \frac1a \sum_{t \in H_s} \ve_{X_t} \ve_{X_{t+1}}^\t ,
  \quad s \in [\mu_H] ,
\]
so
\[
  \wh{\vM}_H = \frac1{\mu_H} \sum_{s=1}^{\mu_H} \vY_s ,
\]
an average of the random matrices $\vY_s$.
For each $s \in [\mu_H]$, the random matrix $\vY_s$ is a function of
\[ (X_t : a' + 2(s-1)a + 1 \leq t \leq a' + (2s-1)a + 1) \]
(note the $+1$ in the upper limit of $t$),
so $\vY_{s+1}$ is $a$ time steps ahead of $\vY_s$.
When $a$ is sufficiently large, we will be able to effectively treat
the random matrices $\vY_s$ as if they were independent.
In the sequel, we shall always assume that the block length $a$
satisfies
\begin{equation}
  \label{eq:block-length}
  a \geq
  a_\delta
  :=
  \frac{1}{\gap} \ln \frac{2(n-2)}{\delta\pimin}
\end{equation}
for $\delta \in (0,1)$.

Define
\begin{align*}
  \vpi^{(H_s)}  := \frac1a \sum_{t \in H_s} \vpi^{(t)} , \qquad \qquad
  \vpi^{(H)}  := \frac1{\mu_H} \sum_{s=1}^{\mu_H} \vpi^{(H_s)} .
\end{align*}
Observe that
\[
  \bbE(\vY_s)
  = \Diag(\vpi^{(H_s)}) \vP
\]
so
\[
  \bbE\Parens{
    \frac1{\mu_H} \sum_{s=1}^{\mu_H} \vY_s
  } = \Diag(\vpi^{(H)}) \vP
  .
\]

Define
\[
  \vZ_s
  := \Diag(\vpi)^{-1/2} \left( \vY_s 
  - \bbE(\vY_s) \right)\Diag(\vpi)^{-1/2}
  .
\]
We apply a matrix tail inequality to the average of \emph{independent}
copies of the $\vZ_s$'s.
More precisely, we will apply the tail inequality to independent
copies $\wt{\vZ}_s$, $s\in \iset{\mu_H}$ of the random variables
$\vZ_s$ and then relate the average of $\wt{\vZ}_s$ to that of
$\vZ_s$.
The following probability inequality is from \citet[Theorem
6.1.1.]{tropp2015intro}.
\begin{theorem}[Matrix Bernstein inequality]
\label{thm:mxbernstein}
Let $\vQ_1,\vQ_2,\dotsc,\vQ_m$ be a sequence of independent, random
$d_1 \times d_2$ matrices.
Assume that $\EE{\vQ_i} = \v0$ and $\Norm{\vQ_i}\le R$ for each $1\le
i \le m$.
Let $\vS = \sum_{i=1}^m \vQ_i$ and let 
\begin{align*}
v = \max\left\{ \norm{\bbE \textstyle\sum_i \vQ_i \vQ_i^\top  }, 
						      \norm{\bbE \textstyle\sum_i \vQ_i^\top \vQ_i  }
			\right\}\,.
\end{align*}
Then, for all $t\ge 0$, 
\[
\bbP\left( \Norm{\vS} \ge t \right) \le 2(d_1 + d_2) \exp\left( -\frac{t^2/2}{v+R t/3}\right)\,.
\]
In other words, for any $\delta \in (0,1)$,
\[
  \bbP\Parens{
    \norm{\vS} > \sqrt{2 v \ln \frac{2(d_1+d_2)}{\delta}} + \frac{2R}{3}
    \ln\frac{2(d_1+d_2)}{\delta}
  } \leq \delta
  \,.
\]
\end{theorem}

To apply \cref{thm:mxbernstein}, it suffices to bound the spectral
norms of $\vZ_s$ (almost surely), $\bbE(\vZ_s \vZ_s^\t)$, and
$\bbE(\vZ_s^\t \vZ_s)$.

\paragraph{Range bound.}
By the triangle inequality,
\[
  \norm{\vZ_s}
  \leq \norm{\Diag(\vpi)^{-1/2} \vY_s \Diag(\vpi)^{-1/2}}
  + \norm{\Diag(\vpi)^{-1/2} \bbE(\vY_s) \Diag(\vpi)^{-1/2}}
  \,.
\]
For the first term, we have
\begin{align}
\label{eq:vysbound}
  \norm{\Diag(\vpi)^{-1/2} \vY_s \Diag(\vpi)^{-1/2}} \leq \frac1{\pimin} .
\end{align}
For the second term, we use the fact $\norm{\vL}\le 1$ to bound
\begin{align*}
  \norm{\Diag(\vpi)^{-1/2} (\bbE(\vY_s) -\vM)\Diag(\vpi)^{-1/2}}
  & = \norm{ \bigl(\Diag(\vpi^{(H_s)}) \Diag(\vpi)^{-1} - \vI  \bigr) \vL }
  \\
  & \le \norm{ \Diag(\vpi^{(H_s)}) \Diag(\vpi)^{-1} - \vI }\,.
\end{align*}
Then, using~\cref{eq:dvpi-ratio-bound},
\begin{equation}
  \norm{ \Diag(\vpi^{(H_s)}) \Diag(\vpi)^{-1} - \vI }
  \le \frac{(1-\gap)^{a'+2(s-1)a}}{\pimin}
  \le \frac{(1-\gap)^{a}}{\pimin} \le 1\,,
  \label{eq:ysbias}
\end{equation}
where the last inequality follows from the assumption that the
block length $a$ satisfies~\cref{eq:block-length}.
Combining this with
$\norm{\Diag(\vpi)^{-1/2} \vM\Diag(\vpi)^{-1/2}} = \norm{\vL}\le 1$,
it follows that
\begin{align}
\label{eq:evysbound}
\norm{\Diag(\vpi)^{-1/2} \bbE(\vY_s) \Diag(\vpi)^{-1/2}} \le 2
\end{align}
by the triangle inequality.
Therefore, together with~\cref{eq:vysbound}, we obtain the range bound
\[
  \norm{\vZ_s} \leq \frac1{\pimin} + 2
  .
\]

\paragraph{Variance bound.}
We now determine bounds on the spectral norms of $\bbE(\vZ_s
\vZ_s^\t)$ and $\bbE(\vZ_s^\t \vZ_s)$.
Observe that
\begin{align}
  \lefteqn{
    \bbE(\vZ_s\vZ_s^\t)
  } \notag \\
  & =
  \frac1{a^2} \sum_{t \in H_s}
  \bbE\Parens{
    \Diag(\vpi)^{-1/2}
    \ve_{X_t} \ve_{X_{t+1}}^\t 
    \Diag(\vpi)^{-1}
    \ve_{X_{t+1}} \ve_{X_t}^\t
    \Diag(\vpi)^{-1/2}
  }
  \label{eq:var1}
  \\
  & \quad
  + \frac1{a^2} \sum_{ \substack{ t \neq t' \\ t,t'\in H_s} } 
  \bbE\Parens{
    \Diag(\vpi)^{-1/2}
    \ve_{X_t} \ve_{X_{t+1}}^\t 
    \Diag(\vpi)^{-1}
    \ve_{X_{t'+1}} \ve_{X_{t'}}^\t
    \Diag(\vpi)^{-1/2}
  }
  \label{eq:var2}
  \\
  & \quad
  - \Diag(\vpi)^{-1/2}
  \bbE(\vY_s)
  \Diag(\vpi)^{-1} 
  \bbE(\vY_s^\t)
  \Diag(\vpi)^{-1/2}
  .
  \label{eq:var3}
\end{align}
The first sum, \cref{eq:var1}, easily simplifies to the diagonal matrix
\begin{align*}
  \lefteqn{
    \frac1{a^2}
    \sum_{t \in H_s}
    \sum_{i=1}^d \sum_{j=1}^d \Pr(X_t = i, X_{t+1} = j) \cdot
    \frac{1}{\pi_i\pi_j} \ve_i\ve_j^\t\ve_j\ve_i^\t
  } \\
  & =
  \frac1{a^2}
  \sum_{t \in H_s}
  \sum_{i=1}^d \sum_{j=1}^d \pi_i^{(t)} P_{i,j} \cdot
  \frac{1}{\pi_i\pi_j} \ve_i\ve_i^\t
  =
  \frac1{a}
  \sum_{i=1}^d \frac{\pi_i^{(H_s)}}{\pi_i}
  \Parens{ \sum_{j=1}^d \frac{P_{i,j}}{\pi_j} }
   \ve_i \ve_i^\t
  .
\end{align*}
For the second sum, \cref{eq:var2},  a symmetric matrix, consider
\[
  \vu^\t
  \Parens{
    \frac1{a^2} \sum_{ \substack{ t \neq t' \\ t,t'\in H_s} } 
    \bbE\Parens{
      \Diag(\vpi)^{-1/2}
      \ve_{X_t} \ve_{X_{t+1}}^\t 
      \Diag(\vpi)^{-1}
      \ve_{X_{t'+1}} \ve_{X_{t'}}^\t
      \Diag(\vpi)^{-1/2}
    }
  } \vu
\]
for an arbitrary unit vector $\vu$.
By Cauchy-Schwarz and AM/GM, this is bounded from above by
\begin{multline*}
  \frac1{2a^2}
  \sum_{ \substack{ t \neq t' \\ t,t'\in H_s} }
  \biggl[
  \bbE\Parens{
    \vu^\t
    \Diag(\vpi)^{-1/2} \ve_{X_t} \ve_{X_{t+1}}^\t \Diag(\vpi)^{-1}
    \ve_{X_{t+1}} \ve_{X_t}^\t \Diag(\vpi)^{-1/2}
    \vu
  }
  \\
  +
  \bbE\Parens{
    \vu^\t
    \Diag(\vpi)^{-1/2} \ve_{X_{t'}} \ve_{X_{t'+1}}^\t \Diag(\vpi)^{-1}
    \ve_{X_{t'+1}} \ve_{X_{t'}}^\t \Diag(\vpi)^{-1/2}
    \vu
  }
  \biggr]
  ,
\end{multline*}
which simplifies to
\[
  \frac{a-1}{a^2}
  \vu^\t \bbE\Parens{
    \sum_{t \in H_s}
    \Diag(\vpi)^{-1/2} \ve_{X_t} \ve_{X_{t+1}}^\t \Diag(\vpi)^{-1}
    \ve_{X_{t+1}} \ve_{X_t}^\t \Diag(\vpi)^{-1/2}
  } \vu\,.
\]
The expectation is the same as that for the first term,
\cref{eq:var1}.

Finally, the spectral norm of the third term, \cref{eq:var3}, is
bounded using \cref{eq:evysbound}:
\[
  \norm{\Diag(\vpi)^{-1/2} \bbE(\vY_s) \Diag(\vpi)^{-1/2}}^2
  \leq 4
  .
\]

Therefore, by the triangle inequality, the bound $\pi_i^{(H)}/\pi_i
\leq 2$ from \cref{eq:ysbias}, and simplifications, 
\begin{align*}
  \Norm{
    \bbE(\vZ_s \vZ_s^\t)
  }
  &\leq
  \max_{i\in[d]}
  \Parens{ \sum_{j=1}^d \frac{P_{i,j}}{\pi_j} }
  \frac{\pi_i^{(H)}}{\pi_i}
  + 4
  \leq
  2\max_{i\in[d]}
  \Parens{ \sum_{j=1}^d \frac{P_{i,j}}{\pi_j} }
  + 4
  .
\end{align*}

We can bound $\bbE(\vZ_s^\t \vZ_s)$
in a similar way; the only difference is that the reversibility needs
to be used at one place to simplify an expectation:
\begin{align*}
\MoveEqLeft
    \frac1{a^2} \sum_{t \in H_s}
    \bbE\Parens{
      \Diag(\vpi)^{-1/2}
      \ve_{X_{t+1}} \ve_{X_t}^\t
      \Diag(\vpi)^{-1}
      \ve_{X_t}\ve_{X_{t+1}}^\t
      \Diag(\vpi)^{-1/2}
    }
	\\
  & =
  \frac1{a^2} \sum_{t \in H_s}
  \sum_{i=1}^d \sum_{j=1}^d \Pr(X_t = i, X_{t+1} = j) \cdot
  \frac{1}{\pi_i\pi_j} \ve_j\ve_j^\t
  \\
  & =
  \frac1{a^2} \sum_{t \in H_s}
  \sum_{i=1}^d \sum_{j=1}^d \pi_i^{(t)} P_{i,j} \cdot
  \frac{1}{\pi_i\pi_j} \ve_j\ve_j^\t
  \\
  & =
  \frac1{a^2} \sum_{t \in H_s}
  \sum_{j=1}^d \Parens{
    \sum_{i=1}^d \frac{\pi_i^{(t)}}{\pi_i} \cdot \frac{P_{j,i}}{\pi_i}
  } \ve_j\ve_j^\t
\end{align*}
where the last step uses \cref{eq:reversibility}.
As before, we get
\begin{align*}
  \Norm{
    \bbE(\vZ_s^\t\vZ_s)
  }
  &\leq
  \max_{i\in[d]}
  \Parens{
    \sum_{j=1}^d \frac{P_{i,j}}{\pi_j}
    \cdot \frac{\pi_j^{(H)}}{\pi_j}
  }
  + 4
  \leq
  2 \max_{i\in[d]}
  \Parens{
    \sum_{j=1}^d \frac{P_{i,j}}{\pi_j}
  }
  + 4
\end{align*}
again using the bound $\pi_i^{(H)}/\pi_i \leq 2$ from
\cref{eq:ysbias}.

\paragraph{Independent copies bound.}
Let $\wt\vZ_s$ for $s \in [\mu_H]$ be independent copies of
$\vZ_s$ for $s \in [\mu_H]$.
Applying \cref{thm:mxbernstein} to the average of these random
matrices, we have
\begin{equation}
  \bbP\Parens{
    \Norm{ \frac1{\mu_H} \sum_{s=1}^{\mu_H} \wt\vZ_s}
    >
    \sqrt{
      \frac{4\Parens{ d_{\vP} + 2 } \ln\frac{4d}{\delta}}
      {\mu_H}
    }
    + \frac{2\Parens{ \frac1{\pimin} + 2 } \ln\frac{4d}{\delta}}
    {3\mu_H}
  } \leq \delta
  \label{eq:indpt-bound}
\end{equation}
where
\begin{align*}
  d_{\vP}
  & := \max_{i \in [d]} \sum_{j=1}^d \frac{P_{i,j}}{\pi_j}
  \leq
  \frac{1}{\pimin}
  \,.
\end{align*}

\paragraph{The actual bound.}
To bound the probability that $\norm{\sum_{s=1}^{\mu_H} \vZ_s/\mu_H}$
is large, we appeal to the following result, which is a consequence of
\citep[Corollary 2.7]{Yu94}.
For each $s \in [\mu_H]$, let $X^{(H_s)} := (X_t : a' + 2(s-1)a + 1
\leq t \leq a' + (2s-1)a + 1)$, which are the random variables
determining $\vZ_s$.
Let $\bbP$ denote the joint distribution of $(X^{(H_s)} : s \in
[\mu_H])$; let $\bbP_s$ be its marginal over $X^{(H_s)}$, and let
$\bbP_{1:s+1}$ be its marginal over
$(X^{(H_1)},X^{(H_2)},\dotsc,X^{(H_{s+1})})$.
Let $\wt\bbP$ be the product distribution formed from the marginals
$\bbP_1,\bbP_2,\dotsc,\bbP_{\mu_H}$, so $\wt\bbP$ governs the joint
distribution of $(\wt\vZ_s : s \in [\mu_H])$.
The result from \citep[Corollary 2.7]{Yu94} implies for any event $E$,
\[
  |\bbP(E) - \wt\bbP(E)| \leq (\mu_H-1) \beta(\bbP)
\]
where
\[
  \beta(\bbP)
  :=
  \max_{1 \leq s \leq \mu_H-1}
  \bbE\Parens{
    \Norm{
      \bbP_{1:s+1}(\cdot\,|X^{(H_1)},X^{(H_2)},\dotsc,X^{(H_s)}) - \bbP_{s+1}
    }_{\tv}
  }
  \,.
\]
Here, $\norm{\cdot}_{\tv}$ denotes the total variation norm.
The number $\beta(\bbP)$ can be recognized to be the
\emph{$\beta$-mixing coefficient} of the stochastic process
$\{X^{(H_s)}\}_{s \in [\mu_H]}$.
This result implies that the bound from \cref{eq:indpt-bound} for
$\norm{\sum_{s=1}^{\mu_H} \wt\vZ_s/\mu_H}$ also holds for
$\norm{\sum_{s=1}^{\mu_H} \vZ_s/\mu_H}$, except the probability
bound increases from $\delta$ to $\delta + (\mu_H-1)\beta(\bbP)$:
\begin{equation}
  \bbP\Parens{
    \Norm{ \frac1{\mu_H} \sum_{s=1}^{\mu_H} \vZ_s}
    >
    \sqrt{
      \frac{4\Parens{ d_{\vP} + 2 } \ln\frac{4d}{\delta}}
      {\mu_H}
    }
    + \frac{2\Parens{ \frac1{\pimin} + 2 } \ln\frac{4d}{\delta}}
    {3\mu_H}
  } \leq \delta + (\mu_H-1)\beta(\bbP)
  .
  \label{eq:dpt-bound}
\end{equation}
By the triangle inequality,
\[
  \beta(\bbP)
  \leq
  \max_{1 \leq s \leq \mu_H-1}
  \bbE\Parens{
    \vphantom{\bigg\vert}
    \Norm{
      \bbP_{1:s+1}(\cdot\,|X^{(H_1)},X^{(H_2)},\dotsc,X^{(H_s)})
      - \bbP^{\vpi}
    }_{\tv}
    +
    \Norm{
      \bbP_{s+1}
      - \bbP^{\vpi}
    }_{\tv}
  }
\]
where $\bbP^{\vpi}$ is the marginal distribution of $X^{(H_1)}$ under
the stationary chain.
Using the Markov property and integrating out $X_t$ for $t >
\min H_{s+1} = a'+2sa+1$,
\[
  \Norm{
    \bbP_{1:s+1}(\cdot\,|X^{(H_1)},X^{(H_2)},\dotsc,X^{(H_s)})
    - \bbP^{\vpi}
  }_{\tv}
  =
  \Norm{
    \cL(X_{a'+2sa+1}\,|X_{a'+(2s-1)a+1})
    - \vpi
  }_{\tv}
\]
where $\cL(Y|Z)$ denotes the conditional distribution of $Y$ given
$Z$.
We bound this distance using standard arguments for bounding the
mixing time in terms of the \emph{relaxation time}
$1/\gap$~\citep[see, e.g., the proof of Theorem 12.3 of][]{LePeWi08}:
for any $i \in [d]$,
\[
  \Norm{
    \cL(X_{a'+2sa+1}\,|X_{a'+(2s-1)a+1}=i)
    - \vpi
  }_{\tv}
  =
  \Norm{
    \cL(X_{a+1}\,|X_1=i)
    - \vpi
  }_{\tv}
  \leq
  \frac{
    \exp\Parens{ -a\gap }
  }{\pimin}
  .
\]
The distance $\norm{ \bbP_{s+1} - \bbP^{\vpi} }_{\tv}$ can be bounded
similarly:
\begin{align*}
  \Norm{
    \bbP_{s+1}
    - \bbP^{\vpi}
  }_{\tv}
  & =
  \Norm{
    \cL(X_{a'+2sa+1})
    - \vpi
  }_{\tv}
  \\
  & =
  \Norm{
    \sum_{i=1}^d \bbP(X_1 = i)
    \cL(X_{a'+2sa+1}\,|X_1 = i)
    - \vpi
  }_{\tv}
  \\
  & \leq
  \sum_{i=1}^d \bbP(X_1 = i)
  \Norm{
    \cL(X_{a'+2sa+1}\,|X_1 = i)
    - \vpi
  }_{\tv}
  \\
  & \leq
  \frac{
    \exp\Parens{-(a'+2sa)\gap}
  }{\pimin}
  \leq
  \frac{
    \exp\Parens{-a\gap}
  }{\pimin}
  .
\end{align*}
We conclude
\[
  (\mu_H-1)\beta(\bbP)
  \leq (\mu_H-1)\frac{2\exp(-a\gap)}{\pimin}
  \leq \frac{2(n-2)\exp(-a\gap)}{\pimin}
  \leq \delta
\]
where the last step follows from the block length assumption
\cref{eq:block-length}.

We return to the decomposition from \cref{eq:blocking}.
We apply \cref{eq:dpt-bound} to both the $H_s$ blocks and the $T_s$
blocks, and combine with \cref{eq:first-block} to obtain the following
probabilistic bound.
Pick any $\delta \in (0,1)$, let the block length be
\[
  a
  := \ceil{ a_\delta }
  =
  \Ceil{
    \frac{1}{\gap}\ln\frac{2(n-2)}{\pimin\delta}
  }
  ,
\]
so
\[
  \min\{\mu_H,\mu_T\}
  =
  \Floor{
    \frac{n-1-a'}{2a}
  }
  \geq
  \frac{n-1}
  {
    2\Parens{
      1 + \frac1{\gap}\ln\frac{2(n-2)}{\pimin\delta}
    }
  }
  -2
  =: \mu
  .
\]
If
\begin{equation}
  n \geq 7 + \frac6{\gap} \ln \frac{2(n-2)}{\pimin\delta}
  \geq 3a
  ,
  \label{eq:min-n2}
\end{equation}
then with probability at least $1-4\delta$,
\begin{multline*}
  \Norm{\Diag(\vpi)^{-1/2}\Parens{\wh\vM - \bbE[\wh\vM]} \Diag(\vpi)^{-1/2} }
  \\
  \leq
  \frac{4
    \Ceil{
      \frac{1}{\gap}\ln\frac{2(n-2)}{\pimin\delta}
    }
  }{\pimin(n-1)}
  +
  \sqrt{
    \frac{4\Parens{ d_{\vP} + 2 } \ln\frac{4d}{\delta}}
    {\mu}
  }
  + \frac{2\Parens{ \frac1{\pimin} + 2 } \ln\frac{4d}{\delta}}
  {3\mu}
  .
\end{multline*}

\subsubsection{The bound on $\norm{\errm}$}
Combining the probabilistic bound from above with the bound on the
bias from \cref{eq:biasbound}, we obtain the following.
Assuming the condition on $n$ from \cref{eq:min-n2}, with probability
at least $1-4\delta$,
\begin{equation}
  \norm{\errm}
  \leq
  \frac{1}{(n-1)\gap\pimin}
  +
  \frac{4
    \Ceil{
      \frac{1}{\gap}\ln\frac{2(n-2)}{\pimin\delta}
    }
  }{\pimin(n-1)}
  +
  \sqrt{
    \frac{4\Parens{ d_{\vP} + 2 } \ln\frac{4d}{\delta}}
    {\mu}
  }
  + \frac{2\Parens{ \frac1{\pimin} + 2 } \ln\frac{4d}{\delta}}
  {3\mu}
  .
  \label{eq:doubletbound}
\end{equation}

\subsection{Overall error bound}
Assume the sequence length $n$ satisfies \cref{eq:min-n1} and
\cref{eq:min-n2}.
Consider the $1-5\delta$ probability event in which
\cref{eq:singletonboundsimple,eq:hatpimin,eq:doubletbound} hold.
In this event, \cref{eq:singletonbound} also holds, and hence by
\cref{eq:vlsimple},
\begin{multline*}
  \norm{\wh\vL - \vL}
  \leq
  3
  \Parens{
    \sqrt{
      \frac{
        8\ln\Parens{\frac{d}\delta\sqrt{\frac{2}{\pimin}}}
      }{
        \pimin\gap n
      }
    }
    +
    \frac{
      20\ln\Parens{\frac{d}\delta\sqrt{\frac{2}{\pimin}}}
    }{
      \pimin\gap n
    }
  }
  \\
  +
  3\Parens{
    \frac{1}{(n-1)\gap\pimin}
    +
    \frac{4
      \Ceil{
        \frac{1}{\gap}\ln\frac{2(n-2)}{\pimin\delta}
      }
    }{\pimin(n-1)}
    +
    \sqrt{
      \frac{4\Parens{ d_{\vP} + 2 } \ln\frac{4d}{\delta}}
      {\mu}
    }
    + \frac{2\Parens{ \frac1{\pimin} + 2 } \ln\frac{4d}{\delta}}
    {3\mu}
  }
  \\
  =
  O
  \Parens{
    \sqrt{
      \frac{
        \log\Parens{\frac{d}{\delta}}
        \log\Parens{\frac{n}{\pimin\delta}}
      }{\pimin\gap n}
    }
  }
  .
\end{multline*}

To finish the proof of \cref{lem:gap}, we replace $\delta$ with
$\delta/5$, and now observe that the bound on $\norm{\wh\vL-\vL}$ is
trivial if \cref{eq:min-n1} is violated (recalling that
$\norm{\wh\vL-\vL} \leq 2$ always holds).
We tack on an additional term $\log(1/(\pimin\gap\delta)) / (\gap n)$
to also ensure a trivial bound if \cref{eq:min-n2} is violated.
\hfill $\qed$

