%!TEX root =  matrix-est.tex
%\usepackage{etex} % handle MANY packages

% comment this out for the final version, or add the final option:
%\usepackage[notcite,notref]{showkeys}


\usepackage{amsmath,amsbsy,amsfonts,amssymb,amsthm,color,dsfont,mleftright,nccmath}
\usepackage{url}
\usepackage{dsfont} % double stroke font

%%%%%%%%%%%%%%%%%%%%%%%%%%%%%%%%
% PACKAGES
%%%%%%%%%%%%%%%%%%%%%%%%%%%%%%%%
\setlength{\marginparwidth}{25mm}
%\usepackage[textsize=tiny]{todonotes}
\usepackage[disable]{todonotes}
\newcommand{\todoc}[2][]{\todo[color=red!20!white,#1]{Cs: #2}}
\newcommand{\todod}[2][]{\todo[color=blue!20!white,#1]{D: #2}}
\newcommand{\todoa}[2][]{\todo[color=green!20!white,#1]{A: #2}}

\newcommand{\hide}[1]{}

% turn off previously defined macros as we want to use ntheorem (jmlr foolishly defines these)
\if0
\let\proof\relax
\let\endproof\relax
\let\theorem\relax
\let\endtheorem\relax
\let\proposition\relax
\let\endproposition\relax
\let\lemma\relax
\let\endlemma\relax
\let\corollary\relax
\let\endcorollary\relax
\let\example\relax
\let\endexample\relax
\let\remark\relax
\let\endremark\relax
\let\definition\relax
\let\enddefinition\relax
\fi
%\usepackage[amsmath,standard,thmmarks]{ntheorem} % ntheorem makes cleveref work properly 
\newtheorem{lemma}{Lemma}
\newtheorem{claim}{Claim}
\newtheorem{proposition}{Proposition}
\newtheorem{theorem}{Theorem}
\newtheorem{corollary}{Corollary}

\usepackage{latexsym}
%\usepackage{times}
\usepackage{mathtools}
\usepackage{enumerate}
\usepackage[ruled]{algorithm}
\usepackage{algorithmic}
\usepackage{hyperref}
\hypersetup{
    bookmarks=true,         % show bookmarks bar?
    unicode=false,          % non-Latin characters in Acrobat's bookmarks
    pdftoolbar=true,        % show Acrobat's toolbar?
    pdfmenubar=true,        % show Acrobat's menu?
    pdffitwindow=false,     % window fit to page when opened
    pdfstartview={FitH},    % fits the width of the page to the window
    pdftitle={My title},    % title
    pdfauthor={Author},     % author
    pdfsubject={Subject},   % subject of the document
    pdfcreator={Creator},   % creator of the document
    pdfproducer={Producer}, % producer of the document
    pdfkeywords={keyword1} {key2} {key3}, % list of keywords
    pdfnewwindow=true,      % links in new window
    colorlinks=true,       % false: boxed links; true: colored links
    linkcolor=red,          % color of internal links (change box color with linkbordercolor)
    citecolor=blue,        % color of links to bibliography
    filecolor=magenta,      % color of file links
    urlcolor=black          % color of external links
}
%\usepackage[standard]{ntheorem}
%\usepackage{times}
\usepackage[numbers,sort&compress]{natbib}
\usepackage{enumitem}
\usepackage{bibunits}


%\setlist{nolistsep}

%%%%%%%%%%%%%%%%%%%%%%%%%%%%%%%%
% THEOREMS
%%%%%%%%%%%%%%%%%%%%%%%%%%%%%%%%
\if0
\newtheorem{lemma}{Lemma}
\newtheorem{claim}{Claim}
\newtheorem{proposition}{Proposition}
\newtheorem{theorem}{Theorem}
\newtheorem{corollary}{Corollary}
\theoremstyle{remark}
\newtheorem{remark}{Remark}
\theoremstyle{definition}
\newtheorem{definition}{Definition}
\newtheorem{condition}{Condition}
\newtheorem{example}{Example}
\newtheorem{problem}{Problem}
\fi
\usepackage[capitalize]{cleveref}
\usepackage{fullpage}

%%%%%%%%%%%%%%%%%%%%%%%%%%%%%%%%
% MACROS
%%%%%%%%%%%%%%%%%%%%%%%%%%%%%%%%

\def\ddefloop#1{\ifx\ddefloop#1\else\ddef{#1}\expandafter\ddefloop\fi}

% \bA, \bB, ...
\def\ddef#1{\expandafter\def\csname b#1\endcsname{\ensuremath{\mathbf{#1}}}}
\ddefloop ABCDEFGHIJKLMNOPQRSTUVWXYZ\ddefloop

% \bbA, \bbB, ...
\def\ddef#1{\expandafter\def\csname bb#1\endcsname{\ensuremath{\mathbb{#1}}}}
\ddefloop ABCDEFGHIJKLMNOPQRSTUVWXYZ\ddefloop

% \cA, \cB, ...
\def\ddef#1{\expandafter\def\csname c#1\endcsname{\ensuremath{\mathcal{#1}}}}
\ddefloop ABCDEFGHIJKLMNOPQRSTUVWXYZ\ddefloop

% \vA, \vB, ..., \va, \vb, ...
\def\ddef#1{\expandafter\def\csname v#1\endcsname{\ensuremath{\boldsymbol{#1}}}}
\ddefloop ABCDEFGHIJKLMNOPQRSTUVWXYZabcdefghijklmnopqrstuvwxyz\ddefloop

% \valpha, \vbeta, ...,  \vGamma, \vDelta, ...,
\def\ddef#1{\expandafter\def\csname v#1\endcsname{\ensuremath{\boldsymbol{\csname #1\endcsname}}}}
\ddefloop {alpha}{beta}{gamma}{delta}{epsilon}{varepsilon}{zeta}{eta}{theta}{vartheta}{iota}{kappa}{lambda}{mu}{nu}{xi}{pi}{varpi}{rho}{varrho}{sigma}{varsigma}{tau}{upsilon}{phi}{varphi}{chi}{psi}{omega}{Gamma}{Delta}{Theta}{Lambda}{Xi}{Pi}{Sigma}{varSigma}{Upsilon}{Phi}{Psi}{Omega}\ddefloop

\newcommand\Sig{\ensuremath{\varSigma}}
\newcommand\veps{\ensuremath{\varepsilon}}
\newcommand\eps{\ensuremath{\epsilon}}

\renewcommand\t{{\ensuremath{\scriptscriptstyle{\top}}}}

\DeclareMathOperator{\tr}{tr}
\DeclareMathOperator{\diag}{diag}
\DeclareMathOperator{\Diag}{Diag}
\DeclareMathOperator{\rank}{rank}
\DeclareMathOperator{\sign}{sign}
\DeclareMathOperator{\supp}{supp}
\DeclareMathOperator{\vol}{vol}

\DeclareMathOperator{\var}{var}
\DeclareMathOperator{\Var}{Var}

\DeclareMathOperator{\Bd}{bd}
\DeclareMathOperator{\Cl}{cl}
\DeclareMathOperator{\Conv}{conv}
\DeclareMathOperator{\Int}{int}
\DeclareMathOperator{\Null}{ker}
\DeclareMathOperator{\Span}{span}
\DeclareMathOperator{\Range}{ran}

\DeclareMathOperator*{\argmin}{arg\,min}
\DeclareMathOperator*{\argmax}{arg\,max}

\newcommand\wt{\ensuremath{\widetilde}}
\newcommand\wh{\ensuremath{\widehat}}
\renewcommand\v{\ensuremath{\boldsymbol}}

\newcommand\comment[1]{{\color{blue}\{\textbf{Comment}: #1\}}}

\newcommand\parens[1]{(#1)}
\newcommand\norm[1]{\|#1\|}
\newcommand\braces[1]{\{#1\}}
\newcommand\brackets[1]{[#1]}
\newcommand\ceil[1]{\lceil#1\rceil}
\newcommand\abs[1]{|#1|}
\newcommand\ind[1]{\ensuremath{\mathds{1}\{#1\}}}
\newcommand\dotp[1]{\langle #1 \rangle}

\newcommand\Parens[1]{\left(#1\right)}
\newcommand\Norm[1]{\left\|#1\right\|}
\newcommand\Braces[1]{\left\{#1\right\}}
\newcommand\Brackets[1]{\left[#1\right]}
\newcommand\Ceil[1]{\left\lceil#1\right\rceil}
\newcommand\Floor[1]{\left\lfloor#1\right\rfloor}
\newcommand\Abs[1]{\left|#1\right|}
\newcommand\Ind[1]{\ensuremath{\mathds{1}\left\{#1\right\}}}
\newcommand\Dotp[1]{\left\langle #1 \right\rangle}

\newcommand\pimin{\ensuremath{\pi_{\star}}}
\newcommand\gap{\ensuremath{\gamma_{\star}}}
\newcommand\slem{\ensuremath{\lambda_{\star}}}

\newcommand\hatpimin{\ensuremath{\hat\pi_{\star}}}
\newcommand\hatgap{\ensuremath{\hat\gamma_{\star}}}

\newcommand\Geom{\ensuremath{\operatorname{Geom}}}
\newcommand\Sym{\ensuremath{\operatorname{Sym}}}
\newcommand\errm{\ensuremath{\v{\cE}_{\vM}}}
\newcommand\errp{\ensuremath{\v{\cE}_{\vpi}}}
\newcommand\errP{\ensuremath{\v{\cE}_{\vP}}}
\newcommand\errpil{\ensuremath{\v{\cE}_{\vpi,1}}}
\newcommand\errpir{\ensuremath{\v{\cE}_{\vpi,2}}}
\newcommand\tv{\ensuremath{\operatorname{tv}}}
\newcommand\tmix{\ensuremath{t_{\operatorname{mix}}}}
\newcommand\trelax{\ensuremath{t_{\operatorname{relax}}}}
\newcommand\avg{\ensuremath{\operatorname{avg}}}
\newcommand{\iset}[1]{[#1]}
\newcommand{\opencloseint}[1]{\ensuremath{(#1]}}

\newcommand{\Dvpi}{\Diag(\vpi)}
\newcommand{\Dvpit}{\Diag(\vpi^{(t)})}

\newcommand{\EE}[1]{\bbE\left(#1\right)}

\newcommand\giAh{\ensuremath{\widehat{\boldsymbol{A}}^{\raisebox{-2.9pt}{$\scriptstyle\#$}}}}
\newcommand\giA{\ensuremath{\boldsymbol{A}^{\raisebox{-0pt}{$\scriptstyle\#$}}}}
\newcommand{\defeq}{:=}

\newcommand\emptail{\ensuremath{\tau_{n,\delta}}}

