
\begin{proof}[Proof of \cref{thm:lb-pimin}]
  Fix $\bar\pi \in (0,1/4)$.
  Consider two Markov chains given by the following stochastic
  matrices:
  \[
    \vP^{(1)} :=
    \begin{bmatrix}
      1-\bar\pi & \bar\pi \\
      1-\bar\pi & \bar\pi
    \end{bmatrix}
    , \quad
    \vP^{(2)} :=
    \begin{bmatrix}
      1-\bar\pi & \bar\pi \\
      1/2 & 1/2
    \end{bmatrix}
    .
  \]
  Each Markov chain is ergodic and reversible; their stationary
  distributions are, respectively, $\vpi^{(1)} = (1-\bar\pi,\bar\pi)$
  and $\vpi^{(2)} = (1/(1+2\bar\pi),2\bar\pi/(1+2\bar\pi))$.
  We have $\pimin \geq \bar\pi$ in both cases.
  For the first Markov chain, $\slem = 0$, and hence the spectral gap
  is $1$; for the second Markov chain, $\slem = 1/2-\bar\pi$, so the
  spectral gap is $1/2+\bar\pi$.

  In order to guarantee $|\hatgap - \gap| < 1/8 < |1 -
  (1/2+\bar\pi)|/2$, it must be possible to distinguish the two Markov
  chains.
  Assume that the initial state distribution has mass at least $1/2$
  on state $1$.
  (If this is not the case, we swap the roles of states $1$ and $2$ in
  the constructions above.)
  With probability at least half, the initial state is $1$; and both
  chains have the same transition probabilities from state $1$.
  The chains are indistinguishable less the sample path eventually
  reaches state $2$.
  But with probability at least $1/3$, a sample path of length $n <
  1/(4\bar\pi)$ starting from state $1$ always remains in the same
  state (this follows from properties of the geometric distribution).
\end{proof}


\begin{proof}[Proof of \cref{thm:lb-gap}]
  We consider $d$-state Markov chains of the following form:
  \[
    P_{i,j} =
    \begin{cases}
      1-\veps_i & \text{if $i = j$} ; \\
      \displaystyle\frac{\veps_i}{d-1} & \text{if $i \neq j$}
    \end{cases}
  \]
  for some $\veps_1, \veps_2, \dotsc, \veps_d \in (0,1)$.
  Such a chain is ergodic and reversible, and its unique stationary
  distribution $\vpi$ satisfies
  \[
    \pi_i = \frac{1/\veps_i}{\sum_{j=1}^d 1/\veps_j}
    .
  \]
  We fix $\veps := \frac{d-1}{d/2}\bar\gamma$ and set $\veps' :=
  \frac{d/2-1}{d-1} \veps < \veps$.
  Consider the following $d+1$ different Markov chains of the type
  described above:
  \begin{itemize}
    \item
      $\vP^{(0)}$: $\veps_1 = \dotsb = \veps_d = \veps$.
      For this Markov chain, $\lambda_2 = \lambda_d = \slem =
      1-\frac{d}{d-1}\veps$.

    \item
      $\vP^{(i)}$ for $i \in [d]$: $\veps_j = \veps$ for $j \neq i$,
      and $\veps_i = \veps'$.
      For these Markov chains, $\lambda_2 = 1 - \veps' -
      \frac{1}{d-1}\veps = 1 - \frac{d/2}{d-1} \veps$,
      and $\lambda_d = 1 - \frac{d}{d-1} \veps$.
      So $\slem = 1-\frac{d/2}{d-1} \veps$.

  \end{itemize}
  The spectral gap in each chain satisfies $\gap \in
  [\bar\gamma,2\bar\gamma]$; in $\vP^{(i)}$ for $i \in [d]$, it is half
  of what it is in $\vP^{(0)}$.
  Also $\pi_i \geq 1/(2d)$ for each $i \in [d]$.

  In order to guarantee $|\hatgap - \gap| \leq \bar\gamma$, it must
  be possible to distinguish $\vP^{(0)}$ from each $\vP^{(i)}$,
  $i\in[d]$.
  But $\vP^{(0)}$ is identical to $\vP^{(i)}$ except for the transition
  probabilities from state $i$.
  Therefore, regardless of the initial state, the sample path must
  visit all states in order to distinguish $\vP^{(0)}$ from each
  $\vP^{(i)}$, $i \in [d]$.
  For any of the $d+1$ Markov chains above, the earliest time in which
  a sample path visits all $d$ states
  stochastically dominates a generalized coupon collection time $T = 1 +
  \sum_{i=1}^{d-1} T_i$, where $T_i$ is the number of steps required to
  see the $(i+1)$-th distinct state in the sequence beyond the first
  $i$.
  The random variables $T_1,T_2,\dotsc,T_{d-1}$ are independent, and are
  geometrically distributed, $T_i \sim \Geom(\veps -
  (i-1)\veps/(d-1))$.
  We have that
  \[
    \bbE[T_i] = \frac{d-1}{\veps(d-i)} , \quad
    \var(T_i) = \frac{1 - \veps \frac{d-i}{d-1}}{%
      \Parens{ \veps \frac{d-i}{d-1} }^2
    }
    .
  \]
  Therefore
  \[
    \bbE[T] = 1 + \frac{d-1}{\veps} H_{d-1} , \quad
    \var(T) \leq \Parens{ \frac{d-1}{\veps} }^2 \frac{\pi^2}{6} 
  \]
  where $H_{d-1} = 1 + 1/2 + 1/3 + \dotsb + 1/(d-1)$.
  By the Paley-Zygmund inequality,
  \[
    \Pr\Parens{ T > \frac13\bbE[T] }
    \geq \frac{1}{1 + \frac{\var(T)}{(4/9)\bbE[T]^2}}
    \geq \frac{1}{1 + \frac{\Parens{ \frac{d-1}{\veps} }^2
    \frac{\pi^2}{6}}{(4/9)\Parens{ \frac{d-1}{\veps} H_2}^2}}
    \geq \frac14
    .
  \]
  Since $n < c d\log(d) / \bar\gamma \leq (1/3) (1 + d H_{d-1} /
  (2\bar\gamma)) = \bbE[T]/3$ (for an appropriate absolute constant
  $c$), with probability at least $1/4$, the sample path does not
  visit all $d$ states.
\end{proof}


